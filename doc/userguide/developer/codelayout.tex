\section{Code Layout}
\label{sec:code:layout}

The PyLith software suite is composed of a C++ library, Python
modules, a Python application, and a few Python preprocessing and
post-processing utilities.

\subsection{Directory Structure}

\begin{description}
\item[\filename{applications}] Top-level Python application and
  utility drivers.
\item[\filename{libsrc}] C++ source for PyLith library.
\item[\filename{modulesrc}] SWIG interface files for C++ Python
  bindings.
\item[\filename{pylith}] Python source code for PyLith Python modules.
\item[\filename{doc}] Documentation.
\item[\filename{examples}] PyLith example suite.
\item[\filename{share}] Useful settings and configuration files.
\item[\filename{unittests}] Python and C++ unit tests source
  files. Run using \filename{make check}.
\item[\filename{tests\_auto}] Full-scale tests. Run using
  \filename{make check}.
\item[\filename{tests}] Full-scale tests that require manual checking
  of results.
\item[\filename{travis}] Helper scripts for building PyLith in
  Travis-CI.
\item[\filename{m4}] Autoconf macros (link to
  geodynamics/autoconfig Git repository).
\item[\filename{playpen}] Scratch area (obsolete). Use branches for
  scratch work instead.
\end{description}

We use the Pyre Python framework to collect all user parameters and to
launch the MPI application. As a result, the top-level code is written
in Python. In most cases there is a low-level C++ object of the same
name with the low-level implementation of the object. We limit the
Python code to collection of the user parameters, some simple
checking of the parameters, and passing the parameters to the
corresponding C++ objects.

\subsection{Build}

We strongly recommend using the PyLith Installer utility (see Section
\vref{sec:pylith:installer}) to build PyLith and its dependencies from
source.

\todo{Hackathon}{Hackathon participants should {\em first} build using
  the PyLith master branch. This corresponds to PyLith v2.2.1 with
  some very minor updates to keep the code in sync with
  PETSc. Configure the installer with
  \commandline{-{}-with-pylith-git=master}. See the \filename{INSTALL}
  file in the PyLith Installer source directory for additional
  instructions, including how to build in a separate directory from
  the source code. Those working on poroelasticity and infinite medium
  boundary condition (BIEM/FEM coupling) will be using different
  branches at the hackathon.}

\warning{The PyLith Installer utility has the configure option
  \commandline{-{}-with-pylith-git=BRANCH} to specify which branch to
  use in the main Git repository, but it does not have a configure
  option to specify a different Git repository (e.g., a user
  fork). When using a fork, we recommend using the PyLith Installer
  utility to build the dependencies, and then configuring and building
  the PyLith fork manually.}

\tip{For all development, we build with debugging turned on. By
  building in directories separate from the source code, we can build
  an optimized version for production runs in a different directory and
  use environment variables to select the desired build.}

\subsubsection{Overview}

Pylith uses the GNU Build System (autotools: autoconf, automake,
libtool) to configure and build. The configure settings are defined in
\filename{configure.ac} with additional macros in the \filename{m4}
directory. Note that the \filename{m4} directory is Git submodule
corresponding to the \filename{geodynamics/autoconf} Git repository.

The principal developers of PyLith use platform independent tools when
developing PyLith. The tools include Emacs and Atom.io for editing
source code, gdb and lldb for debugging, valgrind for checking for
memory errors, and gcov for examining code coverage of tests.

\subsubsection{Makefiles}

The PyLith \filename{configure} script uses automake to convert each
\filename{Makefile.am} file into a \filename{Makefile}. The
organization and content of the \filename{Makefile.am} file depends on
whether it is related to the C++ library, SWIG interface files, Python
modules, C++ unit tests, Python unit tests, or examples.

For the C++ library files within the \filename{libsrc} directory, the
\filename{libsrc/Makefile.am} contains the implementation files while
the header files are listed in the \filename{Makefile.am} file within
the underlying directories.

For the SWIG interface files within the \filename{modulesrc}
directory, the \filename{Makefile.am} file contains information on how
to build the SWIG Python module.

The Python modules use a single \filename{Makefile.am} file in
the \filename{pylith} directory.

Each test suite (C++ unit tests for each subpackage, Python unit tests
for each subpackage, and each suite of full-scale tests) use a single
\filename{Makefile.am}. These files define how the tests are built,
additional input files that should be included in the source
distribution, and temporary files that should be deleted.

Each suite of examples contains a \filename{Makefile.am} that defines
the files that are to be included in the source distribution. It also
defines which files are created and should be deleted upon
\commandline{make clean}.

\subsubsection{Build Targets}

Several build targets are defined to make it easy to rebuild/reinstall
PyLith rerun tests whenever the source code changes. Each target can
be run using \commandline{make TARGET} where \commandline{TARGET} is
one of the following:

\begin{description}
\item[all] Build.
\item[install] Build and install.
\item[check] Run all unit tests and full-scale automated tests.
\end{description}

\tip{On a machine with multiple cores, faster builds of the C++ code
(library and C++ unit tests) are available using the \commandline{-j
  NTHREADS} command line argument, where \commandline{NTHREADS} is the
number of cores to use.}

\subsection{PyLith Application Flow}

The PyLith application driver performs two main functions. First, it
collects all user parameters from input files (e.g., \filename{.cfg}
files) and the command line and performs from simple checks on the
parameters. Second, it launches the MPI job.

Once the job starts, the application flow is:
\begin{enumerate}
\item Read the finite-element mesh; \object{pylith/meshio/MeshImporter}.
  \begin{enumerate}
  \item Read the mesh (serial); \object{libsrc/meshio/MeshIO}.
  \item Reorder the mesh; \object{libsrc/topology/ReverseCuthillMcKee}.
  \item Insert cohesive cells as necessary (serial); \object{libsrc/topology/FaultCohesive}.
  \item Distribute the mesh across processes (parallel); \object{libsrc/topology/Distributor}.
  \item Refine the mesh if desired (parallel); \object{libsrc/topology/RefineUniform}.
  \end{enumerate}
\item Setup the problem.
  \begin{enumerate}
  \item Preinitialize the problem by passing information from Python
    to C++ and doing minimal setup; \object{pylith.Problem.preinitialize()}.
  \item Perform consistency checks and additional checks of user
    parameters; \object{pylith.Problem.verifyConfiguration()}.
  \item Complete initialization of the problem;
    \object{pylith::problems::Problem::initialize()}.
  \end{enumerate}
\item Run the problem; \object{pylith.problems.Problem.run()}.
\item Cleanup; \object{pylith.problems.Problem.finalize()}.
  \begin{enumerate}
  \item Close output files.
  \item Deallocate memory.
  \item Output PETSc log summary if desired.
  \end{enumerate}
\end{enumerate}

In the first step, we list the object performing the work, whereas in
subsequent steps we list the top-level object method responsible for
the work. Note that a child class may redefine or perform additional
work compared to what is listed in the parent class method.

\subsubsection{Time-Dependent Problem}

In a time-dependent problem, the PETSc \object{TS} object (relabeled
\object{PetscTS} within PyLith) controls the time stepping. At the
beginning of each time step, the \object{PetscTS} object calls
\object{problems::TimeDependent::prestep()}, and at the end of each
time step, it calls \object{problems::TimeDependent::poststep()}.

Within each time step, the \object{PetscTS} object calls the PETSc
linear and nonlinear solvers as needed, which call the following
methods of the C++ \object{pylith::problems::TimeDependent} object as
needed
\object{computeRHSResidual()},
\object{computeRHSJacobian()},
\object{computeLHSResidual()}, and
\object{computeLHSJacobian()}.

\section{Adding New Governing Equations and/or Materials}

\todo{brad}{Complete this section before the hackathon.}

\begin{itemize}
\item Pointwise functions
\item Solution subfields
\item Auxiliary subfields
\item Constants
\end{itemize}

\subsection{Python}

\begin{itemize}
\item Define solution subfields
\item Define auxiliary subfields
\item Flags to turn on/off terms in governing equation.
\end{itemize}

\subsection{C++}

\begin{itemize}
\item IntegratorPointwise methods
\item Setting pointwise functions for residual, Jacobian, updating
  state variables, and calculating derived fields.
\item Setting up auxiliary subfields.
\end{itemize}
  
\subsection{C++ Unit Tests}

\subsection{Python Unit Tests}

\section{Adding New Boundary Conditions}

\todo{brad}{Complete this section before the hackathon.}

\begin{itemize}
\item Pointwise functions
\item Auxiliary subfields
\end{itemize}

\subsection{Python}

\begin{itemize}
\item Define auxiliary subfields
\item Flags to turn on/off terms.
\end{itemize}

\subsection{C++}

\begin{itemize}
\item IntegratorPointwise or ConstraintPointwise methods
\item Setting pointwise functions for residual, Jacobian, updating
  state variables, and calculating derived fields.
\item Setting up auxiliary subfields.
\end{itemize}

\subsection{C++ Unit Tests}

\subsection{Python Unit Tests}


\section{Adding Support for New Mesh File Formats}

\brad{Future: Add stuff here.}
