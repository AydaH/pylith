\chapter{Multiphysics Finite-Element Formulation}
\label{cha:multiphysics:formulation}

This chapter will become part of the governing equations chapter in
the PyLith Manual.

\section{General Finite-Element Formulation}

Within the PETSc solver framework, we want to solve a system of
partial differential equations in which the strong form can be
expressed as $F(t,s,\dot{s}) = G(t,s)$, $s(t_0) = s_0$, where $F$ and
$G$ are vector functions, $t$ is time, and $s$ is the solution vector.

Using the finite-element method we manipulate the weak form of the
system of equations involving a vector field $\vec{u}$ into integrals
over the domain $\Omega$ with the form,
\begin{equation}
  \label{eqn:problem:form}
  \int_\Omega \trialvec[u] \cdot \vec{f}_0(t,s,\dot{s}) + \nabla \trialvec[u] : \tensor{f}
_1(t,s,\dot{s}) \, 
d\Omega =
  \int_\Omega \trialvec[u] \cdot \vec{g}_0(t,s) + \nabla \trialvec[u] : \tensor{g}_1(t,s) \, 
d\Omega,
\end{equation}
where $\trialvec[u]$ is the trial function for field $\vec{u}$,
$\vec{f}_0$ and $\vec{g}_0$ are vectors, and $\tensor{f}_1$ and
$\tensor{g}_1$ are tensors. With multiple partial differential
equations we will have multiple equations of this form, and the
solution vector $s$, which we usually write as $\vec{s}$, will be
composed of several different fields, such as displacement $\vec{u}$,
velocity $\vec{v}$, pressure $p$, and temperature $T$.

For consistency with the PETSc time stepping formulation, we call
$G(t,s)$ the RHS function and call $F(t,s,\dot{s})$ the LHS (or I)
function. Likewise, the Jacobian of $G(t,s)$ is the RHS Jacobian and
the Jacobian of $F(t,s,\dot{s})$ is the LHS Jacobian. In most cases,
we can take $F(t,s,\dot{s}) = \dot{s}$, or as close to this as
possible. This results in miminal changes to the formulation in order
to accommodate both implicit and explicit time stepping algorithms.

Using a finite-element discretization we break up the domain and
boundary integrals into sums over the cells and boundary faces/edges,
respectively. Using numerical quadrature those sums in turn involve
sums over the values at the quadrature points with appropriate
weights. Thus, computation of the RHS function boils down to
point-wise evaluation of $\vec{g}_0(t,s)$ and $\tensor{g}_1(t,s)$, and
computation of the LHS function boils down to point-wise evaluation of
$\vec{f}_0(t,s,\dot{s})$ and $\tensor{f}_1(t,s,\dot{s})$.

\subsection{Jacobian}

The LHS Jacobian $J_F$ is $\frac{\partial F}{\partial s} +
s_\mathit{tshift} \frac{\partial F}{\partial \dot{s}}$ and the RHS
Jacobian $J_G$ is $\frac{\partial G}{\partial s}$, where
$s_\mathit{tshift}$ arises from the temporal discretization . We put
the Jacobians for each equation in the form:
\begin{align}
  \label{eqn:jacobian:form}
  J_F &= \int_\Omega \trialvec \cdot \tensor{J}_{f0}(t,s,\dot{s}) \cdot \basisvec
  + \trialvec \cdot \tensor{J}_{f1}(t,s,\dot{s}) : \nabla \basisvec
  + \nabla \trialvec : \tensor{J}_{f2}(t,s,\dot{s}) \cdot \basisvec
  + \nabla \trialvec : \tensor{J}_{f3}(t,s,\dot{s}) : \nabla \basisvec \, d\Omega \\
%
  J_G &= \quad \int_\Omega \trialvec \cdot \tensor{J}_{g0}(t,s) \cdot \basisvec
  + \trialvec \cdot \tensor{J}_{g1}(t,s) : \nabla \basisvec
  + \nabla \trialvec : \tensor{J}_{g2}(t,s) \cdot \basisvec
  + \nabla \trialvec : \tensor{J}_{g3}(t,s) : \nabla \basisvec \, d\Omega,
\end{align}
where $\basisvec$ is a basis function.  Expressed in index notation
the Jacobian coupling solution field components $s_i$ and $s_j$ is
\begin{equation}
\label{eqn:jacobian:index:form}
J^{s_is_j} = \int_\Omega \trialscalar_i J_0^{s_is_j} \basisscalar_j + \trialscalar_i 
J_1^{s_js_jl} 
\basisscalar_{j,l} + \trialscalar_{i,k} J_2^{s_is_jk} \basisscalar_j + \trialscalar_{i,k} 
J_3^{s_is_jkl} 
\basisscalar_{j,l} \, d\Omega, 
\end{equation}
It is clear that the tensors $J_0$, $J_1$, $J_2$, and $J_3$ have
various sizes: $J_0(n_i,n_j)$, $J_1(n_i,n_j,d)$, $J_2(n_i,n_j,d)$,
$J_3(n_i,n_j,d,d)$, where $n_i$ is the number of components in
solution field $s_i$, $n_j$ is the number of components in solution
field $s_j$, and $d$ is the spatial dimension.  Alternatively,
expressed in discrete form, the Jacobian for the coupling between
solution fields $s_i$ and $s_j$ is
\begin{equation}
  \label{eqn:jacobian:discrete:form}
  J^{s_is_j} = J_{0}^{s_is_j} + J_{1}^{s_is_j} B + B^T J_{2}^{s_is_j} + B^T J_{3}^{s_is_j} B,
\end{equation}
where $B$ is a matrix of the derivatives of the basis functions and $B^T$
is a matrix of the derivatives of the trial functions. 

\important{See
  \url{http://www.mcs.anl.gov/petsc/petsc-master/docs/manualpages/DT/PetscFEIntegrateJacobian.html}
  for the ordering of indicies in the Jacobian pointwise functions.}

\brad{I think it is very important that we have a way to control
  allocation of the sparse matrix. We do not want to allocate portions
  that are not coupled, because it is way too much memory. A simple
  way to do this would be to create an array that is \#fields x
  \#fields and have the materials populate it with values to indicate
  whether they couple those fields or not. We could use a value to
  indicate if the Jacobian was diagonal or not as well.}

\subsection{PETSc TS Notes}

\begin{itemize}
\item If no LHS (or I) function is given, then the PETSc TS assumes $F(t,s,\dot{s}) = \dot{s}
$.
\item If no RHS function is given, then the PETSc TS assumes $G(t,s) = 0$.
\item Explicit time stepping with the PETSc TS requires
  $F(t,s,\dot{s}) = \dot{s}$.
  \begin{itemize}
  \item Because $F(t,s,\dot{s}) = \dot{s}$, we do not specify the
    functions $\vec{f}_0(t,s,\dot{s})$ and $\tensor{f}_1(t,s,\dot{s})$
    because the PETSc TS will assume this is the case if no LHS (or I)
    function is given.
  \item We also do not specify $J_F$ or $J_G$.
  \item This leaves us with only needing to specify $\vec{g}_0(t,s)$
    and $\tensor{g}_1(t,s)$. 
  \item The PETSc TS will verify that the LHS (or I) function is null.
  \end{itemize}
\end{itemize}

\subsection{Explicit Time Stepping}

For explicit time stepping with the PETSc TS, we need $F(t,s,\dot{s})
= \dot{s}$. Using a finite-element formulation, $F(t,s,\dot{s})$
generally involves integrals of the inertia over the domain. It is
tempting to simply move these terms to the RHS, but that introduces
inertial terms into the boundary conditions, which makes them less
intuitive. Instead, we transform our equation into the form
$F^*(t,s,\dot{s}) = \dot{s} = G^*(t,s)$ by writing $F(t,s,\dot{s}) = M
\dot{s}$, so that $\dot{s} = M^{-1} G(t,s) = G^*(t,s)$. We take $M$ to
be a lumped (diagonal) matrix, so that $M^{-1}$ is trivial to
compute. In computing the RHS function, $G^*(t,s)$, we compute
$G(t,s)$, then compute $M$ and $M^{-1}$, and then $M^{-1}G(t,s)$. For
the elasticity equation with inertial terms, $M$ contains the mass
matrix.

% ----------------------------------------------------------------------
\section{Elasticity With Infinitesimal Strain and No Faults}

We begin with the elasticity equation including the inertial term,
\begin{gather}
  \label{eqn:elasticity:strong:form}
  \rho \frac{\partial^2\vec{u}}{\partial t^2} - \vec{f}(\vec{x},t) - \tensor{\nabla} \cdot 
\tensor{\sigma}
(\vec{u}) = \vec{0} \text{ in }\Omega, \\
%
  \label{eqn:bc:Neumann}
  \tensor{\sigma} \cdot \vec{n} = \vec{\tau}(\vec{x},t) \text{ on }\Gamma_\tau, \\
%
  \label{eqn:bc:Dirichlet}
  \vec{u} = \vec{u}_0(\vec{x},t) \text{ on }\Gamma_u,
\end{gather}
where $\vec{u}$ is the displacement vector, $\rho$ is the mass
density, $\vec{f}$ is the body force vector, $\tensor{\sigma}$ is the
Cauchy stress tensor, $\vec{x}$ is the spatial coordinate, and $t$ is
time. We specify tractions $\vec{\tau}$ on boundary $\Gamma_\tau$, and
displacements $\vec{u}_0$ on boundary $\Gamma_u$. Because both $\vec{\tau}$
and $\vec{u}$ are vector quantities, there can be some spatial overlap
of boundaries $\Gamma_\tau$ and $\Gamma_u$; however, a degree of freedom at
any location cannot be associated with both prescribed displacements
(Dirichlet) and traction (Neumann) boundary conditions simultaneously.

\subsection{Notation}
\begin{itemize}
\item Unknowns
  \begin{description}
  \item[$\vec{u}$] Displacement field
  \item[$\vec{v}$] Velocity field (if including inertial term)
  \end{description}
\item Derived quantities
  \begin{description}
    \item[$\tensor{\sigma}$] Stress tensor
    \item[$\tensor{\epsilon}$] Strain tensor
  \end{description}
\item Constitutive parameters
  \begin{description}
  \item[$\mu$] Shear modulus
  \item[$K$] Bulk modulus
  \item[$\rho$] Density
  \end{description}
\item Source terms
  \begin{description}
    \item[$\vec{f}$] Body force per unit volume, for example $\rho \vec{g}$
  \end{description}
\end{itemize}

\subsection{Neglecting Inertia}

If we neglect the inertial term, then time dependence only arises
from history-dependent constitutive equations and boundary
conditions. Considering the displacement $\vec{u}$ as the unknown, we
have
\begin{align}
  \vec{s}^T &= (\vec{u})^T, \\
%
  \vec{0} &= \vec{f}(\vec{x},t) + \tensor{\nabla} \cdot \tensor{\sigma}(\vec{u}) \text{ in }
\Omega, \\
% Neumann
  \tensor{\sigma} \cdot \vec{n} &= \vec{\tau}(\vec{x},t) \text{ on }\Gamma_\tau, \\
% Dirichlet
  \vec{u} &= \vec{u}_0(\vec{x},t) \text{ on }\Gamma_u.
\end{align}
We create the weak form by taking the dot product with the trial
function $\trialvec[u]$ and integrating over the domain:
\begin{equation}
  0 = \int_\Omega \trialvec[u] \cdot \left( \vec{f}(t) + \tensor{\nabla} \cdot \tensor{\sigma}
(\vec{u})  \right) 
\, d\Omega.
\end{equation}
Using the divergence theorem and incorporating the Neumann bounday
condition yields
\begin{equation}
  0 = \int_\Omega \trialvec[u] \cdot \vec{f}(t) + \nabla \trialvec[u] : -\tensor{\sigma}
(\vec{u}) \, d\Omega + 
\int_{\Gamma_\tau} \trialvec[u] \cdot \vec{\tau}(\vec{x},t) \, d\Gamma.
\end{equation}

Identifying $F(t,s,\dot{s})$ and $G(t,s)$, we have
\begin{alignat}{2}
  F^u(t,s,\dot{s}) &= \vec{0},
  & \qquad
  G^u(t,s) &= \int_\Omega \trialvec[u] \cdot \eqnannotate{\vec{f}(\vec{x},t)}{g_0^u} + \nabla 
\trialvec[u] : 
\eqnannotate{-\tensor{\sigma}(\vec{u})}{g_1^u} \, d\Omega + \int_{\Gamma_\tau} \trialvec[u] 
\cdot 
\eqnannotate{\vec{\tau}(\vec{x},t)}{g_0^u} \, d\Gamma.
\end{alignat}


\subsubsection{Jacobians}

With the solution composed of the displacement field and no LHS function, we only have 
Jacobians for the RHS,
\begin{align}
  J_G^{uu} &= \frac{\partial G^u}{\partial u} = \int_\Omega \nabla \trialvec[u] : 
\frac{\partial}{\partial u}(-
\tensor{\sigma}) \, d\Omega 
  = \int_\Omega \nabla \trialvec[u] : -\tensor{C} : \frac{1}{2}(\nabla + \nabla^T)\basisvec[u] 
\, d\Omega 
  = \int_\Omega \trialscalar[v]_{i,k} \, \eqnannotate{\left( -C_{ikjl} \right)}{J_{g3}^{uu}}  
\, 
\basisscalar[u]_{j,l}\, d\Omega
\end{align}

\subsection{Including Inertia}

For convenience we cast the elasticity equation in the form of a first order
equation by considering both the displacement $\vec{u}$ and velocity $\vec{v}$
as unknowns,
\begin{align}
  \vec{s}^T &= (\vec{u} \quad \vec{v})^T, \\
%
  \label{eqn:velocity:strong:form}
  \frac{\partial\vec{u}}{\partial t} &= \vec{v}, \\
%
  \label{eqn:elasticity:order1:strong:form}
  \rho \frac{\partial\vec{v}}{\partial t} &= \vec{f}(\vec{x},t) + \tensor{\nabla} \cdot 
\tensor{\sigma}(\vec{u}) 
\text{ in }\Omega, \\
% Neumann
  \tensor{\sigma} \cdot \vec{n} &= \vec{\tau}(\vec{x},t) \text{ on }\Gamma_\tau, \\
% Dirichlet
  \vec{u} &= \vec{u}_0(\vec{x},t) \text{ on }\Gamma_u.
\end{align}

For trial functions $\trialvec[u]$ and $\trialvec[v]$ we write the weak form as
\begin{align}
  \int_\Omega \trialvec[u] \cdot \left( \frac{\partial \vec{u}}{\partial t} \right) \, d\Omega 
&= 
  \int_\Omega \trialvec[u] \cdot \vec{v} \, d\Omega, \\
%
  \int_\Omega \trialvec[v] \cdot \left( \rho \frac{\partial \vec{v}}{\partial t} \right) \, 
d\Omega &= 
  \int_\Omega \trialvec[v] \cdot \left( \vec{f}(\vec{x},t) + \tensor{\nabla} \cdot 
\tensor{\sigma}(\vec{u})  
\right) \, d\Omega.
%
\end{align}
Using the divergence theorem and incorporating the Neumann boundary
conditions, we can rewrite the second equation as
\begin{equation}
  \label{eqn:elasticity:displacement}
  \int_\Omega \trialvec[v] \cdot \left( \rho \frac{\partial \vec{v}}{\partial t} \right) \, 
d\Omega =
  \int_\Omega \trialvec[v] \cdot \vec{f}(\vec{x},t) + \nabla \trialvec[v] : -\tensor{\sigma}
(\vec{u}) \, d\Omega + 
\int_{\Gamma_\tau} \trialvec[v] \cdot \vec{\tau}(\vec{x},t) \, d\Gamma.
\end{equation}

% ----------------------------------------------------------------------
\subsubsection{Implicit Time Stepping}
In practice we do not use implicit time stepping when we include
inertia. We provide this section to illustrate the derivation of the
point-wise functions for the residual and Jacobian. The resulting
system of equations to solve is
\begin{align}
  \label{eqn:elasticity:velocity:implicit}
  \int_\Omega \trialvec[u] \cdot \left( \frac{\partial \vec{u}}{\partial t} \right) \, d\Omega 
&= 
  \int_\Omega \trialvec[u] \cdot \vec{v} \, d\Omega, \\
%
  \label{eqn:elasticity:displacement:implicit}
  \int_\Omega \trialvec[v] \cdot \left( \rho \frac{\partial \vec{v}}{\partial t} \right) \, 
d\Omega &=
  \int_\Omega \trialvec[v] \cdot \vec{f}(\vec{x},t) + \nabla \trialvec[v] : -\tensor{\sigma}
(\vec{u}) \, d\Omega + 
\int_{\Gamma_\tau} \trialvec[u] \cdot \vec{\tau}(\vec{x},t) \, d\Gamma.
\end{align}
Identifying $F(t,s,\dot{s})$ and $G(t,s)$, we have
\begin{alignat}{2}
  F^u(t,s,\dot{s}) &= \int_\Omega \trialvec[u] \cdot \eqnannotate{\left( \frac{\partial 
\vec{u}}{\partial t} 
\right)}{f_0^u} \, d\Omega,
  & \qquad
  G^u(t,s) &= \int_\Omega \trialvec[u] \cdot \eqnannotate{\vec{v}}{g_0^u} \, d\Omega, \\
  %  
  F^v(t,s,\dot{s}) &= \int_\Omega \trialvec[v] \cdot \eqnannotate{\left( \rho \frac{\partial 
\vec{v}}{\partial t} 
\right)}{f_0^v} \, d\Omega,
  & \qquad
  G^v(t,s) &= \int_\Omega \trialvec[v] \cdot \eqnannotate{\vec{f}(\vec{x},t)}{g_0^v} + \nabla 
\trialvec[v] : 
\eqnannotate{-\tensor{\sigma}(\vec{u})}{g_1^v} \, d\Omega + \int_{\Gamma_\tau} \trialvec[u] 
\cdot 
\eqnannotate{\vec{\tau}(\vec{x},t)}{g_0^v} \, d\Gamma.
\end{alignat}


\subsubsection{Jacobians}

With two fields we have four Jacobians for each side of the equation associated with the 
coupling of the two 
fields,
\begin{align}
  J_F^{uu} &= \frac{\partial F^u}{\partial u} + s_\mathit{tshift} \frac{\partial F^u}{\partial 
\dot{u}} = \int_\Omega 
\trialvec[u] \cdot s_\mathit{tshift}\,\basisvec[u] \, d\Omega = \int_\Omega \trialscalar[u]_i 
\, 
\eqnannotate{s_\mathit{tshift} \delta_{ij}}{J_{f0}^{uu}} \, \basisscalar[u]_j \, d\Omega \\
  J_F^{uv} &= \frac{\partial F^u}{\partial v} + s_\mathit{tshift} \frac{\partial F^u}{\partial 
\dot{v}} = \tensor{0} \\
  J_F^{vu} &= \frac{\partial F^v}{\partial u} + s_\mathit{tshift} \frac{\partial F^v}{\partial 
\dot{u}} = \tensor{0} \\
  J_F^{vv} &= \frac{\partial F^v}{\partial v} + s_\mathit{tshift} \frac{\partial F^v}{\partial 
\dot{v}} = \int_\Omega 
\trialvec[v] \cdot s_\mathit{tshift}\,\rho\,\basisvec[v] \, d\Omega = \int_\Omega 
\trialscalar[v]_i \, 
\eqnannotate{s_\mathit{tshift} \, \rho \, \delta_{ij}}{J_{f0}^{vv}} \, \basisscalar[v]_j \, 
d\Omega \\
  J_G^{uu} &= \frac{\partial G^u}{\partial u} = \tensor{0} \\
  J_G^{uv} &= \frac{\partial G^u}{\partial v} = \int_\Omega \trialvec[u] \cdot \basisvec[v] \, 
d\Omega = 
\int_\Omega \trialscalar[u]_i \, \eqnannotate{\delta_{ij}}{J_{g0}^{uv}} \, \basisscalar[v]_j 
\, d\Omega \\
  J_G^{vu} &= \frac{\partial G^v}{\partial u} = \int_\Omega \nabla \trialvec[v] : 
\frac{\partial}{\partial u}(-
\tensor{\sigma}) \, d\Omega 
  = \int_\Omega \nabla \trialvec[v] : -\tensor{C} : \frac{1}{2}(\nabla + \nabla^T)\basisvec[u] 
\, d\Omega 
  = \int_\Omega \trialscalar[v]_{i,k} \, \eqnannotate{\left( -C_{ikjl} \right)}{J_{g3}^{vu}}  
\, 
\basisscalar[u]_{j,l}\, d\Omega \\
  J_G^{vv} &= \frac{\partial G^v}{\partial v} = \tensor{0}
\end{align}

% ----------------------------------------------------------------------
\subsection{Explicit Time Stepping}
Recall that explicit time stepping requires $F(t,s,\dot{s})=\dot{s}$. We write $F^*(t,s,
\dot{s}) = \dot{s}$ and
$G^*(t,s) = J_F^{-1}G(t,s)$ and we do not provide functions for $f_0$ and $f_1$. Thus, our 
system of equations to 
solve is
\begin{align}
  \label{eqn:elasticity:velocity:explicit}
  \int_\Omega \trialvec[u] \cdot \frac{\partial \vec{u}}{\partial t} \, d\Omega &= 
  \int_\Omega \trialvec[u] \cdot \vec{v} \, d\Omega, \\
%
  \label{eqn:elasticity:displacement:explicit}
  \int_\Omega \trialvec[v] \cdot \frac{\partial \vec{v}}{\partial t} \, d\Omega &=
  \frac{1}{\int_\Omega \trialvec[v] \cdot \rho\,\basisvec[v] \, d\Omega} \left( \int_\Omega 
\trialvec[v] \cdot 
\vec{f}(\vec{x},t) + \nabla \trialvec[u] : -\tensor{\sigma}(\vec{u}) \, d\Omega + 
\int_{\Gamma_\tau} \trialvec[u] 
\cdot \vec{\tau}(\vec{x},t) \, d\Gamma \right).
\end{align}
Identifying $F(t,s,\dot{s})$ and $G(t,s)$, we have
\begin{align}
  F^u(t,s,\dot{s}) &= \int_\Omega \trialvec[u] \cdot \frac{\partial \vec{u}}{\partial t} \, 
d\Omega, \\
%
  G^u(t,s) &= \int_\Omega \trialvec[u] \cdot \eqnannotate{\vec{v}}{g_0^u} \, d\Omega, \\
  %  
  F^v(t,s,\dot{s}) &= \int_\Omega \trialvec[v] \cdot \frac{\partial \vec{v}}{\partial t}  \, 
d\Omega, \\
%
  G^v(t,s) &= \frac{1}{\int_\Omega \trialvec[v] \cdot {\eqnannotate{\rho}{J_{f0}^{vv}}}
\basisvec[v] \, d\Omega} 
\left( \int_\Omega \trialvec[v] \cdot \eqnannotate{\vec{f}(t)}{g_0^v} + \nabla \trialvec[v] : 
\eqnannotate{-
\tensor{\sigma}(\vec{u})}{g_1^v} \, d\Omega + \int_{\Gamma_\tau} \trialvec[v] \cdot 
\eqnannotate{\vec{\tau}
(\vec{x},t)}{g_0^v} \, d\Gamma \right).
\end{align}
where $J_{f0}^{uu} = \tensor{I}$, and we refer to $J_F$ as the LHS
(or I) Jacobian for explicit time stepping.

% ----------------------------------------------------------------------
\subsection{Elasticity Constitutive Models}

The Jacobian for the elasticity equation is
\begin{equation}
J_{G}^{vu} = \frac{\partial G^{v_i}}{\partial u_j}.
\end{equation}
In computing the derivative, we consider the linearized form:
\begin{align}
  \sigma_{ik} &= C_{ikjl} \epsilon_{jl} \\
  \sigma_{ik} &= C_{ikjl} \frac{1}{2} ( u_{j,l} + u_{l,j} ) \\
  \sigma_{ik} &= \frac{1}{2} ( C_{ikjl} + C_{iklj} ) u_{j,l} \\
  \sigma_{ik} &= C_{ikjl} u_{j,l} \\
\end{align}
In computing the Jacobian, we take the derivative of the stress tensor with respect to the 
displacement field,
\begin{equation}
  \frac{\partial}{\partial u_j} \sigma_{ik} = C_{ikjl} \basisscalar[u]_{j,l},
\end{equation}
so we have
\begin{equation}
\boxed{
  J_{g3}^{vu}(i,j,k,l) = -C_{ikjl}
}
\end{equation}
For many elasticity constitutive models we prefer to separate the
stress into the mean stress and deviatoric stress:
\begin{gather}
  \tensor{\sigma} = \sigma^\mathit{mean} \tensor{I} + \tensor{\sigma}^\mathit{dev} \text{, 
where}\\
  \sigma^\mathit{mean} = \frac{1}{3} \Tr(\tensor{\sigma}) = \frac{1}{3} (\sigma_{11} + 
\sigma_{22} + \sigma_{33}).
\end{gather}
Sometimes it is convenient to use pressure (positive pressure corresponds to compression) 
instead of the mean 
stress:
\begin{gather}
  \tensor{\sigma} = -p \tensor{I} + \tensor{\sigma}^\mathit{dev} \text{, where}\\
  p = -\frac{1}{3} \Tr(\tensor{\sigma}).
\end{gather}

The Jacobian with respect to the deviatoric stress is
\begin{align}
  \frac{\partial \sigma^\mathit{dev}_{ik}}{\partial u_j}  &= \frac{\partial}{\partial u_j} 
\left(\sigma_{ik} - 
\frac{1}{3} \sigma_{mm} \delta_{ik} \right) \\
  \frac{\partial \sigma^\mathit{dev}_{ik}}{\partial u_j}  &= C_{ikjl} \basisscalar[u]_{j,l} - 
\frac{1}{3} C_{mmjl} 
\delta_{ik} \basisscalar[u]_{j,l}.
\end{align}
We call these modified elastic constants $C^\mathit{dev}_{ikjl}$, so that we have
\begin{equation}
\boxed{
  C^\mathit{dev}_{ikjl} = C_{ikjl} - \frac{1}{3} C_{mmjl} \delta_{ik}.
}
\end{equation}.

% ----------------------------------------------------------------------
\subsubsection{Isotropic Linear Elasticity}

We implement isotropic linear elasticity both with and without a
reference stress-strain state. With a linear elastic material it is
often convenient to compute the deformation relative to an unknown
initial stress-strain state. Furthermore, when we use an initial
undeformed configuration with zero stress and strain, the reference
stress and strain are zero, so this presents a simplifcation of the
more general case of the stress-strain state relative to the reference
stress-strain state.

Without a reference stress-strain state, we have
\begin{equation}
  \sigma_{ij} = \lambda \epsilon_{kk} \delta_{ij} + 2\mu\epsilon_{ij},
\end{equation}
and with a reference stress-strain state, we have
\begin{equation}
  \sigma_{ij} = \sigma_{ij}^\mathit{ref} + \lambda \left(\epsilon_{kk} - \epsilon_{kk}
^\mathit{ref}\right)
\delta_{ij} + 2\mu\left(\epsilon_{ij}-\epsilon_{ij}^\mathit{ref}\right).
\end{equation}
The mean stress is
\begin{align}
  \sigma^\mathit{mean} &= \frac{1}{3} \sigma_{kk}, \\
  \sigma^\mathit{mean} &= \frac{1}{3} \sigma_{kk}^\mathit{ref} + \left(\lambda+\frac{2}
{3}\mu\right)
\left(\epsilon_{kk}-\epsilon_{kk}^\mathit{ref}\right),
\end{align}
\begin{equation}
  \boxed{
  \sigma^\mathit{mean} = \frac{1}{3} \sigma_{kk}^\mathit{ref} + K \left(\epsilon_{kk}-
\epsilon_{kk}^\mathit{ref}
\right),
}%boxed
\end{equation}
where $K=\lambda+2\mu/3$ is the bulk modulus. 
If the reference stress and reference strain are both zero, then this reduces to
\begin{equation}
  \boxed{
  \sigma^\mathit{mean} = K \epsilon_{kk}.
}%boxed
\end{equation}
The deviatoric stress is
\begin{align}
  \sigma_{ij}^\mathit{dev} &= \sigma_{ij} - \sigma^\mathit{mean}\delta_{ij}, \\
  \sigma_{ij}^\mathit{dev} &= \sigma_{ij}^\mathit{ref} + \lambda\left(\epsilon_{kk}-
\epsilon_{kk}^\mathit{ref}
\right)\delta_{ij} + 2\mu\left(\epsilon_{ij}-\epsilon_{ij}^\mathit{ref}\right) - 
\left(\frac{1}{3}\sigma_{kk}
^\mathit{ref} + \left(\lambda+\frac{2}{3}\mu\right)\left(\epsilon_{kk}-\epsilon_{kk}
^\mathit{ref}\right)\right)
\delta_{ij}, \\
  \sigma_{ij}^\mathit{dev} &= \sigma_{ij}^\mathit{ref} -\frac{1}{3}\sigma_{kk}^\mathit{ref}
\delta_{ij} + 
2\mu\left(\epsilon_{ij}-\epsilon_{ij}^\mathit{ref}\right) - \frac{2}{3}\mu\left(\epsilon_{kk}-
\epsilon_{kk}
^\mathit{ref}\right)\delta_{ij},
\end{align}
\begin{equation}
  \boxed{
  \sigma_{ij}^\mathit{dev} = \left\{ \begin{array}{lcr}
      \sigma_{ii}^\mathit{ref} -\frac{1}{3}\sigma_{kk}^\mathit{ref} + 2\mu\left(\epsilon_{ii}-
\epsilon_{ii}
^\mathit{ref}\right) - \frac{2}{3}\mu\left(\epsilon_{kk}-\epsilon_{kk}^\mathit{ref}\right) & 
\text{if} & i = j, \\
      \sigma_{ij}^\mathit{ref} + 2\mu\left(\epsilon_{ij}-\epsilon_{ij}^\mathit{ref}\right) & 
\text{if} & i \neq j.
    \end{array} \right.
}%boxed
\end{equation}
If the reference stress and reference strain are both zero, then this reduces to
\begin{equation}
  \boxed{
  \sigma_{ij}^\mathit{dev} = \left\{ \begin{array}{lcr}
      2\mu\epsilon_{ii} - \frac{2}{3}\mu\epsilon_{kk} & \text{if} & i = j, \\
      2\mu\epsilon_{ij} & \text{if} & i \neq j.
    \end{array} \right.
  }%boxed
\end{equation}

For isotropic linear elasticity
\begin{align}
  C_{1112} &= C_{1113} = C_{1113} = C_{1121} = C_{1123} = C_{1131} = C_{1132} = 0\\
  C_{1211} &= C_{1213} = C_{1222} = C_{1223} = C_{1231} = C_{1232} = C_{1233} = 0,
\end{align}
and
\begin{align}
  C_{1111} = C_{2222} = C_{3333} &= \lambda + 2 \mu, \\
  C_{1122} = C_{1133} = C_{2233} &= \lambda, \\
  C_{1212} = C_{2323} = C_{1313} &= \mu.
\end{align}
The deviatoric elastic constants are:
\begin{align}
  C^\mathit{dev}_{1111} = C^\mathit{dev}_{2222} = C^\mathit{dev}_{3333} &= \frac{4}{3}\mu, \\
  C^\mathit{dev}_{1122} = C^\mathit{dev}_{1133} = C^\mathit{dev}_{2233} &= -\frac{2}{3}\mu, \\
  C^\mathit{dev}_{1212} = C^\mathit{dev}_{2323} = C^\mathit{dev}_{1313} &= \mu.
\end{align}

\subsubsection{Isotropic Generalized Maxwell Viscoelasticity}

We use the same general formulation for both the simple Maxwell
viscoelastic model and the generalized Maxwell model (several Maxwell
models in parallel). We implement the Maxwell models both with and
without a reference stress-strain state. Note that it is also possible
to specify an initial state variable value (viscous strain). Viscous
flow is completely deviatoric, so we split the stress into volumetric
and deviatoric parts, as described above.  The volumetric part is
identical to that of an isotropic elastic material. The deviatoric
part is given by:
\begin{equation}
  \sigma^\mathit{dev}_{ij}\left(t\right)=2\mu_{tot}\left(\mu_{0}\epsilon^\mathit{dev}_{ij}
    \left(t\right)+\sum_{m=1}^{N}\mu_{m}h^{m}_{ij}\left(t\right)-\epsilon^\mathit{refdev}_{ij}
    \right)+\sigma^\mathit{refdev}_{ij},
\end{equation}
where $\mu_{tot}$ is the total shear modulus of the model, $\mu_{0}$
is the fraction of the shear modulus accommodated by the elastic
spring in parallel with the Maxwell models, the $\mu_{m}$ are the
fraction of the shear modulus accommodated by each Maxwell model
spring, and $\epsilon^{\mathit{refdev}}_{ij}$ and
$\sigma^{\mathit{refdev}}_{ij}$ are the reference deviatioric strain
and stress, respectively. The viscous strain is:
\begin{equation}
h^{m}_{ij}\left(t\right)=\exp\frac{-\Delta
  t}{\tau_{m}}h^{m}_{ij}\left(t_{n}\right)+\Delta h^{m}_{ij},
\end{equation}
where $t_{n}$ is a time between $t=0$ and $t=t$, $\Delta
h^{m}_{ij}$ is the viscous strain between $t=t_{n}$ and
$t=t$, and $\tau_{m}$ is the Maxwell time:
\begin{equation}
  \tau_{m}=\frac{\eta_{m}}{\mu_{tot}\mu_{m}}.
\end{equation}
Approximating the strain rate as constant over each time step,
this is given as:
\begin{equation}
\Delta h^{m}_{ij}=\frac{\tau_{m}}{\Delta t}\left(1-\exp\frac{-\Delta
  t}{\tau_{m}}\right)\left(\epsilon^{\mathit{dev}}_{ij}\left(t\right)-\epsilon^{\mathit{dev}}_{ij}\left(t_{n}\right)\right)=\Delta
h^{m}\left(\epsilon^{\mathit{dev}}_{ij}\left(t\right)-\epsilon^{\mathit{dev}}_{ij}\left(t_n\right)\right).
\end{equation}
The approximation is singular for zero time steps, but a series
expansion may be used for small time-step sizes:
\begin{equation}
  \Delta h^{m}\approx1-\frac{1}{2}\left(\frac{\Delta
    t}{\tau_{m}}\right)+\frac{1}{3!}\left(\frac{\Delta
    t}{\tau_{m}}\right)^{2}-\frac{1}{4!}\left(\frac{\Delta
    t}{\tau_{m}}\right)^{3}+\cdots\,.
\end{equation}
This converges with only a few terms.


% ----------------------------------------------------------------------
\section{Elasticity With Infinitesimal Strain and Faults With Prescribed Slip}

For each fault, which is an internal interface, we add a boundary
condition prescribing the jump in the displacement field across the
fault,
\begin{gather}
  \rho \frac{\partial^2\vec{u}}{\partial t^2} - \vec{f}(\vec{x},t) - \tensor{\nabla} \cdot 
\tensor{\sigma}
(\vec{u}) = \vec{0} \text{ in }\Omega, \\
%
  \tensor{\sigma} \cdot \vec{n} = \vec{\tau}(\vec{x},t) \text{ on }\Gamma_\tau, \\
%
  \vec{u} = \vec{u}_0(\vec{x},t) \text{ on }\Gamma_u, \\
%
  \label{eqn:bc:prescribed_slip}
  \vec{0} = \vec{d}(\vec{x},t) - \vec{u}^+(\vec{x},t) + \vec{u}^-(\vec{x},t) \text{ on }\Gamma_f,
\end{gather}
where $\vec{u}^+$ is the displacement vector on the ``positive'' side
of the fault, $\vec{u}^-$ is the displacement vector on the ``negative''
side of the fault, $\vec{d}$ is the slip vector on the fault, and
$\vec{n}$ is the fault normal which points from the negative side of
the fault to the positive side of the fault. Using a domain
decomposition approach for constraining the fault slip, yields
additional Neumann-like boundary conditions on the fault surface,
\begin{gather}
  \tensor{\sigma} \cdot \vec{n} = -\vec{\lambda}(\vec{x},t) \text{ on }\Gamma_{f^+}, \\
  \tensor{\sigma} \cdot \vec{n} = +\vec{\lambda}(\vec{x},t) \text{ on }\Gamma_{f^-},
\end{gather}
where $\vec{\lambda}$ is the vector of Lagrange multipliers
corresponding to the tractions applied to the fault surface to
generate the prescribed slip.

\subsection{Notation}
\begin{itemize}
\item Unknowns
  \begin{description}
  \item[$\vec{u}$] Displacement field
  \item[$\vec{v}$] Velocity field (if including inertial term)
  \item[$\vec{\lambda}$] Lagrange multiplier field
  \end{description}
\item Derived quantities
  \begin{description}
    \item[$\tensor{\sigma}$] Stress tensor
    \item[$\tensor{\epsilon}$] Strain tensor
  \end{description}
\item Constitutive parameters
  \begin{description}
  \item[$\mu$] Shear modulus
  \item[$K$] Bulk modulus
  \item[$\rho$] Density
  \end{description}
\item Source terms
  \begin{description}
    \item[$\vec{f}$] Body force per unit volume, for example $\rho \vec{g}$
    \item[$\vec{d}$] Slip vector field on the fault corresponding to a
      jump in the displacement field across the fault
  \end{description}
\end{itemize}

\subsection{Neglecting Inertia}

If we neglect the inertial term, then time dependence only arises
from history-dependent constitutive equations and boundary conditions. Considering the
displacement $\vec{u}$ and Lagrange multiplier $\vec{\lambda}$ fields as the unknowns, we have
\begin{align}
  \vec{s}^T &= (\vec{u} \quad \vec{\lambda})^T, \\
%
  \vec{0} &= \vec{f}(\vec{x},t) + \tensor{\nabla} \cdot \tensor{\sigma}(\vec{u}) \text{ in }
\Omega, \\
% Neumann
  \tensor{\sigma} \cdot \vec{n} &= \vec{\tau}(\vec{x},t) \text{ on }\Gamma_\tau, \\
% Dirichlet
  \vec{u} &= \vec{u}_0(\vec{x},t) \text{ on }\Gamma_u, \\
% Fault
  \vec{0} &= \vec{d}(\vec{x},t) - \vec{u}^+(\vec{x},t) + \vec{u}^-(\vec{x},t) \text{ on }\Gamma_f, \\
  \tensor{\sigma} \cdot \vec{n} &= -\vec{\lambda}(\vec{x},t) \text{ on }\Gamma_{f^+}, \\
  \tensor{\sigma} \cdot \vec{n} &= +\vec{\lambda}(\vec{x},t) \text{ on }\Gamma_{f^-}.
\end{align}

We create the weak form by taking the dot product with the trial
function $\trialvec[u]$ or $\trialvec[\lambda]$ and integrating over the domain:
\begin{align}
  0 &= \int_\Omega \trialvec[u] \cdot \left( \vec{f}(t) + \tensor{\nabla} \cdot \tensor{\sigma} (\vec{u}) \right) \, d\Omega, \\
  0 &= \int_{\Gamma_f} \trialvec[\lambda] \cdot \left( \vec{d}(\vec{x},t) - \vec{u}^+(\vec{x},t) + \vec{u}^-(\vec{x},t) \right) \, d\Gamma.
\end{align}
Using the divergence theorem and incorporating the Neumann boundary and fault interface
conditions, we can rewrite the first equation as
\begin{equation}
  0 = \int_\Omega \trialvec[u] \cdot \vec{f}(t) + \nabla \trialvec[u] : -\tensor{\sigma}
  (\vec{u}) \, d\Omega
  + \int_{\Gamma_\tau} \trialvec[u] \cdot \vec{\tau}(\vec{x},t) \, d\Gamma
  + \int_{\Gamma_{f^+}} \trialvec[u] \cdot -\vec{\lambda}(\vec{x},t) \, d\Gamma
  + \int_{\Gamma_{f^-}} \trialvec[u] \cdot +\vec{\lambda}(\vec{x},t) \, d\Gamma. \\
\end{equation}
In practice we integrate over the fault surface by integrating over
the faces of the cohesive cells on the positive and negative sides of
the fault. Breaking up the integral over the fault surface in the
second equation into integrals over the positive and negative sides of
the fault, we have
\begin{equation}
  0 = \int_{\Gamma_{f^+}} \trialvec[\lambda] \cdot \left( \frac{1}{2}\vec{d}(\vec{x},t) - \vec{u}(\vec{x},t) \right) \, d\Gamma
    + \int_{\Gamma_{f^-}} \trialvec[\lambda] \cdot \left( \frac{1}{2}\vec{d}(\vec{x},t) + \vec{u}(\vec{x},t) \right) \, d\Gamma.
  \end{equation}


Identifying $F(t,s,\dot{s})$ and $G(t,s)$, we have
\begin{align}
  F^u(t,s,\dot{s}) &= 0, \\
  F^\lambda(t,s,\dot{s}) &= 0, \\
  \begin{split}
  G^u(t,s) &=  
   \int_\Omega \trialvec[u] \cdot \eqnannotate{\vec{f}(\vec{x},t)}{g_0^u}
  + \nabla \trialvec[u] : \eqnannotate{-\tensor{\sigma}(\vec{u})}{g_1^u} \, d\Omega \\
  &+ \int_{\Gamma_\tau} \trialvec[u] \cdot \eqnannotate{\vec{\tau}(\vec{x},t)}{g_0^u} \, d\Gamma \\
  &+ \int_{\Gamma_{f^+}} \trialvec[u] \cdot \eqnannotate{-\vec{\lambda}(\vec{x},t)}{g_0^u} \, d\Gamma
  + \int_{\Gamma_{f^-}} \trialvec[u] \cdot \eqnannotate{+\vec{\lambda}(\vec{x},t)}{g_0^u} \, d\Gamma,
  \end{split} \\
  G^\lambda(t,s) &= 
\int_{\Gamma_{f^+}} \trialvec[\lambda] \cdot \eqnannotate{\left( \frac{1}{2}\vec{d}(\vec{x},t) - \vec{u}(\vec{x},t) \right)}{g_0^\lambda} \, d\Gamma
    + \int_{\Gamma_{f^-}} \trialvec[\lambda] \cdot \eqnannotate{\left( \frac{1}{2}\vec{d}(\vec{x},t) + \vec{u}(\vec{x},t) \right)}{g_0^\lambda} \, d\Gamma.
\end{align}
Note that we have multiple $g_0^u$ functions, each associated with an
integral over a different domain or boundary. The integral over the
domain $\Omega$ is implemented by the material, the integral over the
boundary $\Gamma_\tau$ is implemented by the Neumann boundary
condition, and the integrals over the interfaces $\Gamma_{f^+}$ and
$\Gamma_{f^-}$ are implemented by the fault (cohesive cells).

\subsubsection{Jacobians}

With the solution composed of the displacement and Lagrange multiplier fields, the Jacobians are:
\begin{align}
% J_F
  J_F^{uu} &= \tensor{0} \\
  J_F^{\lambda \lambda} &= \tensor{0} \\
% J_G uu
  J_G^{uu} &= \frac{\partial G^u}{\partial u} = \int_\Omega \nabla \trialvec[u] : 
\frac{\partial}{\partial u}(-
\tensor{\sigma}) \, d\Omega 
  = \int_\Omega \nabla \trialvec[u] : -\tensor{C} : \frac{1}{2}(\nabla + \nabla^T)\basisvec[u] 
\, d\Omega 
  = \int_\Omega \trialscalar[v]_{i,k} \, \eqnannotate{\left( -C_{ikjl} \right)}{J_{g3}^{uu}} \, \basisscalar[u]_{j,l}\, d\Omega \\
  % J_G u \lambda
  \begin{split}
J_G^{u\lambda} &= \frac{\partial G^u}{\partial \lambda} =
\int_{\Gamma_{f^+}} \trialvec[u] \cdot \frac{\partial}{\partial \lambda}(-\vec{\lambda}) \, d\Gamma
+ \int_{\Gamma_{f^-}} \trialvec[u] \cdot \frac{\partial}{\partial \lambda}(+\vec{\lambda}) \, d\Gamma \\
& \quad\quad = \int_{\Gamma_{f^+}} \trialscalar[u]_i \eqnannotate{-1}{J_{g0}^{u\lambda}}\basisscalar[\lambda]_j \, d\Gamma
+ \int_{\Gamma_{f^-}} \trialscalar[u]_i \eqnannotate{+1}{J_{g0}^{u\lambda}}\basisscalar[\lambda]_j \, d\Gamma
\end{split} \\
% J_G \lambda u
\begin{split}
J_G^{\lambda u} &= \frac{\partial G^\lambda}{\partial u} =
\int_{\Gamma_{f^+}} \trialvec[\lambda] \cdot \frac{\partial}{\partial u}\left(\frac{1}{2}\vec{d}(\vec{x},t) - \vec{u}(\vec{x},t\right) \, d\Gamma
+ \int_{\Gamma_{f^-}} \trialvec[\lambda] \cdot \frac{\partial}{\partial u}\left(\frac{1}{2}\vec{d}(\vec{x},t) + \vec{u}(\vec{x},t)\right) \, d\Gamma \\
&\quad\quad = \int_{\Gamma_{f^+}} \trialscalar[\lambda]_i (\eqnannotate{-1}{J_{g0}^{\lambda u}})\basisscalar[u]_j \, d\Gamma
            + \int_{\Gamma_{f^-}} \trialscalar[\lambda]_i (\eqnannotate{+1}{J_{g0}^{\lambda u}})\basisscalar[u]_j \, d\Gamma
\end{split} \\
%
  J_G^{\lambda \lambda} &= 0
\end{align}

\subsection{Including Inertia}

For convenience we cast the elasticity equation in the form of a first order
equation by considering the displacement $\vec{u}$, velocity $\vec{v}$, and Lagrange multipliers $\vec{\lambda}$
as unknowns,
\begin{align}
  \vec{s}^T &= (\vec{u} \quad \vec{v} \quad \vec{\lambda})^T, \\
%
  \frac{\partial\vec{u}}{\partial t} &= \vec{v}, \\
%
  \rho \frac{\partial\vec{v}}{\partial t} &= \vec{f}(\vec{x},t) + \tensor{\nabla} \cdot 
\tensor{\sigma}(\vec{u}) 
\text{ in }\Omega, \\
% Neumann
  \tensor{\sigma} \cdot \vec{n} &= \vec{\tau}(\vec{x},t) \text{ on }\Gamma_\tau, \\
% Dirichlet
  \vec{u} &= \vec{u}_0(\vec{x},t) \text{ on }\Gamma_u, \\
% Fault
  \vec{0} &= \vec{d}(\vec{x},t) - \vec{u}^+(\vec{x},t) + \vec{u}^-(\vec{x},t) \text{ on }\Gamma_f, \\
  \tensor{\sigma} \cdot \vec{n} &= -\vec{\lambda}(\vec{x},t) \text{ on }\Gamma_{f^+}, \\
  \tensor{\sigma} \cdot \vec{n} &= +\vec{\lambda}(\vec{x},t) \text{ on }\Gamma_{f^-}.
\end{align}

For trial functions $\trialvec[u]$, $\trialvec[v]$, and $\trialvec[\lambda]$ we write the weak form as
\begin{align}
  \int_\Omega \trialvec[u] \cdot \left( \frac{\partial \vec{u}}{\partial t} \right) \, d\Omega 
    &= \int_\Omega \trialvec[u] \cdot \vec{v} \, d\Omega, \\
%
  \int_\Omega \trialvec[v] \cdot \left( \rho \frac{\partial \vec{v}}{\partial t} \right) \, d\Omega &= 
  \int_\Omega \trialvec[v] \cdot \left( \vec{f}(\vec{x},t) + \tensor{\nabla} \cdot 
  \tensor{\sigma}(\vec{u}) \right) \, d\Omega, \\
  %
  0 &= \int_{\Gamma_{f^+}} \trialvec[\lambda] \cdot \left( \frac{1}{2}\vec{d}(\vec{x},t) - \vec{u}(\vec{x},t) \right) \, d\Gamma
    + \int_{\Gamma_{f^-}} \trialvec[\lambda] \cdot \left( \frac{1}{2}\vec{d}(\vec{x},t) + \vec{u}(\vec{x},t) \right) \, d\Gamma.
\end{align}
Using the divergence theorem and incorporating the Neumann boundary
and fault interface conditions, we can rewrite the second equation as
\begin{multline}
  \int_\Omega \trialvec[v] \cdot \left( \rho \frac{\partial \vec{v}}{\partial t} \right) \, 
d\Omega =
  \int_\Omega \trialvec[v] \cdot \vec{f}(\vec{x},t) + \nabla \trialvec[v] : -\tensor{\sigma}
  (\vec{u}) \, d\Omega
  + \int_{\Gamma_\tau} \trialvec[v] \cdot \vec{\tau}(\vec{x},t) \, d\Gamma \\
+ \int_{\Gamma_{f^+}} \trialvec[v] \cdot -\vec{\lambda}(\vec{x},t) \, d\Gamma
+ \int_{\Gamma_{f^-}} \trialvec[v] \cdot +\vec{\lambda}(\vec{x},t) \, d\Gamma.
\end{multline}

% ----------------------------------------------------------------------
\subsection{Explicit Time Stepping}

Recall that for explicit time stepping we want
$F(t,s,\dot{s})=\dot{s}$. However, our fault interface constraint
equation cannot be put into this form. Nevertheless, we put the first
two equations in this form. The resulting equation will be a
differential algebraic equation (DAE). Our system of equations to
solve is
\begin{align}
  \int_\Omega \trialvec[u] \cdot \frac{\partial \vec{u}}{\partial t} \, d\Omega &= 
  \int_\Omega \trialvec[u] \cdot \vec{v} \, d\Omega, \\
  %
  \begin{split}
  \int_\Omega \trialvec[v] \cdot \frac{\partial \vec{v}}{\partial t} \, d\Omega &=
  \frac{1}{\int_\Omega \trialvec[v] \cdot \rho\,\basisvec[v] \, d\Omega} \left( \int_\Omega 
  \trialvec[v] \cdot \vec{f}(\vec{x},t) + \nabla \trialvec[u] : -\tensor{\sigma}(\vec{u}) \, d\Omega
  + \int_{\Gamma_\tau} \trialvec[v] \cdot \vec{\tau}(\vec{x},t) \, d\Gamma \right. \\
  & \quad \left. + \int_{\Gamma_{f^+}} \trialvec[v] \cdot -\vec{\lambda}(\vec{x},t) \, d\Gamma
  + \int_{\Gamma_{f^-}} \trialvec[v] \cdot +\vec{\lambda}(\vec{x},t) \, d\Gamma \right) ,
  \end{split} \\
%
  0 &= \int_{\Gamma_{f^+}} \trialvec[\lambda] \cdot \left( \frac{1}{2}\vec{d}(\vec{x},t) - \vec{u}(\vec{x},t) \right) \, d\Gamma
    + \int_{\Gamma_{f^-}} \trialvec[\lambda] \cdot \left( \frac{1}{2}\vec{d}(\vec{x},t) + \vec{u}(\vec{x},t) \right) \, d\Gamma.
\end{align}
Identifying $F(t,s,\dot{s})$ and $G(t,s)$, we have
\begin{align}
%% Fu
  F^u(t,s,\dot{s}) &= \int_\Omega \trialvec[u] \cdot \eqnannotate{\frac{\partial \vec{u}}{\partial t}}{f_o^u} \, d\Omega, \\
% Fv
  F^v(t,s,\dot{s}) &= \int_\Omega \trialvec[v] \cdot \eqnannotate{\frac{\partial \vec{v}}{\partial t}}{f_0^v}  \, d\Omega, \\
  % F\lambda
  F^\lambda(t,s,\dot{s}) &= 0, \\
% Gu
  G^u(t,s) &= \int_\Omega \trialvec[u] \cdot \eqnannotate{\vec{v}}{g_0^u} \, d\Omega, \\
  % Gv
  \begin{split}
  G^v(t,s) &= \frac{1}{\int_\Omega \trialvec[v] \cdot {\eqnannotate{\rho}{J_{f0}^{*vv}}}
    \basisvec[v] \, d\Omega} 
  \left( \int_\Omega \trialvec[v] \cdot \eqnannotate{\vec{f}(t)}{g_0^v} + \nabla \trialvec[v] : 
  \eqnannotate{-\tensor{\sigma}(\vec{u})}{g_1^v} \, d\Omega
  + \int_{\Gamma_\tau} \trialvec[v] \cdot \eqnannotate{\vec{\tau} (\vec{x},t)}{g_0^v} \, d\Gamma \right. \\
  &\quad\left. + \int_{\Gamma_{f^+}} \trialvec[v] \cdot \eqnannotate{-\vec{\lambda}(\vec{x},t)}{g_0^v} \, d\Gamma
  + \int_{\Gamma_{f^-}} \trialvec[v] \cdot \eqnannotate{+\vec{\lambda}(\vec{x},t)}{g_0^v} \, d\Gamma \right),
  \end{split} \\
  % G\lambda
  G^\lambda(t,s) &= 
\int_{\Gamma_{f^+}} \trialvec[\lambda] \cdot \eqnannotate{\left( \frac{1}{2}\vec{d}(\vec{x},t) - \vec{u}(\vec{x},t) \right)}{g_0^\lambda} \, d\Gamma
    + \int_{\Gamma_{f^-}} \trialvec[\lambda] \cdot \eqnannotate{\left( \frac{1}{2}\vec{d}(\vec{x},t) + \vec{u}(\vec{x},t) \right)}{g_0^\lambda} \, d\Gamma.
\end{align}
Note that we have multiple $g_0^u$ functions, each associated with an
integral over a different domain or boundary. The integral over the
domain $\Omega$ is implemented by the material, the integral over the
boundary $\Gamma_\tau$ is implemented by the Neumann boundary
condition, and the integrals over the interfaces $\Gamma_{f^+}$ and
$\Gamma_{f^-}$ are implemented by the fault (cohesive cells).


\subsubsection{Jacobians}

With a differential algebraic equation, we only need to specify the Jacobians for the LHS.
\begin{align}
% J_F
  J_F^{uu} &= \frac{\partial F^u}{\partial u} + s_\mathit{tshift} \frac{\partial F^u}{\partial \dot{u}}
  = \int_\Omega \trialscalar[u]_i \eqnannotate{s_\mathit{tshift}}{J_{f0}^{uu}} \basisscalar[u]_j \, d\Omega, \\
%
  J_F^{vv} &= \frac{\partial F^v}{\partial v}  + s_\mathit{tshift} \frac{\partial F^v}{\partial \dot{v}}
  = \int_\Omega \trialscalar[v]_i \eqnannotate{s_\mathit{tshift}}{J_{f0}^{vv}} \basisscalar[v]_j \, d\Omega, \\
%
  J_F^{\lambda \lambda} &= \frac{\partial F^\lambda}{\partial \lambda} + s_\mathit{tshift} \frac{\partial F^\lambda}{\partial \dot{\lambda}}
  = \eqnannotate{\tensor{0}}{J_{f0}^{\lambda \lambda}}
\end{align}

% ----------------------------------------------------------------------
\section{Incompressible Isotropic Elasticity with Infinitesimal Strain (Bathe) and No Faults or Inertia}

Building from the elasticity equation
(equations~\ref{eqn:velocity:strong:form}
and~\ref{eqn:elasticity:order1:strong:form}), we consider an
incompressible material. As the bulk modulus ($K$) approaches
infinity, the volumetric strain ($\Tr(\epsilon)$) approaches zero and
the pressure remains finite, $p = -K \Tr(\epsilon)$. We consider
pressure $p$ as an independent variable and decompose the stress into the
pressure and deviatoric components. As a result, we write the stress tensor in terms of both the displacement and pressure fields,
\begin{equation}
  \tensor{\sigma}(\vec{u},p) = \tensor{\sigma}^\mathit{dev}(\vec{u}) - p\tensor{I}.
\end{equation}

\subsection{Notation}
\begin{itemize}
\item Unknowns
  \begin{description}
  \item[$\vec{u}$] Displacement field
  \item[$p$] Pressure field (positive pressure corresponds to negative stress)
  \end{description}
\item Derived quantities
  \begin{description}
    \item[$\tensor{\sigma}$] Stress tensor
    \item[$\tensor{\epsilon}$] Strain tensor
  \end{description}
\item Constitutive parameters
  \begin{description}
  \item[$\mu$] Shear modulus
  \item[$K$] Bulk modulus
  \item[$\rho$] Density
  \end{description}
\item Source terms
  \begin{description}
    \item[$\vec{f}$] Body force per unit volume, for example $\rho \vec{g}$
  \end{description}
\end{itemize}


\subsection{Implicit Time Stepping}

We only consider the case of an incompressible material while
neglecting inertia. The time dependence only arises from
history-dependent constitutive equations and boundary conditions. We
have
\begin{gather}
  % Solution
  \vec{s}^T = \left( \vec{u} \quad \ p \right)^T, \\
  % Elasticity
  \vec{0} = \vec{f}(t) + \tensor{\nabla} \cdot \left(\tensor{\sigma}^\mathit{dev}(\vec{u}) - p\tensor{I}\right) \text{ in }\Omega, \\
  % Pressure
  0 = \vec{\nabla} \cdot \vec{u} + \frac{p}{K}, \\
  % Neumann
  \tensor{\sigma} \cdot \vec{n} = \vec{\tau} \text{ on }\Gamma_\tau, \\
  % Dirichlet
  \vec{u} = \vec{u}_0 \text{ on }\Gamma_u.
\end{gather}

Using trial functions $\trialvec[u]$ and $\trialscalar[p]$ and
incorporating the Neumann boundary conditions, we write the weak form
as
\begin{gather}
  % Displacement
  0 = 
  \int_\Omega \trialvec[u] \cdot \vec{f}(t) + \nabla \trialvec[u] : \left(-\tensor{\sigma}^\mathit{dev}(\vec{u}) + p\tensor{I}
  \right)\, d\Omega + \int_{\Gamma_\tau} \trialvec[u] \cdot \vec{\tau}(t) \, d\Gamma, \\
  % Pressure
  0 = \int_\Omega \trialscalar[p] \cdot \left(\vec{\nabla} \cdot \vec{u} + \frac{p}{K} \right) 
\, d\Omega.
\end{gather}

Identifying $G(t,s)$, we have
\begin{gather}
  \label{eqn:incompressible:elasticity:displacement}
  0 = \int_\Omega \trialvec[u] \cdot \eqnannotate{\vec{f}(t)}{g_0^u} + \nabla \trialvec[u] :
  \eqnannotate{\left(-\tensor{\sigma}^\mathit{dev}(\vec{u}) + p\tensor{I}\right)}{g_1^u}  \, d\Omega
  + \int_{\Gamma_\tau} \trialvec[u] \cdot \eqnannotate{\vec{\tau}(t)}{g_0^u} \, d\Gamma, \\
%
  \label{eqn:incompressible:elasticity:pressure}
  0 = \int_\Omega \trialscalar[p] \cdot \eqnannotate{\left(\vec{\nabla} \cdot \vec{u} + 
\frac{p}{K} \right)}{g_0^p} \, d\Omega.
\end{gather}


\subsubsection{Jacobians}

With two fields we have four Jacobians for the RHS associated with the coupling of 
the two fields.
\begin{align}
  J_G^{uu} &= \frac{\partial G^u}{\partial u} = \int_\Omega \nabla \trialvec[u] : 
\frac{\partial}{\partial u}(-
\tensor{\sigma}^\mathit{dev}) \, d\Omega 
  = \int_\Omega \trialscalar[u]_{i,k} \, \eqnannotate{\left(-C^\mathit{dev}_{ikjl}\right)}
{J_{g3}^{uu}}  \, 
\basisscalar[u]_{j,l}\, d\Omega \\
  J_G^{up} &= \frac{\partial G^u}{\partial p} = \int_\Omega \nabla\trialvec[u] : \tensor{I} 
\basisscalar[p] \,  d\Omega = \int_\Omega \trialscalar[u]_{i,k} \eqnannotate{\delta_{ik}}{J_{g2}^{up}} \, 
\basisscalar[p] \, d\Omega \\
%
  J_G^{pu} &= \frac{\partial G^p}{\partial u} = \int_\Omega \trialscalar[p] \left(\vec{\nabla} 
\cdot \basisvec[u]\right) \, d\Omega = \int_\Omega \trialscalar[p] \eqnannotate{\delta_{jl}}{J_{g1}^{pu}} 
\basisscalar[u]_{j,l} \, d\Omega\\
  J_G^{pp} &= \frac{\partial G^p}{\partial p} = \int_\Omega \trialscalar[p] \eqnannotate{\frac{1}
{K}}{J_{g0}^{pp}} \basisscalar[p] \, d\Omega
\end{align}


% ----------------------------------------------------------------------
\section{Poroelasticity with Infinitesimal Strain and No Faults or Inertia}

\todo{poroelasticity group}{Update this with revised formulation from
  the poroelasticity group.}

Formulation based on Zheng et al. and Detournay and Cheng (1993).

In this poroelasticity formulation we assume a compressible fluid
completely saturates a porous solid undergoing infinitesimal
strain. We neglect the inertial effects and do not consider faults.

We begin with the elasticity equilibrium equation neglecting the inertial term,
\begin{gather}
  - \vec{f}(\vec{x},t) - \tensor{\nabla} \cdot \tensor{\sigma}(\vec{u},p_f) = \vec{0} 
\text{ in }\Omega, \\
%
  \tensor{\sigma} \cdot \vec{n} = \vec{\tau}(\vec{x},t) \text{ on }\Gamma_\tau, \\
%
  \vec{u} = \vec{u}_0(\vec{x},t) \text{ on }\Gamma_u,
\end{gather}
where $\vec{u}$ is the displacement vector, $\vec{f}$ is the body
force vector, $\tensor{\sigma}$ is the Cauchy stress tensor, and $t$
is time. We specify tractions $\vec{\tau}$ on boundary $\Gamma_\tau$, and
displacements $\vec{u}_0$ on boundary $\Gamma_u$. If gravity is included in
the problem, then usually $\vec{f} = \rho \vec{g}$, where $\rho$ is
the average density $\rho = (1-\phi)\rho_s + \phi \rho_f$, $\phi$ is
the porosity of the solid, $\rho_s$ is the density of the solid, and
$\rho_f$ is the density of the fluid.

Enforcing mass balance of the fluid gives
\begin{gather}
  \frac{\partial \zeta(\vec{u},p_f)}{\partial t} + \nabla \cdot \vec{q}(p_f) = 
\gamma(\vec{x},t) \text{ in }
\Omega, \\
%
  \vec{q} \cdot \vec{n} = q_0(\vec{x},t) \text{ on }\Gamma_q, \\
%
  p_f = p_0(\vec{x},t) \text{ on }\Gamma_p,
\end{gather}
where $\zeta$ is the variation in fluid content, $\vec{q}$ is the rate
of fluid volume crossing a unit area of the porous solid, $\gamma$ is
the rate of injected fluid per unit volume of the porous solid, $q_0$
is the outward fluid velocity normal to the boundary $\Gamma_q$, and
$p_0$ is the fluid pressure on boundary $\Gamma_p$.

We require the fluid flow to follow Darcy's law (Navier-Stokes equation neglecting inertial 
effects),
\begin{gather}
  \vec{q}(p_f) = -\kappa (\nabla p_f - \vec{f}_f), \\
%
  \kappa = \frac{k}{\eta_f}
\end{gather}
where $\kappa$ is the permeability coefficient (Darcy conductivity),
$k$ is the intrinsic permeability, $\eta_f$ is the viscosity of the
fluid, $p_f$ is the fluid pressure, and $\vec{f}_f$ is the body force
in the fluid. If gravity is included in a problem, then usually
$\vec{f}_f = \rho_f \vec{g}$, where $\rho_f$ is the density of the
fluid and $\vec{g}$ is the gravitational acceleration vector.

We assume linear elasticity for the solid phase, so the constitutive behavior can be expressed 
as
\begin{gather}
  \tensor{\sigma}(\vec{u},p_f) = \tensor{C} : \tensor{\epsilon} - \alpha p_f \tensor{I},
\end{gather}
where $\tensor{\sigma}$ is the stress tensor, $\tensor{C}$ is the
tensor of elasticity constants, $\alpha$ is the Biot coefficient
(effective stress coefficient), $\tensor{\epsilon}$ is the strain
tensor, and $\tensor{I}$ is the identity tensor.

For the constitutive behavior of the fluid, we use the volumetric strain to couple the fluid-
solid behavior,
\begin{gather}
  \zeta(\vec{u},p_f) = \alpha \Tr({\tensor{\epsilon}}) + \frac{p_f}{M}, \\
%
  \frac{1}{M} = \frac{\alpha-\phi}{K_s} + \frac{\phi}{K_f},
\end{gather}
where $1/M$ is the specific storage coefficient at constant strain,
$K_s$ is the bulk modulus of the solid, and $K_f$ is the bulk modulus
of the fluid. We can write the trace of the strain tensor as the dot product of the gradient 
and displacement 
field, so we have
\begin{gather}
  \zeta(\vec{u},p_f) = \alpha (\nabla \cdot \vec{u}) + \frac{p_f}{M}.
\end{gather}

\subsection{Notation}
\begin{itemize}
\item Unknowns
  \begin{description}
  \item[$\vec{u}$] Displacement field
  \item[$p_f$] Fluid pressure
  \end{description}
\item Derived quantities
  \begin{description}
    \item[$\tensor{\sigma}$] Stress tensor
    \item[$\tensor{\epsilon}$] Strain tensor
    \item[$\zeta$] Variation of fluid content; variation of fluid volumer per unit volume of 
porous material
    \item[$q$] rate of fluid volume crossing a unit area of the porous
      solid; fluid flux
    \item[$1/M$] Specific storage coefficient at constant strain
    \item[$\kappa$] permability coefficient; Darcy conductivity; $\kappa = k/\eta_f$
    \item[$\rho$] Average density; $\rho = (1-\phi)\rho_s + \phi \rho_f$
  \end{description}
\item Constitutive parameters
  \begin{description}
  \item[$\mu$] Shear modulus of solid
  \item[$K_s$] Bulk modulus of solid
  \item[$K_f$] Bulk modulus of fluid
  \item[$\alpha$] Biot coefficient; effective stress coefficient
  \item[$k$] Intrinsic permeability
  \item[$\eta_f$] Fluid viscosity
  \item[$\rho_s$] Density of solid
  \item[$\rho_f$] Density of fluid
  \item[$\phi$] Porosity; $\frac{V_p}{V}$ ($V_p$ is the volume of the pore space)
  \end{description}
\item Source terms
  \begin{description}
    \item[$\vec{f}$] Body force, for example $\rho \vec{g} = (1-\phi)\rho_s + \phi \rho_f$
    \item[$\vec{f}_f$] Body force in fluid, for example $\rho_f \vec{g}$
    \item[$\gamma$] Source density; rate of injected fluid per unit volume of the porous solid
  \end{description}
\end{itemize}

We consider the displacement $\vec{u}$ and fluid pressure $p_f$ as unknowns,
\begin{align}
  \vec{s}^T &= (\vec{u} \quad p_f)^T, \\
%
% elasticity equilibrium equation
  \vec{0} &= \vec{f}(\vec{x},t) + \tensor{\nabla} \cdot \tensor{\sigma}(\vec{u},p_f) 
\text{ in }\Omega, \\
%
% fluid mass balance
  \frac{\partial \zeta(\vec{u},p_f)}{\partial t} &= \gamma(\vec{x},t) - \nabla \cdot \vec{q}
(p_f) \text{ in }
\Omega, \\
%
% Darcy's law
  \vec{q}(p_f) &= -\kappa (\nabla p_f - \vec{f}_f), \\
%
  \tensor{\sigma} \cdot \vec{n} &= \vec{\tau}(\vec{x},t) \text{ on }\Gamma_\tau, \\
%
  \zeta(\vec{u},p_f) &= \alpha (\nabla \cdot \vec{u}) + \frac{p_f}{M}, \\
%
  \vec{u} &= \vec{u}_0(\vec{x},t) \text{ on }\Gamma_u, \\
%
  \vec{q} \cdot \vec{n} &= q_0(\vec{x},t) \text{ on }\Gamma_q, \\
%
  p_f &= p_0(\vec{x},t) \text{ on }\Gamma_p.
\end{align}
For trial functions $\trialvec[u]$ and $\trialscalar[p]$ we write the weak form
using the elasticity equilibrium and the fluid mass balance equations,
\begin{align}
  0 &= \int_\Omega \trialvec[u] \cdot \left( \vec{f}(\vec{x},t) + \tensor{\nabla} \cdot 
\tensor{\sigma}
(\vec{u},p_f) \right) \, d\Omega, \\
%
 \int_\Omega  \trialscalar[p] \frac{\partial \zeta(\vec{u},p_f)}{\partial t} \, d\Omega &= 
\int_\Omega 
\trialscalar[p] \left(\gamma(\vec{x},t) - \nabla \cdot \vec{q}(p_f)\right) \, d\Omega.
\end{align}
Applying the divergence theorem to each of these two equations and incorporating the Neumann 
boundary conditions 
yields
\begin{align}
  0 &= \int_\Omega \trialvec[u] \cdot \vec{f}(\vec{x},t) + \nabla \trialvec[u] : -
\tensor{\sigma}(\vec{u},p_f) \, 
d\Omega + \int_{\Gamma_\tau} \trialvec[u] \cdot \vec{\tau}(\vec{x},t) \, d\Gamma, \\
%
 \int_\Omega  \trialscalar[p] \frac{\partial \zeta(\vec{u},p_f)}{\partial t} \, d\Omega &= 
 \int_\Omega \trialscalar[p] \gamma(\vec{x},t) + \nabla \trialscalar[p] \cdot \vec{q}(p_f) \, 
d\Omega
 + \int_{\Gamma_q} \trialscalar[p] (-q_0(\vec{x},t)) \, d\Gamma.
\end{align}
Identifying $F(t,s,\dot{s})$ and $G(t,s)$ we have
\begin{alignat}{2}
  F^u(t,s,\dot{s}) &= \vec{0},
  & \quad
  G^u(t,s) &= \int_\Omega \trialvec[u] \cdot \eqnannotate{\vec{f}(\vec{x},t)}{g^u_0} + \nabla 
\trialvec[u] : 
\eqnannotate{-\tensor{\sigma}(\vec{u},p_f)}{g^u_1} \, d\Omega + \int_{\Gamma_\tau} 
\trialvec[u] \cdot 
\eqnannotate{\vec{\tau}(\vec{x},t)}{g^u_0} \, d\Gamma, \\
  F^p(t,s,\dot{s}) &= \int_\Omega  \trialscalar[p] \eqnannotate{\frac{\partial 
\zeta(\vec{u},p_f)}{\partial t}}
{f^p_0} \, d\Omega
  & \quad
  G^p(t,s) &= \int_\Omega \trialscalar[p] \eqnannotate{\gamma(\vec{x},t)}{g^p_0} + \nabla 
\trialscalar[p] \cdot 
\eqnannotate{\vec{q}(p_f)}{g^p_1} \, d\Omega
 + \int_{\Gamma_q} \trialscalar[p] (\eqnannotate{-q_0(\vec{x},t)}{g^p_0}) \, d\Gamma
\end{alignat}
