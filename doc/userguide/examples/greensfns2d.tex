\section{Example Generating and Using Green's Functions in Two Dimensions}
\label{sec:example:greensfns2d}

PyLith features discussed in this example:
\begin{itemize}
\item Green's functions
\item HDF5 output
\item HDF5 point output
\item Reading HDF5 output using h5py
\item Simple inversion procedure
\item Plotting results using matplotlib
\item Cubit mesh generation
\item Variable mesh resolution
\item APREPRO programming language
\item Static solution
\item Linear triangular cells
\item Kinematic fault interface conditions
\item Plane strain linearly elastic material
\item SimpleDB spatial database
\item ZeroDispDB spatial database
\item UniformDB spatial database
\end{itemize}
All of the files necessary to run the examples are contained under the
directory \filename{examples/2d/greensfns.}


\subsection{Overview}

This example examines the steps necessary to generate Green's
functions using PyLith and how they may be used in a linear
inversion. For simplicity we discuss strike-slip and reverse faulting
examples in the context of 2D simulations. In each example, we first
compute surface displacement at a set of points, and these computed
displacements provide the ``data'' for our inversion. Second, we
compute a set of Green's functions using the same fault geometries,
and output the results at the same set of points. Third, we perform a
simple linear inversion. An important aspect for both the forward
problem and the Green's function problem is that the computed solution
is output at a set of user-specified points (not necessarily
coincident with mesh vertices), rather than over a mesh or sub-mesh as
for other types of output. To do this, PyLith internally performs the
necessary interpolation. There is a README file in the top-level
directory that explains how to perform each step in the two problems.


\subsection{Mesh Description}

We use linear triangular cells for the meshes in each of the two
problems.  We construct the mesh in CUBIT following the same
techniques used in the 2D subduction zone example. The main driver is
in the journal file \filename{mesh\_tri3.jou}. It calls the journal
file \filename{geometry.jou} to construct the geometry. It then calls
the journal file \filename{gradient.jou} to set the variable
discretization sizes used in this mesh. Finally, the
\filename{createbc.jou} file is called to set up the groups associated
with boundary conditions and materials. The mesh used for the
strike-slip example is shown in Figure
\vref{fig:greensfns2d-strikeslip-mesh} The journal files are
documented and describe the various steps outlined below.

\begin{enumerate}
\item Create the geometry defining the domain.
\item Create fault surface by splitting domain across the given locations.
\item Define meshing scheme and cell size variation.
\item Define cell size along curves near fault.
\item Increase cell size away from fault at a geometric rate (bias).
\item Generate mesh.
\item Create blocks for materials and nodesets for boundary conditions.
\item Export mesh.
\end{enumerate}

\begin{figure}
  \includegraphics[width=4in]{examples/figs/greensfns2d_strikeslip_ydispl2}
  \caption{Mesh used for both forward and Green's function computations for the
    strike-slip problem. Computed y-displacements for the forward problem
    are shown with the color scale.}
  \label{fig:greensfns2d-strikeslip-mesh}
\end{figure}


\subsection{Additional Common Information}

As in the examples discussed in previous sections of these examples,
we place parameters common to the forward model and Green's function
computation in the \filename{pylithapp.cfg} file so that we do not have
to duplicate them for the two procedures. The settings contained in
\filename{pylithapp.cfg} for this problem consist of:
\begin{inventory}
  \facilityitem{pylithapp.journal.info}{Settings that control the verbosity of
    the output written to stdout for the different components.}
  \facilityitem{pylithapp.mesh\_generator}{Settings that control mesh importing,
    such as the importer type, the filename, and the spatial dimension
    of the mesh.}
  \facilityitem{pylithapp.problem}{Settings that control the problem, such as
    the total time, time-step size, and spatial dimension.}
  \facilityitem{pylithapp.problem.materials}{Settings that control the material
    type, specify which material IDs are to be associated with a particular
    material type, and give the name of the spatial database containing
    the physical properties for the material. The quadrature information
    is also given.}
  \facilityitem{pylithapp.problem.bc}{Settings that control the applied boundary
    conditions.}
  \facilityitem{pylithapp.problem.interfaces}{Settings that control the specification
    of faults, including quadrature information.}
  \facilityitem{pylithapp.problem.formulation.output}{Settings related to output
    of the solution over the domain and points (surface observation locations).}
  \facilityitem{pylithapp.petsc}{PETSc settings to use for the problem, such as
    the preconditioner type.}
\end{inventory}
One aspect that has not been covered previously is the specification
of output at discrete points, rather than over a mesh or sub-mesh.
We do this using the \object{OutputSolnPoints} output type:
\begin{cfg}[Excerpt from \filename{pylithapp.cfg}]
<h>[pylithapp.problem.formulation]</h>
<f>output</f> = [domain, points]
<f>output.points</f> = pylith.meshio.OutputSolnPoints

<h>[pylithapp.problem.formulation.output.points]</h>
<p>coordsys.space_dim</p> = 2
<p>coordsys.units</p> = km
<f>writer</f> = pylith.meshio.DataWriterHDF5
<p>reader.filename</p> = output_points.txt
\end{cfg}
We provide the number of spatial dimensions and the units of the point
coordinates, and then the coordinates are given in a simple ASCII
file (\filename{output\_points.txt}). These same points are used for
both the forward model computation and the generation of the Green's
functions.


\subsection{Step 1: Solution of the Forward Problem}

For both the strike-slip problem and the reverse fault problem, we
first run a static simulation to generate our synthetic
data. Parameter settings that augment those in
\filename{pylithapp.cfg} are contained in the file
\filename{eqsim.cfg}. These settings are:
\begin{inventory}
  \facilityitem{pylithapp.problem.interfaces}{Give the type of fault interface
    condition and provide the slip distribution to use. Linear interpolation
    is used for the slip distribution.}
  \facilityitem{pylithapp.problem.formulation.output}{Gives the output filenames
    for domain output, fault output, point output, and material output.
    All output uses HDF5 format.}
\end{inventory}
The applied fault slip is given in the file
\filename{eqslip.spatialdb}.  For both the strike-slip and reverse
problems, no fault opening is given, so only the left-lateral
component is nonzero. We run the forward models by typing (in the
appropriate directory)
\begin{shell}
$ pylith eqsim.cfg
\end{shell}
Once the problem has run, four HDF5 files will be produced. The file
named \filename{eqsim.h5} (and the associated XDMF file) contains the
solution for the entire domain. This corresponds to the solution shown
in Figure \vref{fig:greensfns2d-strikeslip-mesh}. The
\filename{eqsim-fault.h5} file contains the applied fault slip and the
change in fault tractions, while the \filename{eqsim-fault\_info.h5}
file contains the final slip, the fault normal, and the slip time. The
final file (\filename{eqsim-points.h5}) contains the solution computed
at the point locations provided in the \filename{output\_points.txt}
file. These are the results that will be used as synthetic data for
our inversion. One the problem has run, the results may be viewed with
a visualization package such as ParaView.  In Figure
\vref{fig:greensfns2d-strikeslip-forward} we show the applied fault
slip (from \filename{eqsim-fault.h5}) and the resulting
x-displacements (from \filename{eqsim-points.h5}) for our strike-slip
forward problem.

\begin{figure}
  \includegraphics[scale=0.33]{examples/figs/greensfns2d_strikeslip_forward_points}
  \caption{Applied fault slip for the strike-slip forward problem as well as
    computed x-displacements at a set of points.}
  \label{fig:greensfns2d-strikeslip-forward}
\end{figure}


\subsection{Step 2: Generation of Green's Functions}

The next step is to generate Green's functions that may be used in
an inversion. The procedure is similar to that for running the forward
problem; however, it is necessary to change the problem type from
the default \facility{timedependent} to \facility{greensfns}. This is
accomplished by simply typing
\begin{shell}
pylith --problem=pylith.problems.GreensFns
\end{shell}
This changes the problem type and it also causes PyLith to read the
file \filename{greensfns.cfg} by default, in addition to \filename{pylithapp.cfg}.
These additional parameter settings provide the information necessary
to generate the Green's functions:
\begin{cfg}[Excerpt from \filename{greensfns.cfg}]
<h>[greensfns]</h>
<p>fault_id</p> = 100

# Set the type of fault interface condition.
<h>[greensfns.interfaces]</h>
<f>fault</f> = pylith.faults.FaultCohesiveImpulses

# Set the parameters for the fault interface condition.
<h>[greensfns.interfaces.fault]</h>
# Generate impulses for lateral slip only, no fault opening.
# Fault DOF 0 corresponds to left-lateral slip.
<p>impulse_dof</p> = [0]

# Set the amplitude of the slip impulses (amplitude is nonzero on only
# a subset of the fault)
<p>db_impulse_amplitude.label</p> = Amplitude of slip impulses
<p>db_impulse_amplitude.iohandler.filename</p> = impulse_amplitude.spatialdb
<p>db_impulse_amplitude.query_type</p> = nearest
\end{cfg}
Note that the top-level identifier is now \facility{greensfns} rather
than \facility{pylithapp}. We first set the fault interface condition
type to \object{FaultCohesiveImpulses}, and then specify the slip
component to use. The amplitude of the fault slip and the fault
vertices to use are provided in the
\filename{impulse\_amplitude.spatialdb} file.  Fault vertices for
which zero slip is specified will not have associated Green's
functions generated. The remainder of the \filename{greensfns.cfg}
file provides output information, which is exactly analogous to the
settings in \filename{eqsim.cfg}.

The generation of Green's functions is somewhat similar to the
solution of a time-dependent problem with multiple time steps. In this
case, each `time step' corresponds to the solution computed for a slip
impulse at a particular fault vertex. The output files contain the
solution for each separate impulse (slip on a single fault
vertex). The \filename{greensfns-fault\_info.h5} file simply contains
the slip amplitude and fault normal. In Figure
\vref{fig:greensfns2d-strikeslip-gf6} we show the applied impulse
(from file \filename{greensfns-fault.h5}) and associated point
responses (from file \filename{greensfns-points.h5}) for the seventh
generated Green's function in the strike-slip example. In the next
section we will show how to read these Green's functions and use them
to perform a simple linear inversion.

\begin{figure}
  \includegraphics[scale=0.33]{examples/figs/greensfns2d_strikeslip_gf6}
  \caption{Applied fault slip and computed responses (at points) for the seventh
    Green's function generated for the strike-slip fault example.}
  \label{fig:greensfns2d-strikeslip-gf6}
\end{figure}


\subsection{Step 3: Simple Inversion Using PyLith-generated Green's Functions}
\label{sec:example:greensfns2d:inversion}

In the previous two steps we generated a set of synthetic data as
well as a set of Green's functions. Both are stored in HDF5 files.
To make use of them, we provide a simple Python script that reads
the HDF5 results using the h5py Python package. Once we have read
the necessary information, we will perform a simple least-squares
inversion using the penalty method. We will be solving the equation:
\begin{equation}
G_{a}m=d_{a}\:,
\end{equation}
where $m$ are the model parameters (slip), $G_{a}$ is the augmented
set of Green's functions, and $d_{a}$ is the augmented data vector.
The Green's functions are augmented by the addition of a penalty function:
\begin{equation}
G_{a}=\left[\begin{array}{c}
G\\
\lambda D
\end{array}\right]\:,
\end{equation}
and the data vector is augmented by the addition of the \textit{a
priori} model parameter values:
\begin{equation}
d_{a}=\left[\begin{array}{c}
d\\
m_{ap}
\end{array}\right]\:.
\end{equation}
The matrix $D$ is the penalty function, and $\lambda$ is the penalty
parameter. The solution is obtained using the generalized inverse
(e.g., \cite{Menke:1984}):
\begin{equation}
G^{-g}=\left(G_{a}^{T}G_{a}\right)^{-1}G_{a}^{T}\:,
\end{equation}
and the estimated solution is then:
\begin{equation}
m_{est}=G^{-g}d_{a}\:.
\end{equation}


The code to read the synthetic data and Green's functions and to perform
the inversion is contained in the file \filename{invert\_slip.py}, which
is contained in the top-level directory. For this simple example,
we have simply used a diagonal matrix as the penalty funtion, and
the \textit{a priori} parameter values are assumed to be zero. The
solution is performed for a range of values of the penalty parameter,
which are contained in the file \filename{penalty\_params.txt} within
each subdirectory. The inversion is performed by running the script
in the top-level directory from each subdirectory. 

\begin{shell}[Run the inversion]
$ ../invert_slip.py --impulses=output/greensfns-fault.h5 \
   --responses=output/greensfns-points.h5 --data=output/eqsim-points.h5 \
   --penalty=penalty_params.txt --output=output/slip_inverted.txt \
\end{shell}
This will produce an ASCII file (\filename{slip\_inverted.txt}), which
will contain the estimated solution.


\subsection{Step 4: Visualization of Estimated and True Solutions}

Once we have computed the solution, we would then like to visualize
the results. We do this using another Python script that requires
the matplotlib plotting package (this package is not included in the
PyLith binary). We also use the h5py package again to read the applied
slip for the forward problem. The Python code to plot the results
is contained in the \filename{plot\_invresults.py} file contained within
each subdirectory.
\begin{shell}[Plot the results (requires Matplotlib)]
$ plot_invresults.py --solution=output/eqsim-fault.h5 --predicted=output/slip_inverted.txt
\end{shell}
The script will produce an interactive matplotlib window that shows
the estimated solution compared to the true solution (Figure
\vref{fig:greensfns-invresults}).  As the penalty parameter is
increased, the solution is progressively damped. In a real inversion
we would also include the effects of data uncertainties, and the
penalty parameter would represent a factor controlling the tradeoff
between solution simplicity and fitting the noise in the data.

\begin{figure}
  \includegraphics[width=3in]{examples/figs/greensfns2d_strikeslip_inversion}
  \includegraphics[width=3in]{examples/figs/greensfns2d_reverse_inversion}
  \caption{Inversion results from running Python plotting script.}
  \label{fig:greensfns-invresults}
\end{figure}


% End of file
