\section{Derivation of Elasticity Equation}

\subsection{Vector Notation}

Consider volume $V$ bounded by surface $S$. Applying a Lagrangian
description of the conservation of momentum gives
\begin{equation}
\label{eqn:momentum:vec}
\frac{\partial}{\partial t}\int_{V}\rho\frac{\partial\vec{u}}{\partial t}\, dV=\int_{V}\overrightarrow{f}\, dV+\int_{S}\overrightarrow{T}\, dS.
\end{equation}
The traction vector field is related to the stress tensor through
\begin{equation}
\overrightarrow{T}=\underline{\sigma}\cdot\overrightarrow{n},
\end{equation}
where $\overrightarrow{n}$ is the vector normal to $S$. Substituting
into equation \vref{eqn:momentum:vec} yields
\begin{equation}
\frac{\partial}{\partial t}\int_{V}\rho\frac{\partial\overrightarrow{u}}{\partial t}\, dV=\int_{V}\overrightarrow{f}\, dV+\int_{S}\underline{\sigma}\cdot\overrightarrow{n}\, dS.
\end{equation}
Applying the divergence theorem,
\begin{equation}
\int_{V}\nabla\cdot\overrightarrow{a}\: dV=\int_{S}\overrightarrow{a}\cdot\overrightarrow{n}\: dS,
\end{equation}
to the surface integral results in
\begin{equation}
\frac{\partial}{\partial t}\int_{V}\rho\frac{\partial\overrightarrow{u}}{\partial t}\, dV=\int_{V}\overrightarrow{f}\, dV+\int_{V}\nabla\cdot\underline{\sigma}\, dV,
\end{equation}
which we can rewrite as
\begin{equation}
\int_{V}\left(\rho\frac{\partial^{2}\overrightarrow{u}}{\partial t^{2}}-\overrightarrow{f}-\nabla\cdot\overrightarrow{\sigma}\right)\, dV=\vec{0}.
\end{equation}
Because the volume $V$ is arbitrary, the integrand must be the zero
vector at every location in the volume, so that we end up with
\begin{gather}
\rho\frac{\partial^{2}\overrightarrow{u}}{\partial t^{2}}-\overrightarrow{f}-\nabla\cdot\overrightarrow{\sigma}=\vec{0}\text{ in }V,\\
\underline{\sigma}\cdot\overrightarrow{n}=\overrightarrow{T}\text{ on }S_{T}\text{,}\\
\overrightarrow{u}=\overrightarrow{u^{o}}\text{ on }S_{u},\text{ and}\\
\underbar{R}\cdot(\vec{u^{+}}-\vec{u^{-}})=\vec{d}\text{ on }S_{f}.
\end{gather}
We specify tractions, $\vec{T}$, on surface $S_{f}$, displacements,
$\overrightarrow{u^{o}}$, on surface $S_{u}$, and slip, $\vec{d}$,
on fault surface $S_{f}$ (we will consider the case of fault constitutive
models in Section \vref{sec:fault}). The rotation matrix $\underline{R}$
transforms vectors from the global coordinate system to the fault
coordinate system. Note that since both $\overrightarrow{T}$ and
$\overrightarrow{u}$ are vector quantities, there can be some spatial
overlap of the surfaces $S_{T}$ and $S_{u}$; however, the same degree
of freedom cannot simultaneously have both types of boundary conditions.
