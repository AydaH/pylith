% ----------------------------------------------------------------------
\section{Elasticity with Infinitesimal Strain and Prescribed Slip on Faults}

For each fault, which is an internal interface, we add a boundary
condition to the elasticity equation prescribing the jump in the
displacement field across the fault,
\begin{gather}
  \label{eqn:bc:prescribed_slip}
  \vec{u}^+(\vec{x},t) - \vec{u}^-(\vec{x},t) - \vec{d}(\vec{x},t) = \vec{0} \text{ on }\Gamma_f,
\end{gather}
where $\vec{u}^+$ is the displacement vector on the ``positive'' side
of the fault, $\vec{u}^-$ is the displacement vector on the
``negative'' side of the fault, $\vec{d}$ is the slip vector on the
fault, and $\vec{n}$ is the fault normal which points from the
negative side of the fault to the positive side of the fault. Note
that as an alternative to prescribing the jump in displacement across
the fault, we can also prescribe the jump in velocity or acceleration
across the fault in terms of slip rate or slip acceleration,
respectively.

Using a domain decomposition approach for constraining the fault slip
yields additional Neumann-like boundary conditions on the fault
surface,
\begin{gather}
  \tensor{\sigma} \cdot \vec{n} = -\vec{\lambda}(\vec{x},t) \text{ on }\Gamma_{f^+}, \\
  \tensor{\sigma} \cdot \vec{n} = +\vec{\lambda}(\vec{x},t) \text{ on }\Gamma_{f^-},
\end{gather}
where $\vec{\lambda}$ is the vector of Lagrange multipliers
corresponding to the tractions applied to the fault surface to
generate the prescribed slip. 

\begin{table}[htbp]
  \caption{Mathematical notation for elasticity equation with
    infinitesimal strain and prescribed slip on faults.}
  \label{tab:notation:elasticity:prescribed:slip}
  \begin{tabular}{lcp{3in}}
    \toprule
    {\bf Category} & {\bf Symbol} & {\bf Description} \\
    \midrule
    Unknowns & $\vec{u}$ & Displacement field \\
    & $\vec{v}$ & Velocity field \\
    & $\vec{\lambda}$ & Lagrange multiplier field \\
    Derived quantities & $\tensor{\sigma}$ & Cauchy stress tensor \\
                   & $\tensor{\epsilon}$ & Cauchy strain tensor \\
    Common constitutive parameters & $\rho$ & Density \\
  & $\mu$ & Shear modulus \\
  & $K$ & Bulk modulus \\
Source terms & $\vec{f}$ & Body force per unit volume, for example $\rho \vec{g}$ \\
    & $\vec{d}$ & Slip vector field on the fault corresponding to a
      jump in the displacement field across the fault \\
    \bottomrule
  \end{tabular}
\end{table}

\subsection{Quasistatic}

As in the case of elasticity without faults, we first consider the
quasistatic case in which we neglect the inertial term
($\rho \frac{\partial \vec{v}}{\partial t} \approx \vec{0}$). We place
all of the terms in the elasticity equation on the LHS, consistent
with implicit time stepping. Our solution vector consists of both
displacements and Lagrange multipliers, and the strong form for the
system of equations is
\begin{gather}
  % Solution
  \vec{s}^T = \left( \vec{u} \quad \vec{\lambda} \right)^T \\
  % Elasticity
  \vec{f}(\vec{x},t) + \tensor{\nabla} \cdot \tensor{\sigma}(\vec{u}) = \vec{0} \text{ in }\Omega, \\
  % Neumann
  \tensor{\sigma} \cdot \vec{n} = \vec{\tau}(\vec{x},t) \text{ on }\Gamma_\tau, \\
  % Dirichlet
  \vec{u} = \vec{u}_0(\vec{x},t) \text{ on }\Gamma_u, \\
  % Prescribed slip
  \vec{u}^+(\vec{x},t) - \vec{u}^-(\vec{x},t) - \vec{d}(\vec{x},t) = \vec{0} \text{ on }\Gamma_f,  \\
  \tensor{\sigma} \cdot \vec{n} = -\vec{\lambda}(\vec{x},t) \text{ on }\Gamma_{f^+}, \text{ and}\\
  \tensor{\sigma} \cdot \vec{n} = +\vec{\lambda}(\vec{x},t) \text{ on }\Gamma_{f^-}.
\end{gather}
We create the weak form by taking the dot product with the trial
function $\trialvec[u]$ or $\trialvec[\lambda]$ and integrating over
the domain. After using the divergence theorem and incorporating the
Neumann boundary and fault interface conditions, we have
\begin{gather}
  % Elasticity
  \int_\Omega \trialvec[u] \cdot \vec{f}(\vec{x},t) + \nabla \trialvec[v] : -\tensor{\sigma}(\vec{u}) \, d\Omega
  + \int_{\Gamma_\tau} \trialvec[u] \cdot \vec{\tau}(\vec{x},t) \, d\Gamma,
  + \int_{\Gamma_{f}} \trialvec[u^+] \cdot \left(-\vec{\lambda}(\vec{x},t)\right)
  + \trialvec[u^-] \cdot \left(+\vec{\lambda}(\vec{x},t)\right)\, d\Gamma = 0\\
  % Prescribed slip
  \int_{\Gamma_{f}} \trialvec[\lambda] \cdot \left(
    -\vec{u}^+(\vec{x},t) + \vec{u}^-(\vec{x},t) + \vec{d}(\vec{x},t) \right) \, d\Gamma = 0.
\end{gather}
We have chosen the signs in the last equation so that the Jacobian
will be symmetric with respect to the Lagrange multiplier. We solve
the system of equations using implicit time stepping, making use of
residuals functions and Jacobians for the LHS.


\subsubsection{Residual Pointwise Functions}

Identifying $F(t,s,\dot{s})$ and $G(t,s)$, we have
\begin{align}
 % Fu
F^u(t,s,\dot{s}) &= \int_\Omega \trialvec[u] \cdot \eqnannotate{\vec{f}(\vec{x},t)}{\vec{f}^u_0} + \nabla \trialvec[u] : \eqnannotate{-\tensor{\sigma}(\vec{u})}{\tensor{f^u_1}} \, d\Omega
  + \int_{\Gamma_\tau} \trialvec[u] \cdot \eqnannotate{\vec{\tau}(\vec{x},t)}{\vec{f}^u_0} \, d\Gamma 
  + \int_{\Gamma_{f}} \trialvec[u^+] \cdot \eqnannotate{\left(-\vec{\lambda}(\vec{x},t)\right)}{\vec{f}^u_0}
  + \trialvec[u^-] \cdot \eqnannotate{\left(+\vec{\lambda}(\vec{x},t)\right)}{\vec{f}^u_0}\, d\Gamma \\
  % Fl
  F^\lambda(t,s,\dot{s}) &= \int_{\Gamma_{f}} \trialvec[\lambda] \cdot \eqnannotate{\left(
    -\vec{u}^+(\vec{x},t) + \vec{u}^-(\vec{x},t) + \vec{d}(\vec{x},t) \right)}{\vec{f}^\lambda_0} \, d\Gamma, \\
  % Gu
  G^u(t,s) &= 0 \\
  % Gl
  G^\lambda(t,s) &= 0
\end{align}
Compared to the quasistatic elasticity case without a fault, we have
simply added additional pointwise functions associated with the
fault. Our fault implementation does not change the formulation for
the materials or external Dirichlet or Neumann boundary conditions.

\subsubsection{Jacobian Pointwise Functions}

The LHS Jacobians are:
\begin{align}
  % J_F uu
  J_F^{uu} &= \frac{\partial F^u}{\partial u} + s_\mathit{tshift} \frac{\partial F^u}{\partial \dot{u}}
      = \int_\Omega \nabla \trialvec[u] : -\tensor{C} : \frac{1}{2}(\nabla + \nabla^T)\basisvec[u] 
\, d\Omega 
      = \int_\Omega \trialscalar[u]_{i,k} \, \eqnannotate{\left( -C_{ikjl} \right)}{J_{f3}^{uu}} \, \basisscalar[u]_{j,l}\, d\Omega \\
  % J_F ul
  J_F^{u\lambda} &= \frac{\partial F^u}{\partial \lambda} + s_\mathit{tshift} \frac{\partial F^u}{\partial \dot{\lambda}}
      = \int_{\Gamma_{f}} \trialscalar[u^+]_i \eqnannotate{\left(-\delta_{ij}\right)}{J^{u\lambda}_{f0}} \basisscalar[\lambda]_j
                   + \trialscalar[u^-]_i \eqnannotate{\left(+\delta_{ij}\right)}{J^{u\lambda}_{f0}} \basisscalar[\lambda]_j\, d\Gamma, \\
  % J_F lu
  J_F^{\lambda u} &= \frac{\partial F^\lambda}{\partial u} + s_\mathit{tshift} \frac{\partial F^\lambda}{\partial \dot{u}}
      = \int_{\Gamma_{f}} \trialscalar[\lambda]_i 
                    \eqnannotate{\left(-\delta_{ij}\right)}{J^{\lambda u}_{f0}} \basisscalar[u^+]_j
                    + \trialscalar[\lambda]_i \eqnannotate{\left(+\delta_{ij}\right)}{J^{\lambda u}_{f0}} \basisscalar[u^-]_j \, d\Gamma, \\
  % J_F ll
  J_F^{\lambda \lambda} &= \tensor{0}
\end{align}
This LHS Jacobian has the structure
\begin{equation}
  J_F = \left( \begin{array} {cc} J_F^{uu} & J_F^{u\lambda} \\ J_F^{\lambda u} & 0 \end{array} \right)
      = \left( \begin{array} {cc} J_F^{uu} & C^T \\ C & 0 \end{array} \right),
\end{equation}
where $C$ contains entries of $\pm 1$ for degrees of
freedom on the two sides of the fault. The Schur complement of $J$
with respect to $J_F^{uu}$ is $-C\left(J_F^{uu}\right)^{-1}C^T$.


\subsection{Dynamic}

The equation prescribing fault slip is independent of the Lagrange
multiplier, so we do not have a system of equations that we can put in
the form $\dot{s} = G^*(t,s)$. Instead, we have a
differential-algebraic set of equations (DAEs), which we solve using
an implicit-explicit (IMEX) time integration scheme. As in the case of
dynamic elasticity without faults, we introduce the velocity
($\vec{v}$) as an unknown to turn the elasticity equation into two
first order equations. The strong form for our system of equations is:
\begin{gather}
  % Solution
  \vec{s}^T = \left( \vec{u} \quad \vec{v} \quad \vec{\lambda} \right)^T \\
  % Displacement-velocity
  \frac{\partial \vec{u}}{\partial t} = \vec{v}, \\
  % Elasticity
  \rho(\vec{x}) \frac{\partial \vec{v}}{\partial t} =
  \vec{f}(\vec{x},t) + \nabla \cdot \tensor{\sigma}(\vec{u}), \\
  % Neumann BC
  \tensor{\sigma} \cdot \vec{n} = \vec{\tau} \text{ on } \Gamma_\tau. \\
  % Dirichlet BC
  \vec{u} = \vec{u}_0 \text{ on } \Gamma_u, \\
  % Presribed slip
  \vec{u}^+(\vec{x},t) - \vec{u}^-(\vec{x},t) - \vec{d}(\vec{x},t) = \vec{0} \text{ on }\Gamma_f,  \\
  \tensor{\sigma} \cdot \vec{n} = -\vec{\lambda}(\vec{x},t) \text{ on }\Gamma_{f^+}, \text{ and}\\
  \tensor{\sigma} \cdot \vec{n} = +\vec{\lambda}(\vec{x},t) \text{ on }\Gamma_{f^-}.
\end{gather}
The differentiation index is 2 because we must take the second time
derivative of the prescribed slip equation to match the order of the
time derivative in the elasticity equation,
\begin{gather}
  \frac{\partial \vec{v}^+}{\partial t} - \frac{\partial \vec{v}^-}{\partial t} -
  \frac{\partial^2 \vec{d}(\vec{x},t)}{\partial t^2} = \vec{0}.
\end{gather}
We generate the weak form in the usual way,
\begin{gather}
  % Displacement-velocity
  \int_{\Omega} \trialvec[u] \cdot \frac{\partial \vec{u}}{\partial t} \, d\Omega = 
  \int_{\Omega} \trialvec[u] \cdot \vec{v} \, d\Omega, \\
  % Elasticity
  \begin{multlined}
  \int_{\Omega} \trialvec[v] \cdot \rho(\vec{x}) \frac{\partial \vec{v}}{\partial t} \, d\Omega 
  = \int_\Omega \trialvec[v] \cdot \vec{f}(\vec{x},t) + \nabla \trialvec[v] : -\tensor{\sigma}(\vec{u}) \, d\Omega
  + \int_{\Gamma_\tau} \trialvec[v] \cdot \vec{\tau}(\vec{x},t) \, d\Gamma \\
  + \int_{\Gamma_{f}} \trialvec[v^+] \cdot \left(-\vec{\lambda}(\vec{x},t)\right)
  + \trialvec[v^-] \cdot \left(+\vec{\lambda}(\vec{x},t)\right)\, d\Gamma,
  \end{multlined}\\
  % Prescribed slip
  \int_{\Gamma_f} \trialvec[\lambda] \cdot \left(
    \frac{\partial \vec{v}^+}{\partial t} - \frac{\partial \vec{v}^-}{\partial t} -
    \frac{\partial^2 \vec{d}(\vec{x},t)}{\partial t^2} \right) \, d\Gamma = 0.
\end{gather}

For compatibility with PETSc TS IMEX implementations, we need
$\dot{\vec{s}}$ on the LHS for the explicit part (displacement-velocity and
elasticity equations) and we need $\vec{\lambda}$ in the equation for
the implicit part (prescribed slip equation). We first focus on the
explicit part and create a lumped LHS Jacobian matrix, $M$, so that we
have
\begin{gather}
  % Displacement-velocity
  \label{eqn:displacement:velocity:prescribed:slip:weak:form}
  \frac{\partial \vec{u}}{\partial t} = M_u^{-1} \int_{\Omega} \trialvec[u] \cdot \vec{v} \, d\Omega, \\
  % Elasticity
  \label{eqn:elasticity:prescribed:slip:dynamic:weak:form}
  \begin{multlined}
  \frac{\partial \vec{v}}{\partial t}
  = M_v^{-1} \int_\Omega \trialvec[v] \cdot \vec{f}(\vec{x},t) + \nabla \trialvec[v] : -\tensor{\sigma}(\vec{u}) \, d\Omega
  + M_v^{-1} \int_{\Gamma_\tau} \trialvec[v] \cdot \vec{\tau}(\vec{x},t) \, d\Gamma \\
  + M_{v^+}^{-1} \int_{\Gamma_{f}} \trialvec[v^+] \cdot \left(-\vec{\lambda}(\vec{x},t)\right) \, d\Gamma
  + M_{v^-}^{-1} \int_{\Gamma_{f}}\trialvec[v^-] \cdot \left(+\vec{\lambda}(\vec{x},t)\right) \, d\Gamma,
\end{multlined}\\
% Mu
M_u = \mathit{Lump}\left( \int_\Omega \trialscalar[u]_i \delta_{ij} \basisscalar[u]_j \, d\Omega \right), \\
% Mv
M_v = \mathit{Lump}\left( \int_\Omega \trialscalar[v]_i \rho(\vec{x}) \delta_{ij} \basisscalar[v]_j \, d\Omega \right).
\end{gather}
Now, focusing on the implicit part we want to introduce
$\vec{\lambda}$ into the prescribed slip equation. We solve the
elasticity equation for $\frac{\partial \vec{v}}{\partial t}$, create
the weak form, and substitute into the prescribed slip
equation. Solving the elasticity equation for
$\frac{\partial \vec{v}}{\partial t}$, we have
\begin{equation}
  \frac{\partial \vec{v}}{\partial t} = \frac{1}{\rho(x)} \vec{f}(\vec{x},t) + \frac{1}{\rho(x)} \left(\nabla \cdot \tensor{\sigma}(\vec{u}) \right),
\end{equation}
and the corresponding weak form is
\begin{equation}
  \label{eqn:prescribed:slip:DAE:weak:form}
  \int_{\Omega} \trialvec[v] \cdot \frac{\partial \vec{v}}{\partial t} \, d\Omega
  = \int_\Omega \trialvec[v] \cdot \frac{1}{\rho(x)} \vec{f}(\vec{x},t) + \trialvec[v] \cdot \frac{1}{\rho(x)} \left(\nabla \cdot \tensor{\sigma}(\vec{u}) \right) \, d\Omega,
\end{equation}
We apply the divergence theorem,
\begin{equation}
  \int_{\Omega} \nabla \cdot \vec{F} \, d\Omega = \int_\Gamma \vec{F} \cdot \vec{n} \, d\Gamma,
\end{equation}
with $\vec{F} = \trialvec[v] \cdot \frac{1}{\rho(x)} \left(\nabla \cdot \tensor{\sigma}(\vec{u})\right)$ to get
\begin{equation}
  \int_\Omega \trialvec[v] \cdot \frac{1}{\rho(x)} \left(\nabla \cdot \tensor{\sigma}(\vec{u}) \right) \, d\Omega
  = \int_\Gamma \trialvec[v] \cdot \left( \frac{1}{\rho(x)} \tensor{\sigma}(\vec{u}) \cdot \vec{n} \right) \, d\Gamma
  + \int_\Omega \nabla\trialvec[v] : \left(-\frac{1}{\rho(\vec{x})} \tensor{\sigma}(\vec{u}) \right)
  + \trialvec[v] \cdot \left(-\frac{\nabla \rho(\vec{x})}{\rho^2} \cdot \tensor{\sigma}(\vec{u}) \right) \, d\Omega.
\end{equation}
Restricting the trial function to $v^+$ and $v^-$ while recognizing that there is no overlap between the external Neumann boundary conditions $\Gamma_\tau$ and the fault interfaces $\Gamma_f$, yields
\begin{gather}
  \int_\Omega \trialvec[v^+] \cdot \frac{1}{\rho(x)} \left(\nabla \cdot \tensor{\sigma}(\vec{u}) \right) \, d\Omega
  = \int_{\Gamma_f} \trialvec[v^+] \cdot \left(-\frac{1}{\rho(x)} \vec{\lambda} \right) \, d\Gamma
  + \int_\Omega \nabla\trialvec[v^+] : \left(-\frac{1}{\rho(\vec{x})} \tensor{\sigma}(\vec{u}) \right)
  + \trialvec[v^+] \cdot \, \left(-\frac{\nabla \rho(\vec{x})}{\rho^2} \cdot \tensor{\sigma}(\vec{u}) \right) \, d\Omega, \\
  \int_\Omega \trialvec[v^-] \cdot \frac{1}{\rho(x)} \left(\nabla \cdot \tensor{\sigma}(\vec{u}) \right) \, d\Omega
  = \int_{\Gamma_f} \trialvec[v^-] \cdot \left(+\frac{1}{\rho(x)} \vec{\lambda} \right) \, d\Gamma
  + \int_\Omega \nabla\trialvec[v^-] : \left(-\frac{1}{\rho(\vec{x})} \tensor{\sigma}(\vec{u}) \right)
  + \trialvec[v^-] \cdot \, \left(-\frac{\nabla \rho(\vec{x})}{\rho^2} \cdot \tensor{\sigma}(\vec{u}) \right) \, d\Omega. \end{gather}
Picking $\trialvec[v]=\trialvec[\lambda]$ and substituting into equation~\vref{eqn:prescribed:slip:DAE:weak:form} gives
\begin{multline}
  \int_{\Gamma_f} \trialvec[\lambda] \cdot \left(
    \frac{\partial \vec{v}^+}{\partial t} - \frac{\partial \vec{v}^-}{\partial t} -
    \frac{\partial^2 \vec{d}(\vec{x},t)}{\partial t^2} \right) \, d\Gamma = \\
  \int_\Omega \trialvec[v^+] \cdot \left( \frac{1}{\rho(\vec{x})} \vec{f}(\vec{x}, t)
    -\frac{\nabla \rho(\vec{x})}{\rho^2(\vec{x})} \cdot \tensor{\sigma}(\vec{u}) \right) 
  + \nabla \trialvec[v^+] : \left(-\frac{1}{\rho(\vec{x})} \tensor{\sigma}(\vec{u}) \right) \, d\Omega
  + \int_{\Gamma_f} \trialvec[v^+] \cdot \left(-\frac{1}{\rho(\vec{x})} \vec{\lambda} \right) \, d\Gamma \\
  - \int_\Omega \trialvec[v^-] \cdot \left( \frac{1}{\rho(\vec{x})} \vec{f}(\vec{x}, t)
  -\frac{\nabla \rho(\vec{x})}{\rho^2(\vec{x})} \cdot \tensor{\sigma}(\vec{u}) \right)
  + \nabla \trialvec[v^-] : \left(-\frac{1}{\rho(\vec{x})} \tensor{\sigma}(\vec{u}) \right) \, d\Omega
  + \int_{\Gamma_f} \trialvec[v^-] \cdot \left(-\frac{1}{\rho(\vec{x})} \vec{\lambda} \right) \, d\Gamma \\
  - \int_{\Gamma_f} \trialvec[\lambda] \cdot \frac{\partial^2 \vec{d}(\vec{x}, t)}{\partial t^2} \, d\Gamma.
\end{multline}
We rewrite the integrals over the domain involving the degrees of
freedom adjacent to the fault as integrals over the positive and
negative sides of the fault. These are implemented as integrals over
the faces of cells adjacent to the fault; they involve quantities,
such as density, that are defined only within the domain cells. After
collecting and rearranging terms, we have
\begin{multline}
  \label{eqn:elasticity:prescribed:slip:dynamic:DAE:weak:form}
  \int_{\Gamma_{f^+}} \trialvec[\lambda] \cdot \frac{1}{\rho(\vec{x})} \left(
    \vec{\lambda} - \vec{f}(\vec{x},t) + \frac{\nabla\rho(\vec{x})}{\rho(\vec{x})} \cdot \tensor{\sigma}(\vec{u}) \right)
  + \nabla \trialvec[\lambda] : \left(+\frac{1}{\rho(\vec{x})} \tensor{\sigma}(\vec{u}) \right) \, d\Gamma \\
  + \int_{\Gamma_{f^-}} \trialvec[\lambda] \cdot \frac{1}{\rho(\vec{x})} \left(
    \vec{\lambda} + \vec{f}(\vec{x},t) - \frac{\nabla\rho(\vec{x})}{\rho(\vec{x})} \cdot \tensor{\sigma}(\vec{u}) \right)
  + \nabla \trialvec[\lambda] : \left(-\frac{1}{\rho(\vec{x})} \tensor{\sigma}(\vec{u}) \right)  \, d\Gamma \\
  + \int_{\Gamma_f} \trialvec[\lambda] \cdot \frac{\partial^2 \vec{d}(\vec{x}, t)}{\partial t^2} \, d\Gamma
  = 0.
\end{multline}


\subsection{Residual Pointwise Functions}

Combining the explicit parts of the weak form in equations~\ref{eqn:displacement:velocity:prescribed:slip:weak:form} and \ref{eqn:elasticity:prescribed:slip:dynamic:weak:form} with the implicit part of the weak form in equation~\ref{eqn:elasticity:prescribed:slip:dynamic:DAE:weak:form} and identifying $F(t,s,\dot{s})$ and $G(t,s)$, we have
\begin{gather}
  % Fu
  F^u(t,s,\dot{s}) = \frac{\partial \vec{u}}{\partial t} \\
  % Fv
  F^v(t,s,\dot{s}) = \frac{\partial \vec{v}}{\partial t} \\
  % Fl
  \begin{multlined}
    F^\lambda(t,s,\dot{s}) =   \int_{\Gamma_{f^+}} \trialvec[\lambda] \cdot \eqnannotate{\frac{1}{\rho(\vec{x})} \left(
    \vec{\lambda} - \vec{f}(\vec{x},t) + \frac{\nabla\rho(\vec{x})}{\rho(\vec{x})} \cdot \tensor{\sigma}(\vec{u}) \right)}{f^\lambda_0}
  + \nabla \trialvec[\lambda] : \eqnannotate{\left(+\frac{1}{\rho(\vec{x})} \tensor{\sigma}(\vec{u})\right)}{f^\lambda_1} \, d\Gamma \\
  + \int_{\Gamma_{f^-}} \trialvec[\lambda] \cdot \eqnannotate{\frac{1}{\rho(\vec{x})} \left(
    \vec{\lambda} + \vec{f}(\vec{x},t) - \frac{\nabla\rho(\vec{x})}{\rho(\vec{x})} \cdot \tensor{\sigma}(\vec{u}) \right)}{f^\lambda_0}
  + \nabla \trialvec[\lambda] : \eqnannotate{\left(-\frac{1}{\rho(\vec{x})} \tensor{\sigma}(\vec{u}) \right)}{f^\lambda_1}  \, d\Gamma \\
  + \int_{\Gamma_f} \trialvec[\lambda] \cdot \eqnannotate{\frac{\partial^2 \vec{d}(\vec{x}, t)}{\partial t^2}}{f^\lambda_0} \, d\Gamma
  \end{multlined}\\
  % Gu
  G^u(t,s) = \int_\Omega \trialvec[u] \cdot \eqnannotate{\vec{v}}{\vec{g}^u_0} \, d\Omega, \\
  % Gv
  G^v(t,s) =  \int_\Omega \trialvec[v] \cdot \eqnannotate{\vec{f}(\vec{x},t)}{\vec{g}^v_0} + \nabla \trialvec[v] : \eqnannotate{-\tensor{\sigma}(\vec{u})}{\tensor{g^v_1}} \, d\Omega
  + \int_{\Gamma_\tau} \trialvec[v] \cdot \eqnannotate{\vec{\tau}(\vec{x},t)}{\vec{g}^v_0} \, d\Gamma,
  + \int_{\Gamma_{f}} \trialvec[v^+] \cdot \eqnannotate{\left(-\vec{\lambda}(\vec{x},t)\right)}{\vec{g}^v_0}
             + \trialvec[v^-] \cdot \eqnannotate{\left(+\vec{\lambda}(\vec{x},t)\right)}{\vec{g}^v_0} \, d\Gamma, \\
  % Gl
  G^l(t,s) = 0
\end{gather}


\subsection{Jacobian Pointwise Functions}

For the explicit part we have pointwise functions for computing the
lumped LHS Jacobian. These are exactly the same pointwise functions as
in the dynamic case without a fault,
\begin{align}
  % J_F uu
  J_F^{uu} &= \frac{\partial F^u}{\partial u} + s_\mathit{tshift} \frac{\partial F^u}{\partial \dot{u}} =
             \int_\Omega \trialscalar[u]_i \eqnannotate{s_\mathit{tshift} \delta_{ij}}{J^{uu}_{f0}} \basisscalar[u]_j  \, d\Omega, \\
  % J_F vv
  J_F^{vv} &= \frac{\partial F^v}{\partial v} + s_\mathit{tshift} \frac{\partial F^v}{\partial \dot{v}} =
             \int_\Omega \trialscalar[v]_i \eqnannotate{\rho(\vec{x}) s_\mathit{tshift} \delta_{ij}}{J ^{vv}_{f0}} \basisscalar[v]_j \, d\Omega
\end{align}
For the implicit part, we have pointwise functions for the LHS Jacobians associated with the prescribed slip,
\begin{gather}
  \begin{multlined}
  % J_F lu
  J_F^{\lambda u} = \frac{\partial F^\lambda}{\partial u} + s_\mathit{tshift} \frac{\partial F^\lambda}{\partial \dot{u}} = \\
                    \int_{\Gamma_{f^+}} \trialscalar[\lambda]_i \eqnannotate{\frac{\rho_{,j}(\vec{x})}{\rho^2(\vec{x})} C_{ikjl} \basisscalar[u]_{j,l}}{J^{\lambda u}_{fX}}
                    + \trialscalar[\lambda]_{i,k} \eqnannotate{+\frac{1}{\rho(\vec{x})} C_{ikjl} \basisscalar[u]_{k,l}}{J^{\lambda u}_{f3}} \, d\Gamma \\
                    +\int_{\Gamma_{f^-}} \trialscalar[\lambda]_i \eqnannotate{-\frac{\rho_{,j}(\vec{x})}{\rho^2(\vec{x})} C_{ikjl} \basisscalar[u]_{j,l}}{J^{\lambda u}_{fX}}
                    + \trialscalar[\lambda]_{i,k} \eqnannotate{-\frac{1}{\rho(\vec{x})} C_{ikjl} \basisscalar[u]_{k,l}}{J^{\lambda u}_{f3}} \, d\Gamma
                  \end{multlined} \\
  % J_F ll
  J_F^{\lambda \lambda} = \frac{\partial F^\lambda}{\partial \lambda} + s_\mathit{tshift} \frac{\partial F^\lambda}{\partial \dot{\lambda}} =
             \int_{\Gamma_{f^+}} \trialscalar[\lambda]_i \eqnannotate{\frac{1}{\rho(\vec{x})} \delta_{ij}}{J^{\lambda\lambda}_{f0}} \basisscalar[\lambda]_j \, d\Gamma
            + \int_{\Gamma_{f^-}} \trialscalar[\lambda]_i \eqnannotate{\frac{1}{\rho(\vec{x})} \delta_{ij}}{J^{\lambda\lambda}_{f0}} \basisscalar[\lambda]_j \, d\Gamma
\end{gather}



% End of file