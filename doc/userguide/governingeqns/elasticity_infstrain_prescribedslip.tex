% ----------------------------------------------------------------------
\section{Elasticity With Infinitesimal Strain and Faults With Prescribed Slip}

For each fault, which is an internal interface, we add a boundary
condition to the elasticity equation prescribing the jump in the
displacement field across the fault,
\begin{gather}
  \rho \frac{\partial^2\vec{u}}{\partial t^2} - \vec{f}(\vec{x},t) - \tensor{\nabla} \cdot 
\tensor{\sigma}
(\vec{u}) = \vec{0} \text{ in }\Omega, \\
%
  \tensor{\sigma} \cdot \vec{n} = \vec{\tau}(\vec{x},t) \text{ on }\Gamma_\tau, \\
%
  \vec{u} = \vec{u}_0(\vec{x},t) \text{ on }\Gamma_u, \\
%
  \label{eqn:bc:prescribed_slip}
  \vec{0} = \vec{d}(\vec{x},t) - \vec{u}^+(\vec{x},t) + \vec{u}^-(\vec{x},t) \text{ on }\Gamma_f,
\end{gather}
where $\vec{u}^+$ is the displacement vector on the ``positive'' side
of the fault, $\vec{u}^-$ is the displacement vector on the ``negative''
side of the fault, $\vec{d}$ is the slip vector on the fault, and
$\vec{n}$ is the fault normal which points from the negative side of
the fault to the positive side of the fault. Using a domain
decomposition approach for constraining the fault slip, yields
additional Neumann-like boundary conditions on the fault surface,
\begin{gather}
  \tensor{\sigma} \cdot \vec{n} = -\vec{\lambda}(\vec{x},t) \text{ on }\Gamma_{f^+}, \\
  \tensor{\sigma} \cdot \vec{n} = +\vec{\lambda}(\vec{x},t) \text{ on }\Gamma_{f^-},
\end{gather}
where $\vec{\lambda}$ is the vector of Lagrange multipliers
corresponding to the tractions applied to the fault surface to
generate the prescribed slip.

\subsection{Notation}
\begin{itemize}
\item Unknowns
  \begin{description}
  \item[$\vec{u}$] Displacement field
  \item[$\vec{v}$] Velocity field (if including inertial term)
  \item[$\vec{\lambda}$] Lagrange multiplier field
  \end{description}
\item Derived quantities
  \begin{description}
    \item[$\tensor{\sigma}$] Cauchy stress tensor
    \item[$\tensor{\epsilon}$] Cauchy strain tensor
  \end{description}
\item Constitutive parameters
  \begin{description}
  \item[$\mu$] Shear modulus
  \item[$K$] Bulk modulus
  \item[$\rho$] Density
  \end{description}
\item Source terms
  \begin{description}
    \item[$\vec{f}$] Body force per unit volume, for example $\rho \vec{g}$
    \item[$\vec{d}$] Slip vector field on the fault corresponding to a
      jump in the displacement field across the fault
  \end{description}
\end{itemize}

\subsection{Neglecting Inertia}

If we neglect the inertial term, then time dependence only arises
from history-dependent constitutive equations and boundary conditions. Considering the
displacement $\vec{u}$ and Lagrange multiplier $\vec{\lambda}$ fields as the unknowns, we have
\begin{align}
  \vec{s}^T &= (\vec{u} \quad \vec{\lambda})^T, \\
%
  \vec{0} &= \vec{f}(\vec{x},t) + \tensor{\nabla} \cdot \tensor{\sigma}(\vec{u}) \text{ in }
\Omega, \\
% Neumann
  \tensor{\sigma} \cdot \vec{n} &= \vec{\tau}(\vec{x},t) \text{ on }\Gamma_\tau, \\
% Dirichlet
  \vec{u} &= \vec{u}_0(\vec{x},t) \text{ on }\Gamma_u, \\
% Fault
  \vec{0} &= \vec{d}(\vec{x},t) - \vec{u}^+(\vec{x},t) + \vec{u}^-(\vec{x},t) \text{ on }\Gamma_f, \\
  \tensor{\sigma} \cdot \vec{n} &= -\vec{\lambda}(\vec{x},t) \text{ on }\Gamma_{f^+}, \\
  \tensor{\sigma} \cdot \vec{n} &= +\vec{\lambda}(\vec{x},t) \text{ on }\Gamma_{f^-}.
\end{align}

We create the weak form by taking the dot product with the trial
function $\trialvec[u]$ or $\trialvec[\lambda]$ and integrating over the domain:
\begin{align}
  0 &= \int_\Omega \trialvec[u] \cdot \left( \vec{f}(t) + \tensor{\nabla} \cdot \tensor{\sigma} (\vec{u}) \right) \, d\Omega, \\
  0 &= \int_{\Gamma_f} \trialvec[\lambda] \cdot \left( \vec{d}(\vec{x},t) - \vec{u}^+(\vec{x},t) + \vec{u}^-(\vec{x},t) \right) \, d\Gamma.
\end{align}
Using the divergence theorem and incorporating the Neumann boundary and fault interface
conditions, we can rewrite the first equation as
\begin{equation}
% u+
  0 = \int_\Omega \trialvec[u] \cdot \vec{f}(t) + \nabla
  \trialvec[u] : -\tensor{\sigma}
  (\vec{u}) \, d\Omega
  + \int_{\Gamma_\tau} \trialvec[u] \cdot \vec{\tau}(\vec{x},t) \, d\Gamma
  + \int_{\Gamma_{f}} \trialvec[u^+] \cdot -\vec{\lambda}(\vec{x},t) + \trialvec[u^-] \cdot +\vec{\lambda}(\vec{x},t)\, d\Gamma, \\
\end{equation}
In practice we integrate over the fault surface by integrating over
the faces of the cohesive cells. 

Identifying $F(t,s,\dot{s})$ and $G(t,s)$, we have
\begin{align}
  F^u(t,s,\dot{s}) &= 0, \\
  F^\lambda(t,s,\dot{s}) &= 0, \\
  G^{u}(t,s) &=  
   \int_\Omega \trialvec[u] \cdot \eqnannotate{\vec{f}(\vec{x},t)}{g_0^u}
  + \nabla \trialvec[u] : \eqnannotate{-\tensor{\sigma}(\vec{u})}{g_1^u} \, d\Omega 
  + \int_{\Gamma_\tau} \trialvec[u] \cdot \eqnannotate{\vec{\tau}(\vec{x},t)}{g_0^u} \, d\Gamma 
  + \int_{\Gamma_{f}} \trialvec[u^+] \cdot \eqnannotate{-\vec{\lambda}(\vec{x},t)}{g_0^{u^+}} + \trialvec[u^-] \cdot \eqnannotate{+\vec{\lambda}(\vec{x},t)}{g_0^{u^-}}\, d\Gamma,\\
  G^\lambda(t,s) &= 
\int_{\Gamma_{f}} \trialvec[\lambda] \cdot \eqnannotate{\left(
    \vec{d}(\vec{x},t) - \vec{u^+}(\vec{x},t) + \vec{u^-}(\vec{x},t)\right)}{g_0^\lambda} \, d\Gamma.
\end{align}
Note that we have multiple $g_0^u$ functions, each associated with an
integral over a different domain or boundary. The integral over the
domain $\Omega$ is implemented by the material, the integral over the
boundary $\Gamma_\tau$ is implemented by the Neumann boundary
condition, and the integrals over the interface $\Gamma_{f}$ is
implemented by the fault (cohesive cells).

\subsubsection{Jacobians}

With the solution composed of the displacement and Lagrange multiplier fields, the Jacobians are:
\begin{align}
% J_F
  J_F^{uu} &= \tensor{0} \\
  J_F^{\lambda \lambda} &= \tensor{0} \\
% J_G uu
  J_G^{uu} &= \frac{\partial G^u}{\partial u} = \int_\Omega \nabla \trialvec[u] : 
\frac{\partial}{\partial u}(-
\tensor{\sigma}) \, d\Omega 
  = \int_\Omega \nabla \trialvec[u] : -\tensor{C} : \frac{1}{2}(\nabla + \nabla^T)\basisvec[u] 
\, d\Omega 
  = \int_\Omega \trialscalar[v]_{i,k} \, \eqnannotate{\left( -C_{ikjl} \right)}{J_{g3}^{uu}} \, \basisscalar[u]_{j,l}\, d\Omega \\
  % J_G u \lambda
  \begin{split}
J_G^{u\lambda} &= \frac{\partial G^u}{\partial \lambda} =
\int_{\Gamma_{f^+}} \trialvec[u] \cdot \frac{\partial}{\partial \lambda}(-\vec{\lambda}) \, d\Gamma
+ \int_{\Gamma_{f^-}} \trialvec[u] \cdot \frac{\partial}{\partial \lambda}(+\vec{\lambda}) \, d\Gamma \\
& \quad\quad = \int_{\Gamma_{f}} \trialscalar[u^+]_i \eqnannotate{-1}{J_{g0}^{u^+\lambda}}\basisscalar[\lambda]_j + \trialscalar[u^-]_i \eqnannotate{+1}{J_{g0}^{u^-\lambda}}\basisscalar[\lambda]_j \, d\Gamma
\end{split} \\
% J_G \lambda u
\begin{split}
J_G^{\lambda u} &= \frac{\partial G^\lambda}{\partial u} =
\int_{\Gamma_{f}} \trialvec[\lambda] \cdot \frac{\partial}{\partial u}\left(\vec{d}(\vec{x},t) - \vec{u^+}(\vec{x},t) + \vec{u^-}(\vec{x},t) \right) \, d\Gamma \\
&\quad\quad\quad = \int_{\Gamma_{f}} \trialscalar[\lambda]_i (\eqnannotate{-1}{J_{g0}^{\lambda u^+}})\basisscalar[u^+]_j
+ \trialscalar[\lambda]_i (\eqnannotate{+1}{J_{g0}^{\lambda u^-}})\basisscalar[u^-]_j \, d\Gamma
\end{split} \\
%
  J_G^{\lambda \lambda} &= 0
\end{align}

\subsection{Including Inertia}

For convenience we cast the elasticity equation in the form of a first order
equation by considering the displacement $\vec{u}$, velocity $\vec{v}$, and Lagrange multipliers $\vec{\lambda}$
as unknowns,
\begin{align}
  \vec{s}^T &= (\vec{u} \quad \vec{v} \quad \vec{\lambda})^T, \\
%
  \frac{\partial\vec{u}}{\partial t} &= \vec{v}, \\
%
  \rho \frac{\partial\vec{v}}{\partial t} &= \vec{f}(\vec{x},t) + \tensor{\nabla} \cdot 
\tensor{\sigma}(\vec{u}) 
\text{ in }\Omega, \\
% Neumann
  \tensor{\sigma} \cdot \vec{n} &= \vec{\tau}(\vec{x},t) \text{ on }\Gamma_\tau, \\
% Dirichlet
  \vec{u} &= \vec{u}_0(\vec{x},t) \text{ on }\Gamma_u, \\
% Fault
  \vec{0} &= \vec{d}(\vec{x},t) - \vec{u}^+(\vec{x},t) + \vec{u}^-(\vec{x},t) \text{ on }\Gamma_f, \\
  \tensor{\sigma} \cdot \vec{n} &= -\vec{\lambda}(\vec{x},t) \text{ on }\Gamma_{f^+}, \\
  \tensor{\sigma} \cdot \vec{n} &= +\vec{\lambda}(\vec{x},t) \text{ on }\Gamma_{f^-}.
\end{align}

For trial functions $\trialvec[u]$, $\trialvec[v]$, and $\trialvec[\lambda]$ we write the weak form as
\begin{align}
  \int_\Omega \trialvec[u] \cdot \left( \frac{\partial \vec{u}}{\partial t} \right) \, d\Omega 
    &= \int_\Omega \trialvec[u] \cdot \vec{v} \, d\Omega, \\
%
  \int_\Omega \trialvec[v] \cdot \left( \rho \frac{\partial \vec{v}}{\partial t} \right) \, d\Omega &= 
  \int_\Omega \trialvec[v] \cdot \left( \vec{f}(\vec{x},t) + \tensor{\nabla} \cdot 
  \tensor{\sigma}(\vec{u}) \right) \, d\Omega, \\
  %
  0 &= \int_{\Gamma_{f}} \trialvec[\lambda] \cdot \left( \vec{d}(\vec{x},t) - \vec{u^+}(\vec{x},t) + \vec{u^-}(\vec{x},t) \right) \, d\Gamma.
\end{align}
Using the divergence theorem and incorporating the Neumann boundary
and fault interface conditions, we can rewrite the second equation as
\begin{multline}
  \int_\Omega \trialvec[v] \cdot \left( \rho \frac{\partial \vec{v}}{\partial t} \right) \, 
d\Omega =
  \int_\Omega \trialvec[v] \cdot \vec{f}(\vec{x},t) + \nabla \trialvec[v] : -\tensor{\sigma}
  (\vec{u}) \, d\Omega
  + \int_{\Gamma_\tau} \trialvec[v] \cdot \vec{\tau}(\vec{x},t) \, d\Gamma \\
+ \int_{\Gamma_{f^+}} \trialvec[v^+] \cdot -\vec{\lambda}(\vec{x},t) + \trialvec[v^-] \cdot +\vec{\lambda}(\vec{x},t)\, d\Gamma.
\end{multline}

% ----------------------------------------------------------------------
\subsection{Explicit Time Stepping}

Recall that for explicit time stepping we want
$F(t,s,\dot{s})=\dot{s}$. However, our fault interface constraint
equation cannot be put into this form. Nevertheless, we put the first
two equations in this form. The resulting equation will be a
differential algebraic equation (DAE). Our system of equations to
solve is
\begin{align}
  \int_\Omega \trialvec[u] \cdot \frac{\partial \vec{u}}{\partial t} \, d\Omega &= 
  \int_\Omega \trialvec[u] \cdot \vec{v} \, d\Omega, \\
  %
  \begin{split}
  \int_\Omega \trialvec[v] \cdot \frac{\partial \vec{v}}{\partial t} \, d\Omega &=
  \frac{1}{\int_\Omega \trialvec[v] \cdot \rho\,\basisvec[v] \, d\Omega} \left( \int_\Omega 
  \trialvec[v] \cdot \vec{f}(\vec{x},t) + \nabla \trialvec[u] : -\tensor{\sigma}(\vec{u}) \, d\Omega
  + \int_{\Gamma_\tau} \trialvec[v] \cdot \vec{\tau}(\vec{x},t) \, d\Gamma \right. \\
  & \quad \left. + \int_{\Gamma_{f}} \trialvec[v^+] \cdot -\vec{\lambda}(\vec{x},t) + \trialvec[v^-] \cdot +\vec{\lambda}(\vec{x},t) \, d\Gamma \right).
  \end{split} \\
%
  0 &= \int_{\Gamma_{f^+}} \trialvec[\lambda] \cdot \left( \vec{d}(\vec{x},t) - \vec{u^+}(\vec{x},t) + \vec{u^-}(\vec{x},t) \right) \, d\Gamma.
\end{align}
Identifying $F(t,s,\dot{s})$ and $G(t,s)$, we have
\begin{align}
%% Fu
  F^u(t,s,\dot{s}) &= \int_\Omega \trialvec[u] \cdot \eqnannotate{\frac{\partial \vec{u}}{\partial t}}{f_o^u} \, d\Omega, \\
% Fv
  F^v(t,s,\dot{s}) &= \int_\Omega \trialvec[v] \cdot \eqnannotate{\frac{\partial \vec{v}}{\partial t}}{f_0^v}  \, d\Omega, \\
  % F\lambda
  F^\lambda(t,s,\dot{s}) &= 0, \\
% Gu
  G^u(t,s) &= \int_\Omega \trialvec[u] \cdot \eqnannotate{\vec{v}}{g_0^u} \, d\Omega, \\
  % Gv
  \begin{split}
  G^v(t,s) &= \frac{1}{\int_\Omega \trialvec[v] \cdot {\eqnannotate{\rho}{J_{f0}^{*vv}}}
    \basisvec[v] \, d\Omega} 
  \left( \int_\Omega \trialvec[v] \cdot \eqnannotate{\vec{f}(t)}{g_0^v} + \nabla \trialvec[v] : 
  \eqnannotate{-\tensor{\sigma}(\vec{u})}{g_1^v} \, d\Omega
  + \int_{\Gamma_\tau} \trialvec[v] \cdot \eqnannotate{\vec{\tau} (\vec{x},t)}{g_0^v} \, d\Gamma \right. \\
&\quad\left. + \int_{\Gamma_{f}} \trialvec[v^+] \cdot \eqnannotate{-\vec{\lambda}(\vec{x},t)}{g_0^v}
  + \trialvec[v^-] \cdot \eqnannotate{+\vec{\lambda}(\vec{x},t)}{g_0^v} \, d\Gamma \right),
  \end{split} \\
  % G\lambda
  G^\lambda(t,s) &= 
\int_{\Gamma_{f}} \trialvec[\lambda] \cdot \eqnannotate{\left( \vec{d}(\vec{x},t) - \vec{u^+}(\vec{x},t) + \vec{u^-}(\vec{x},t) \right)}{g_0^\lambda} \, d\Gamma.
\end{align}
Note that we have multiple $g_0^u$ functions, each associated with an
integral over a different domain or boundary. The integral over the
domain $\Omega$ is implemented by the material, the integral over the
boundary $\Gamma_\tau$ is implemented by the Neumann boundary
condition, and the integral over the interface $\Gamma_{f}$ is
implemented by the fault (cohesive cells).


\subsubsection{Jacobians}

With a differential algebraic equation, we only need to specify the Jacobians for the LHS.
\begin{align}
% J_F
  J_F^{uu} &= \frac{\partial F^u}{\partial u} + s_\mathit{tshift} \frac{\partial F^u}{\partial \dot{u}}
  = \int_\Omega \trialscalar[u]_i \eqnannotate{s_\mathit{tshift}}{J_{f0}^{uu}} \basisscalar[u]_j \, d\Omega, \\
%
  J_F^{vv} &= \frac{\partial F^v}{\partial v}  + s_\mathit{tshift} \frac{\partial F^v}{\partial \dot{v}}
  = \int_\Omega \trialscalar[v]_i \eqnannotate{s_\mathit{tshift}}{J_{f0}^{vv}} \basisscalar[v]_j \, d\Omega, \\
%
  J_F^{\lambda \lambda} &= \frac{\partial F^\lambda}{\partial \lambda} + s_\mathit{tshift} \frac{\partial F^\lambda}{\partial \dot{\lambda}}
  = \eqnannotate{\tensor{0}}{J_{f0}^{\lambda \lambda}}
\end{align}

