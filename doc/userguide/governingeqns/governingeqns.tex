\chapter{Governing Equations}
\label{cha:governing:equations}

We present here a brief derivation of the equations for both quasi-static
and dynamic computations. Since the general equations are the same
(except for the absence of inertial terms in the quasi-static case),
we first derive these equations. We then present solution methods
for each specific case. In all of our derivations, we use the notation
described in Table \vref{tab:notation} for both index
and vector notation.

\begin{table}[htbp]
  \caption{Mathematical notation}
  \label{tab:notation}
  \begin{tabular}{ccp{3in}}
    \multicolumn{2}{c}{{\bf Symbol}} & {\bf Description} \\
    {\bf Index notation} & {\bf Vector Notation} & \\
    \hline 
    $a_{i}$ & \raisebox{12pt}{}$\overrightarrow{a}$ & Vector field a \\
    $a_{ij}$ & $\underline{a}$ & Second order tensor field a \\
    $u_{i}$ & $\overrightarrow{u}$ & Displacement vector field \\
    $d_{i}$ & $\vec{{d}}$ & Fault slip vector field \\
    $f_{i}$ & $\overrightarrow{f}$ & Body force vector field \\
    $T_{i}$ & $\overrightarrow{\tau}$ & Traction vector field \\
    $\sigma_{ij}$ & $\underline{\sigma}$ & Stress tensor field \\
    $n_{i}$ & $\overrightarrow{n}$ & Normal vector field \\
    $\rho$ & $\rho$ & Mass density scalar field \\
    \hline 
  \end{tabular}
\end{table}

\section{Derivation of Elasticity Equation}

For completeness we start our discussing of the governing equations
with a derivation of the elasticity equation. Consider domain $\Omega$
bounded by boundary $\Gamma$. Applying a Lagrangian description of the
conservation of momentum gives
\begin{equation}
\label{eqn:momentum:vec}
\frac{\partial}{\partial t}\int_{\Omega}\rho(\vec{x})\frac{\partial\vec{u}}{\partial t}\, d\Omega=\int_{\Omega}\vec{f}(\vec{x},t)\, d\ + \int_{\Gamma}\vec{\tau}(\vec{x},t)\, d\Gamma.
\end{equation}
The traction vector field is related to the stress tensor through
\begin{equation}
\vec{\tau}(\vec{x},t) = \tensor{\sigma}(\vec{u}) \cdot \vec{n},
\end{equation}
where $\vec{n}$ is the outward normal vector to $\Gamma$. Substituting
into equation \vref{eqn:momentum:vec} yields
\begin{equation}
\frac{\partial}{\partial t}\int_{\Omega}\rho(\vec{x})\frac{\partial\vec{u}}{\partial t}\, d\Omega=\int_{\Omega}\vec{f}(\vec{x},t)\, d\Omega+\int_{\Gamma}\tensor{\sigma}(\vec{u})\cdot\vec{n}\, d\Gamma.
\end{equation}
Applying the divergence theorem,
\begin{equation}
\int_{\Omega}\tensor{\nabla}\cdot\vec{a}\: d\Omega=\int_{\Gamma}\vec{a}\cdot\vec{n}\: d\Gamma,
\end{equation}
to the boundary integral results in
\begin{equation}
\frac{\partial}{\partial t}\int_{\Omega}\rho(\vec{x})\frac{\partial\vec{u}}{\partial t}\, d\Omega=\int_{\Omega}\vec{f}(\vec{x},t)\, d\Omega+\int_{\Omega}\tensor{\nabla}\cdot\tensor{\sigma}(\vec{u})\, d\Omega,
\end{equation}
which we can rewrite as
\begin{equation}
\int_{\Omega}\left(\rho(\vec{x})\frac{\partial^{2}\vec{u}}{\partial t^{2}}-\vec{f}(\vec{x},t)-\tensor{\nabla}\cdot\tensor{\sigma}(\vec{u})\right)\, d\Omega=\vec{0}.
\end{equation}
Because the domain $\Omega$ is arbitrary, the integrand must be the zero
vector at every location in the domain, so that we end up with
\begin{gather}
\rho(\vec{x})\frac{\partial^{2}\vec{u}}{\partial t^{2}}-\vec{f}(\vec{x},t)-\tensor{\nabla}\cdot\tensor{\sigma}=\vec{0}\text{ in }\Omega,\\
\tensor{\sigma}(\vec{u})\cdot\vec{n}=\vec{\tau}(\vec{x},t)\text{ on }\Gamma_{\tau}\text{,}\\
\vec{u}=\vec{u}_0(\vec{x},t)\text{ on }\Gamma_{u},\text{ and}\\
\vec{u}^{+}-\vec{u}^{-}=\vec{d}\text{ on }\Gamma_{f}.
\end{gather}
We specify tractions, $\vec{\tau}$, on boundary $\Gamma_{f}$, displacements,
$\vec{u^{o}}$, on boundary $\Gamma_{u}$, and slip, $\vec{d}$,
on fault interface $\Gamma_{f}$.
%(we will consider the case of fault constitutive models in Section \vref{sec:fault}).

\section{Multiphysics Finite-Element Formulation}
\label{sec:multiphysics:formulation}

Within the PETSc solver framework, we want to solve a system of
partial differential equations in which the weak form can be
expressed as $F(t,s,\dot{s}) = G(t,s)$, $s(t_0) = s_0$, where $F$ and
$G$ are vector functions, $t$ is time, and $s$ is the solution vector.

Using the finite-element method we manipulate the weak form of the
system of equations involving a vector field $\vec{u}$ into integrals
over the domain $\Omega$ matching the form,
\begin{equation}
  \label{eqn:problem:form}
  \int_\Omega \trialvec[u] \cdot \vec{f}_0(t,s,\dot{s}) + \nabla \trialvec[u] : \tensor{f}
_1(t,s,\dot{s}) \, 
d\Omega =
  \int_\Omega \trialvec[u] \cdot \vec{g}_0(t,s) + \nabla \trialvec[u] : \tensor{g}_1(t,s) \, 
d\Omega,
\end{equation}
where $\trialvec[u]$ is the trial function for field $\vec{u}$,
$\vec{f}_0$ and $\vec{g}_0$ are vectors, and $\tensor{f}_1$ and
$\tensor{g}_1$ are tensors. With multiple partial differential
equations we will have multiple equations of this form, and the
solution vector $s$, which we usually write as $\vec{s}$, will be
composed of several different fields, such as displacement $\vec{u}$,
velocity $\vec{v}$, pressure $p$, and temperature $T$. Boundary
conditions will also contribute similar terms with integrals over the
corresponding boundaries.

For consistency with the PETSc time stepping formulation, we call
$G(t,s)$ the RHS function and call $F(t,s,\dot{s})$ the LHS (or I)
function. Likewise, the Jacobian of $G(t,s)$ is the RHS Jacobian and
the Jacobian of $F(t,s,\dot{s})$ is the LHS Jacobian. In most cases,
we can take $F(t,s,\dot{s}) = \dot{s}$, or as close to this as
possible. This results in miminal changes to the formulation in order
to accommodate both implicit and explicit time stepping algorithms.

Using a finite-element discretization we break up the domain and
boundary integrals into sums over the cells and boundary faces/edges,
respectively. Using numerical quadrature those sums in turn involve
sums over the values at the quadrature points with appropriate
weights. Thus, computation of the RHS function boils down to
pointwise evaluation of $\vec{g}_0(t,s)$ and $\tensor{g}_1(t,s)$, and
computation of the LHS function boils down to pointwise evaluation of
$\vec{f}_0(t,s,\dot{s})$ and $\tensor{f}_1(t,s,\dot{s})$.

\subsection{Jacobian}

The LHS Jacobian $J_F = \frac{\partial F}{\partial s} +
s_\mathit{tshift} \frac{\partial F}{\partial \dot{s}}$ and the RHS
Jacobian $J_G = \frac{\partial G}{\partial s}$, where
$s_\mathit{tshift}$ arises from the temporal discretization. We put
the Jacobians for each equation into the form:
\begin{align}
  \label{eqn:jacobian:form}
  J_F &= \int_\Omega \trialvec \cdot \tensor{J}_{f0}(t,s,\dot{s}) \cdot \basisvec
  + \trialvec \cdot \tensor{J}_{f1}(t,s,\dot{s}) : \nabla \basisvec
  + \nabla \trialvec : \tensor{J}_{f2}(t,s,\dot{s}) \cdot \basisvec
  + \nabla \trialvec : \tensor{J}_{f3}(t,s,\dot{s}) : \nabla \basisvec \, d\Omega \\
%
  J_G &= \quad \int_\Omega \trialvec \cdot \tensor{J}_{g0}(t,s) \cdot \basisvec
  + \trialvec \cdot \tensor{J}_{g1}(t,s) : \nabla \basisvec
  + \nabla \trialvec : \tensor{J}_{g2}(t,s) \cdot \basisvec
  + \nabla \trialvec : \tensor{J}_{g3}(t,s) : \nabla \basisvec \, d\Omega,
\end{align}
where $\basisvec$ is a basis function.  Expressed in index notation
the Jacobian coupling solution field components $s_i$ and $s_j$ is
\begin{equation}
\label{eqn:jacobian:index:form}
J^{s_is_j} = \int_\Omega \trialscalar_i J_0^{s_is_j} \basisscalar_j + \trialscalar_i 
J_1^{s_js_jl} 
\basisscalar_{j,l} + \trialscalar_{i,k} J_2^{s_is_jk} \basisscalar_j + \trialscalar_{i,k} 
J_3^{s_is_jkl} 
\basisscalar_{j,l} \, d\Omega, 
\end{equation}
It is clear that the tensors $J_0$, $J_1$, $J_2$, and $J_3$ have
various sizes: $J_0(n_i,n_j)$, $J_1(n_i,n_j,d)$, $J_2(n_i,n_j,d)$,
$J_3(n_i,n_j,d,d)$, where $n_i$ is the number of components in
solution field $s_i$, $n_j$ is the number of components in solution
field $s_j$, and $d$ is the spatial dimension.  Alternatively,
expressed in discrete form, the Jacobian for the coupling between
solution fields $s_i$ and $s_j$ is
\begin{equation}
  \label{eqn:jacobian:discrete:form}
  J^{s_is_j} = J_{0}^{s_is_j} + J_{1}^{s_is_j} B + B^T J_{2}^{s_is_j} + B^T J_{3}^{s_is_j} B,
\end{equation}
where $B$ is a matrix of the derivatives of the basis functions and $B^T$
is a matrix of the derivatives of the trial functions. 

\important{See
  \url{https://www.mcs.anl.gov/petsc/petsc-master/docs/manualpages/FE/PetscFEIntegrateJacobian.html}
  for the ordering of indicies in the Jacobian pointwise functions.}

\subsection{PETSc TS Notes}

\begin{itemize}
\item If no LHS (or I) function is given, then the PETSc TS assumes $F(t,s,\dot{s}) = \dot{s}
$.
\item If no RHS function is given, then the PETSc TS assumes $G(t,s) = 0$.
\item Explicit time stepping with the PETSc TS requires
  $F(t,s,\dot{s}) = \dot{s}$.
  \begin{itemize}
  \item Because $F(t,s,\dot{s}) = \dot{s}$, we do not specify the
    functions $\vec{f}_0(t,s,\dot{s})$ and $\tensor{f}_1(t,s,\dot{s})$
    because the PETSc TS will assume this is the case if no LHS (or I)
    function is given.
  \item We also do not specify $J_F$ or $J_G$.
  \item This leaves us with only needing to specify $\vec{g}_0(t,s)$
    and $\tensor{g}_1(t,s)$. 
  \item The PETSc TS will verify that the LHS (or I) function is null.
  \end{itemize}
\end{itemize}

For explicit time stepping with the PETSc TS, we need
$F(t,s,\dot{s}) = \dot{s}$. Using a finite-element formulation for
elastodynamics, $F(t,s,\dot{s})$ generally involves integrals of the
inertia over the domain. It is tempting to simply move these terms to
the RHS, but that introduces inertial terms into the boundary
conditions, which makes them less intuitive. Instead, we transform our
equation into the form $\dot{s} = G^*(t,s)$ where writing
$G^*(t,s) = M^{-1} G(t,s)$. We take $M$ to be a lumped (diagonal)
matrix, so that $M^{-1}$ is trivial to compute. In computing the RHS
function, $G^*(t,s)$, we compute $G(t,s)$, then compute $M$ and
$M^{-1}$, and then $M^{-1}G(t,s)$. For the elasticity equation with
inertial terms, $M$ contains the mass matrix.


\section{Elasticity With Infinitesimal Strain and No Faults}

We begin with the elasticity equation including the inertial term,
\begin{gather}
  \label{eqn:elasticity:strong:form}
  \rho \frac{\partial^2\vec{u}}{\partial t^2} - \vec{f}(\vec{x},t) - \tensor{\nabla} \cdot 
\tensor{\sigma}
(\vec{u}) = \vec{0} \text{ in }\Omega, \\
%
  \label{eqn:bc:Neumann}
  \tensor{\sigma} \cdot \vec{n} = \vec{\tau}(\vec{x},t) \text{ on }\Gamma_\tau, \\
%
  \label{eqn:bc:Dirichlet}
  \vec{u} = \vec{u}_0(\vec{x},t) \text{ on }\Gamma_u,
\end{gather}
where $\vec{u}$ is the displacement vector, $\rho$ is the mass
density, $\vec{f}$ is the body force vector, $\tensor{\sigma}$ is the
Cauchy stress tensor, $\vec{x}$ is the spatial coordinate, and $t$ is
time. We specify tractions $\vec{\tau}$ on boundary $\Gamma_\tau$, and
displacements $\vec{u}_0$ on boundary $\Gamma_u$. Because both $\vec{\tau}$
and $\vec{u}$ are vector quantities, there can be some spatial overlap
of boundaries $\Gamma_\tau$ and $\Gamma_u$; however, a degree of freedom at
any location cannot be associated with both prescribed displacements
(Dirichlet) and traction (Neumann) boundary conditions simultaneously.

\subsection{Notation}
\begin{itemize}
\item Unknowns
  \begin{description}
  \item[$\vec{u}$] Displacement field
  \item[$\vec{v}$] Velocity field (if including inertial term)
  \end{description}
\item Derived quantities
  \begin{description}
    \item[$\tensor{\sigma}$] Stress tensor
    \item[$\tensor{\epsilon}$] Strain tensor
  \end{description}
\item Constitutive parameters
  \begin{description}
  \item[$\mu$] Shear modulus
  \item[$K$] Bulk modulus
  \item[$\rho$] Density
  \end{description}
\item Source terms
  \begin{description}
    \item[$\vec{f}$] Body force per unit volume, for example $\rho \vec{g}$
  \end{description}
\end{itemize}

\subsection{Neglecting Inertia}

If we neglect the inertial term, then time dependence only arises
from history-dependent constitutive equations and boundary
conditions. Considering the displacement $\vec{u}$ as the unknown, we
have
\begin{align}
  \vec{s}^T &= (\vec{u})^T, \\
%
  \vec{0} &= \vec{f}(\vec{x},t) + \tensor{\nabla} \cdot \tensor{\sigma}(\vec{u}) \text{ in }
\Omega, \\
% Neumann
  \tensor{\sigma} \cdot \vec{n} &= \vec{\tau}(\vec{x},t) \text{ on }\Gamma_\tau, \\
% Dirichlet
  \vec{u} &= \vec{u}_0(\vec{x},t) \text{ on }\Gamma_u.
\end{align}
We create the weak form by taking the dot product with the trial
function $\trialvec[u]$ and integrating over the domain:
\begin{equation}
  0 = \int_\Omega \trialvec[u] \cdot \left( \vec{f}(t) + \tensor{\nabla} \cdot \tensor{\sigma}
(\vec{u})  \right) 
\, d\Omega.
\end{equation}
Using the divergence theorem and incorporating the Neumann bounday
condition yields
\begin{equation}
  0 = \int_\Omega \trialvec[u] \cdot \vec{f}(t) + \nabla \trialvec[u] : -\tensor{\sigma}
(\vec{u}) \, d\Omega + 
\int_{\Gamma_\tau} \trialvec[u] \cdot \vec{\tau}(\vec{x},t) \, d\Gamma.
\end{equation}

Identifying $F(t,s,\dot{s})$ and $G(t,s)$, we have
\begin{alignat}{2}
  F^u(t,s,\dot{s}) &= \vec{0},
  & \qquad
  G^u(t,s) &= \int_\Omega \trialvec[u] \cdot \eqnannotate{\vec{f}(\vec{x},t)}{g_0^u} + \nabla 
\trialvec[u] : 
\eqnannotate{-\tensor{\sigma}(\vec{u})}{g_1^u} \, d\Omega + \int_{\Gamma_\tau} \trialvec[u] 
\cdot 
\eqnannotate{\vec{\tau}(\vec{x},t)}{g_0^u} \, d\Gamma.
\end{alignat}


\subsubsection{Jacobians}

With the solution composed of the displacement field and no LHS function, we only have 
Jacobians for the RHS,
\begin{align}
  J_G^{uu} &= \frac{\partial G^u}{\partial u} = \int_\Omega \nabla \trialvec[u] : 
\frac{\partial}{\partial u}(-
\tensor{\sigma}) \, d\Omega 
  = \int_\Omega \nabla \trialvec[u] : -\tensor{C} : \frac{1}{2}(\nabla + \nabla^T)\basisvec[u] 
\, d\Omega 
  = \int_\Omega \trialscalar[v]_{i,k} \, \eqnannotate{\left( -C_{ikjl} \right)}{J_{g3}^{uu}}  
\, 
\basisscalar[u]_{j,l}\, d\Omega
\end{align}

\subsection{Including Inertia}

For convenience we cast the elasticity equation in the form of a first order
equation by considering both the displacement $\vec{u}$ and velocity $\vec{v}$
as unknowns,
\begin{align}
  \vec{s}^T &= (\vec{u} \quad \vec{v})^T, \\
%
  \label{eqn:velocity:strong:form}
  \frac{\partial\vec{u}}{\partial t} &= \vec{v}, \\
%
  \label{eqn:elasticity:order1:strong:form}
  \rho \frac{\partial\vec{v}}{\partial t} &= \vec{f}(\vec{x},t) + \tensor{\nabla} \cdot 
\tensor{\sigma}(\vec{u}) 
\text{ in }\Omega, \\
% Neumann
  \tensor{\sigma} \cdot \vec{n} &= \vec{\tau}(\vec{x},t) \text{ on }\Gamma_\tau, \\
% Dirichlet
  \vec{u} &= \vec{u}_0(\vec{x},t) \text{ on }\Gamma_u.
\end{align}

For trial functions $\trialvec[u]$ and $\trialvec[v]$ we write the weak form as
\begin{align}
  \int_\Omega \trialvec[u] \cdot \left( \frac{\partial \vec{u}}{\partial t} \right) \, d\Omega 
&= 
  \int_\Omega \trialvec[u] \cdot \vec{v} \, d\Omega, \\
%
  \int_\Omega \trialvec[v] \cdot \left( \rho \frac{\partial \vec{v}}{\partial t} \right) \, 
d\Omega &= 
  \int_\Omega \trialvec[v] \cdot \left( \vec{f}(\vec{x},t) + \tensor{\nabla} \cdot 
\tensor{\sigma}(\vec{u})  
\right) \, d\Omega.
%
\end{align}
Using the divergence theorem and incorporating the Neumann boundary
conditions, we can rewrite the second equation as
\begin{equation}
  \label{eqn:elasticity:displacement}
  \int_\Omega \trialvec[v] \cdot \left( \rho \frac{\partial \vec{v}}{\partial t} \right) \, 
d\Omega =
  \int_\Omega \trialvec[v] \cdot \vec{f}(\vec{x},t) + \nabla \trialvec[v] : -\tensor{\sigma}
(\vec{u}) \, d\Omega + 
\int_{\Gamma_\tau} \trialvec[v] \cdot \vec{\tau}(\vec{x},t) \, d\Gamma.
\end{equation}

% ----------------------------------------------------------------------
\subsubsection{Implicit Time Stepping}
In practice we do not use implicit time stepping when we include
inertia. We provide this section to illustrate the derivation of the
point-wise functions for the residual and Jacobian. The resulting
system of equations to solve is
\begin{align}
  \label{eqn:elasticity:velocity:implicit}
  \int_\Omega \trialvec[u] \cdot \left( \frac{\partial \vec{u}}{\partial t} \right) \, d\Omega 
&= 
  \int_\Omega \trialvec[u] \cdot \vec{v} \, d\Omega, \\
%
  \label{eqn:elasticity:displacement:implicit}
  \int_\Omega \trialvec[v] \cdot \left( \rho \frac{\partial \vec{v}}{\partial t} \right) \, 
d\Omega &=
  \int_\Omega \trialvec[v] \cdot \vec{f}(\vec{x},t) + \nabla \trialvec[v] : -\tensor{\sigma}
(\vec{u}) \, d\Omega + 
\int_{\Gamma_\tau} \trialvec[u] \cdot \vec{\tau}(\vec{x},t) \, d\Gamma.
\end{align}
Identifying $F(t,s,\dot{s})$ and $G(t,s)$, we have
\begin{alignat}{2}
  F^u(t,s,\dot{s}) &= \int_\Omega \trialvec[u] \cdot \eqnannotate{\left( \frac{\partial 
\vec{u}}{\partial t} 
\right)}{f_0^u} \, d\Omega,
  & \qquad
  G^u(t,s) &= \int_\Omega \trialvec[u] \cdot \eqnannotate{\vec{v}}{g_0^u} \, d\Omega, \\
  %  
  F^v(t,s,\dot{s}) &= \int_\Omega \trialvec[v] \cdot \eqnannotate{\left( \rho \frac{\partial 
\vec{v}}{\partial t} 
\right)}{f_0^v} \, d\Omega,
  & \qquad
  G^v(t,s) &= \int_\Omega \trialvec[v] \cdot \eqnannotate{\vec{f}(\vec{x},t)}{g_0^v} + \nabla 
\trialvec[v] : 
\eqnannotate{-\tensor{\sigma}(\vec{u})}{g_1^v} \, d\Omega + \int_{\Gamma_\tau} \trialvec[u] 
\cdot 
\eqnannotate{\vec{\tau}(\vec{x},t)}{g_0^v} \, d\Gamma.
\end{alignat}


\subsubsection{Jacobians}

With two fields we have four Jacobians for each side of the equation associated with the 
coupling of the two 
fields,
\begin{align}
  J_F^{uu} &= \frac{\partial F^u}{\partial u} + s_\mathit{tshift} \frac{\partial F^u}{\partial 
\dot{u}} = \int_\Omega 
\trialvec[u] \cdot s_\mathit{tshift}\,\basisvec[u] \, d\Omega = \int_\Omega \trialscalar[u]_i 
\, 
\eqnannotate{s_\mathit{tshift} \delta_{ij}}{J_{f0}^{uu}} \, \basisscalar[u]_j \, d\Omega \\
  J_F^{uv} &= \frac{\partial F^u}{\partial v} + s_\mathit{tshift} \frac{\partial F^u}{\partial 
\dot{v}} = \tensor{0} \\
  J_F^{vu} &= \frac{\partial F^v}{\partial u} + s_\mathit{tshift} \frac{\partial F^v}{\partial 
\dot{u}} = \tensor{0} \\
  J_F^{vv} &= \frac{\partial F^v}{\partial v} + s_\mathit{tshift} \frac{\partial F^v}{\partial 
\dot{v}} = \int_\Omega 
\trialvec[v] \cdot s_\mathit{tshift}\,\rho\,\basisvec[v] \, d\Omega = \int_\Omega 
\trialscalar[v]_i \, 
\eqnannotate{s_\mathit{tshift} \, \rho \, \delta_{ij}}{J_{f0}^{vv}} \, \basisscalar[v]_j \, 
d\Omega \\
  J_G^{uu} &= \frac{\partial G^u}{\partial u} = \tensor{0} \\
  J_G^{uv} &= \frac{\partial G^u}{\partial v} = \int_\Omega \trialvec[u] \cdot \basisvec[v] \, 
d\Omega = 
\int_\Omega \trialscalar[u]_i \, \eqnannotate{\delta_{ij}}{J_{g0}^{uv}} \, \basisscalar[v]_j 
\, d\Omega \\
  J_G^{vu} &= \frac{\partial G^v}{\partial u} = \int_\Omega \nabla \trialvec[v] : 
\frac{\partial}{\partial u}(-
\tensor{\sigma}) \, d\Omega 
  = \int_\Omega \nabla \trialvec[v] : -\tensor{C} : \frac{1}{2}(\nabla + \nabla^T)\basisvec[u] 
\, d\Omega 
  = \int_\Omega \trialscalar[v]_{i,k} \, \eqnannotate{\left( -C_{ikjl} \right)}{J_{g3}^{vu}}  
\, 
\basisscalar[u]_{j,l}\, d\Omega \\
  J_G^{vv} &= \frac{\partial G^v}{\partial v} = \tensor{0}
\end{align}

% ----------------------------------------------------------------------
\subsection{Explicit Time Stepping}
Recall that explicit time stepping requires $F(t,s,\dot{s})=\dot{s}$. We write $F^*(t,s,
\dot{s}) = \dot{s}$ and
$G^*(t,s) = J_F^{-1}G(t,s)$ and we do not provide functions for $f_0$ and $f_1$. Thus, our 
system of equations to 
solve is
\begin{align}
  \label{eqn:elasticity:velocity:explicit}
  \int_\Omega \trialvec[u] \cdot \frac{\partial \vec{u}}{\partial t} \, d\Omega &= 
  \int_\Omega \trialvec[u] \cdot \vec{v} \, d\Omega, \\
%
  \label{eqn:elasticity:displacement:explicit}
  \int_\Omega \trialvec[v] \cdot \frac{\partial \vec{v}}{\partial t} \, d\Omega &=
  \frac{1}{\int_\Omega \trialvec[v] \cdot \rho\,\basisvec[v] \, d\Omega} \left( \int_\Omega 
\trialvec[v] \cdot 
\vec{f}(\vec{x},t) + \nabla \trialvec[u] : -\tensor{\sigma}(\vec{u}) \, d\Omega + 
\int_{\Gamma_\tau} \trialvec[u] 
\cdot \vec{\tau}(\vec{x},t) \, d\Gamma \right).
\end{align}
Identifying $F(t,s,\dot{s})$ and $G(t,s)$, we have
\begin{align}
  F^u(t,s,\dot{s}) &= \int_\Omega \trialvec[u] \cdot \frac{\partial \vec{u}}{\partial t} \, 
d\Omega, \\
%
  G^u(t,s) &= \int_\Omega \trialvec[u] \cdot \eqnannotate{\vec{v}}{g_0^u} \, d\Omega, \\
  %  
  F^v(t,s,\dot{s}) &= \int_\Omega \trialvec[v] \cdot \frac{\partial \vec{v}}{\partial t}  \, 
d\Omega, \\
%
  G^v(t,s) &= \frac{1}{\int_\Omega \trialvec[v] \cdot {\eqnannotate{\rho}{J_{f0}^{vv}}}
\basisvec[v] \, d\Omega} 
\left( \int_\Omega \trialvec[v] \cdot \eqnannotate{\vec{f}(t)}{g_0^v} + \nabla \trialvec[v] : 
\eqnannotate{-
\tensor{\sigma}(\vec{u})}{g_1^v} \, d\Omega + \int_{\Gamma_\tau} \trialvec[v] \cdot 
\eqnannotate{\vec{\tau}
(\vec{x},t)}{g_0^v} \, d\Gamma \right).
\end{align}
where $J_{f0}^{uu} = \tensor{I}$, and we refer to $J_F$ as the LHS
(or I) Jacobian for explicit time stepping.

% ----------------------------------------------------------------------
\subsection{Elasticity Constitutive Models}

The Jacobian for the elasticity equation is
\begin{equation}
J_{G}^{vu} = \frac{\partial G^{v_i}}{\partial u_j}.
\end{equation}
In computing the derivative, we consider the linearized form:
\begin{align}
  \sigma_{ik} &= C_{ikjl} \epsilon_{jl} \\
  \sigma_{ik} &= C_{ikjl} \frac{1}{2} ( u_{j,l} + u_{l,j} ) \\
  \sigma_{ik} &= \frac{1}{2} ( C_{ikjl} + C_{iklj} ) u_{j,l} \\
  \sigma_{ik} &= C_{ikjl} u_{j,l} \\
\end{align}
In computing the Jacobian, we take the derivative of the stress tensor with respect to the 
displacement field,
\begin{equation}
  \frac{\partial}{\partial u_j} \sigma_{ik} = C_{ikjl} \basisscalar[u]_{j,l},
\end{equation}
so we have
\begin{equation}
\boxed{
  J_{g3}^{vu}(i,j,k,l) = -C_{ikjl}
}
\end{equation}
For many elasticity constitutive models we prefer to separate the
stress into the mean stress and deviatoric stress:
\begin{gather}
  \tensor{\sigma} = \sigma^\mathit{mean} \tensor{I} + \tensor{\sigma}^\mathit{dev} \text{, 
where}\\
  \sigma^\mathit{mean} = \frac{1}{3} \Tr(\tensor{\sigma}) = \frac{1}{3} (\sigma_{11} + 
\sigma_{22} + \sigma_{33}).
\end{gather}
Sometimes it is convenient to use pressure (positive pressure corresponds to compression) 
instead of the mean 
stress:
\begin{gather}
  \tensor{\sigma} = -p \tensor{I} + \tensor{\sigma}^\mathit{dev} \text{, where}\\
  p = -\frac{1}{3} \Tr(\tensor{\sigma}).
\end{gather}

The Jacobian with respect to the deviatoric stress is
\begin{align}
  \frac{\partial \sigma^\mathit{dev}_{ik}}{\partial u_j}  &= \frac{\partial}{\partial u_j} 
\left(\sigma_{ik} - 
\frac{1}{3} \sigma_{mm} \delta_{ik} \right) \\
  \frac{\partial \sigma^\mathit{dev}_{ik}}{\partial u_j}  &= C_{ikjl} \basisscalar[u]_{j,l} - 
\frac{1}{3} C_{mmjl} 
\delta_{ik} \basisscalar[u]_{j,l}.
\end{align}
We call these modified elastic constants $C^\mathit{dev}_{ikjl}$, so that we have
\begin{equation}
\boxed{
  C^\mathit{dev}_{ikjl} = C_{ikjl} - \frac{1}{3} C_{mmjl} \delta_{ik}.
}
\end{equation}.

% ----------------------------------------------------------------------
\subsubsection{Isotropic Linear Elasticity}

We implement isotropic linear elasticity both with and without a
reference stress-strain state. With a linear elastic material it is
often convenient to compute the deformation relative to an unknown
initial stress-strain state. Furthermore, when we use an initial
undeformed configuration with zero stress and strain, the reference
stress and strain are zero, so this presents a simplifcation of the
more general case of the stress-strain state relative to the reference
stress-strain state.

Without a reference stress-strain state, we have
\begin{equation}
  \sigma_{ij} = \lambda \epsilon_{kk} \delta_{ij} + 2\mu\epsilon_{ij},
\end{equation}
and with a reference stress-strain state, we have
\begin{equation}
  \sigma_{ij} = \sigma_{ij}^\mathit{ref} + \lambda \left(\epsilon_{kk} - \epsilon_{kk}
^\mathit{ref}\right)
\delta_{ij} + 2\mu\left(\epsilon_{ij}-\epsilon_{ij}^\mathit{ref}\right).
\end{equation}
The mean stress is
\begin{align}
  \sigma^\mathit{mean} &= \frac{1}{3} \sigma_{kk}, \\
  \sigma^\mathit{mean} &= \frac{1}{3} \sigma_{kk}^\mathit{ref} + \left(\lambda+\frac{2}
{3}\mu\right)
\left(\epsilon_{kk}-\epsilon_{kk}^\mathit{ref}\right),
\end{align}
\begin{equation}
  \boxed{
  \sigma^\mathit{mean} = \frac{1}{3} \sigma_{kk}^\mathit{ref} + K \left(\epsilon_{kk}-
\epsilon_{kk}^\mathit{ref}
\right),
}%boxed
\end{equation}
where $K=\lambda+2\mu/3$ is the bulk modulus. 
If the reference stress and reference strain are both zero, then this reduces to
\begin{equation}
  \boxed{
  \sigma^\mathit{mean} = K \epsilon_{kk}.
}%boxed
\end{equation}
The deviatoric stress is
\begin{align}
  \sigma_{ij}^\mathit{dev} &= \sigma_{ij} - \sigma^\mathit{mean}\delta_{ij}, \\
  \sigma_{ij}^\mathit{dev} &= \sigma_{ij}^\mathit{ref} + \lambda\left(\epsilon_{kk}-
\epsilon_{kk}^\mathit{ref}
\right)\delta_{ij} + 2\mu\left(\epsilon_{ij}-\epsilon_{ij}^\mathit{ref}\right) - 
\left(\frac{1}{3}\sigma_{kk}
^\mathit{ref} + \left(\lambda+\frac{2}{3}\mu\right)\left(\epsilon_{kk}-\epsilon_{kk}
^\mathit{ref}\right)\right)
\delta_{ij}, \\
  \sigma_{ij}^\mathit{dev} &= \sigma_{ij}^\mathit{ref} -\frac{1}{3}\sigma_{kk}^\mathit{ref}
\delta_{ij} + 
2\mu\left(\epsilon_{ij}-\epsilon_{ij}^\mathit{ref}\right) - \frac{2}{3}\mu\left(\epsilon_{kk}-
\epsilon_{kk}
^\mathit{ref}\right)\delta_{ij},
\end{align}
\begin{equation}
  \boxed{
  \sigma_{ij}^\mathit{dev} = \left\{ \begin{array}{lcr}
      \sigma_{ii}^\mathit{ref} -\frac{1}{3}\sigma_{kk}^\mathit{ref} + 2\mu\left(\epsilon_{ii}-
\epsilon_{ii}
^\mathit{ref}\right) - \frac{2}{3}\mu\left(\epsilon_{kk}-\epsilon_{kk}^\mathit{ref}\right) & 
\text{if} & i = j, \\
      \sigma_{ij}^\mathit{ref} + 2\mu\left(\epsilon_{ij}-\epsilon_{ij}^\mathit{ref}\right) & 
\text{if} & i \neq j.
    \end{array} \right.
}%boxed
\end{equation}
If the reference stress and reference strain are both zero, then this reduces to
\begin{equation}
  \boxed{
  \sigma_{ij}^\mathit{dev} = \left\{ \begin{array}{lcr}
      2\mu\epsilon_{ii} - \frac{2}{3}\mu\epsilon_{kk} & \text{if} & i = j, \\
      2\mu\epsilon_{ij} & \text{if} & i \neq j.
    \end{array} \right.
  }%boxed
\end{equation}

For isotropic linear elasticity
\begin{align}
  C_{1112} &= C_{1113} = C_{1113} = C_{1121} = C_{1123} = C_{1131} = C_{1132} = 0\\
  C_{1211} &= C_{1213} = C_{1222} = C_{1223} = C_{1231} = C_{1232} = C_{1233} = 0,
\end{align}
and
\begin{align}
  C_{1111} = C_{2222} = C_{3333} &= \lambda + 2 \mu, \\
  C_{1122} = C_{1133} = C_{2233} &= \lambda, \\
  C_{1212} = C_{2323} = C_{1313} &= \mu.
\end{align}
The deviatoric elastic constants are:
\begin{align}
  C^\mathit{dev}_{1111} = C^\mathit{dev}_{2222} = C^\mathit{dev}_{3333} &= \frac{4}{3}\mu, \\
  C^\mathit{dev}_{1122} = C^\mathit{dev}_{1133} = C^\mathit{dev}_{2233} &= -\frac{2}{3}\mu, \\
  C^\mathit{dev}_{1212} = C^\mathit{dev}_{2323} = C^\mathit{dev}_{1313} &= \mu.
\end{align}

\subsubsection{Isotropic Generalized Maxwell Viscoelasticity}

We use the same general formulation for both the simple Maxwell
viscoelastic model and the generalized Maxwell model (several Maxwell
models in parallel). We implement the Maxwell models both with and
without a reference stress-strain state. Note that it is also possible
to specify an initial state variable value (viscous strain). Viscous
flow is completely deviatoric, so we split the stress into volumetric
and deviatoric parts, as described above.  The volumetric part is
identical to that of an isotropic elastic material. The deviatoric
part is given by:
\begin{equation}
  \sigma^\mathit{dev}_{ij}\left(t\right)=2\mu_{tot}\left(\mu_{0}\epsilon^\mathit{dev}_{ij}
    \left(t\right)+\sum_{m=1}^{N}\mu_{m}h^{m}_{ij}\left(t\right)-\epsilon^\mathit{refdev}_{ij}
    \right)+\sigma^\mathit{refdev}_{ij},
\end{equation}
where $\mu_{tot}$ is the total shear modulus of the model, $\mu_{0}$
is the fraction of the shear modulus accommodated by the elastic
spring in parallel with the Maxwell models, the $\mu_{m}$ are the
fraction of the shear modulus accommodated by each Maxwell model
spring, and $\epsilon^{\mathit{refdev}}_{ij}$ and
$\sigma^{\mathit{refdev}}_{ij}$ are the reference deviatioric strain
and stress, respectively. The viscous strain is:
\begin{equation}
h^{m}_{ij}\left(t\right)=\exp\frac{-\Delta
  t}{\tau_{m}}h^{m}_{ij}\left(t_{n}\right)+\Delta h^{m}_{ij},
\end{equation}
where $t_{n}$ is a time between $t=0$ and $t=t$, $\Delta
h^{m}_{ij}$ is the viscous strain between $t=t_{n}$ and
$t=t$, and $\tau_{m}$ is the Maxwell time:
\begin{equation}
  \tau_{m}=\frac{\eta_{m}}{\mu_{tot}\mu_{m}}.
\end{equation}
Approximating the strain rate as constant over each time step,
this is given as:
\begin{equation}
\Delta h^{m}_{ij}=\frac{\tau_{m}}{\Delta t}\left(1-\exp\frac{-\Delta
  t}{\tau_{m}}\right)\left(\epsilon^{\mathit{dev}}_{ij}\left(t\right)-\epsilon^{\mathit{dev}}_{ij}\left(t_{n}\right)\right)=\Delta
h^{m}\left(\epsilon^{\mathit{dev}}_{ij}\left(t\right)-\epsilon^{\mathit{dev}}_{ij}\left(t_n\right)\right).
\end{equation}
The approximation is singular for zero time steps, but a series
expansion may be used for small time-step sizes:
\begin{equation}
  \Delta h^{m}\approx1-\frac{1}{2}\left(\frac{\Delta
    t}{\tau_{m}}\right)+\frac{1}{3!}\left(\frac{\Delta
    t}{\tau_{m}}\right)^{2}-\frac{1}{4!}\left(\frac{\Delta
    t}{\tau_{m}}\right)^{3}+\cdots\,.
\end{equation}
This converges with only a few terms.

% ----------------------------------------------------------------------
\section{Elasticity with Infinitesimal Strain and Prescribed Slip on Faults}

For each fault, which is an internal interface, we add a boundary condition to the elasticity equation prescribing the jump in the displacement field across the fault,
\begin{gather}
  \label{eqn:bc:prescribed_slip}
  \vec{u}^+ - \vec{u}^- - \vec{d}(\vec{x},t) = \vec{0} \text{ on }\Gamma_f,
\end{gather}
where $\vec{u}^+$ is the displacement vector on the ``positive'' side of the fault, $\vec{u}^-$ is the displacement vector on the ``negative'' side of the fault, $\vec{d}$ is the slip vector on the fault, and $\vec{n}$ is the fault normal which points from the negative side of the fault to the positive side of the fault.
We enforce the jump in displacements across the fault using a Lagrange multiplier corresponding to equal and opposite tractions on the two sides of the fault.

We apply conservation of momemtum,
\begin{equation}
  \int_\Omega \rho(\vec{x}) \frac{\partial \vec{v}}{\partial t} \, d\Omega = \int_\Omega \vec{f}(\vec{x},t) \, d\Omega + \int_\Gamma \vec{\tau}(\vec{x},t) \, d\Gamma,
\end{equation}
to a fault interface $\Omega_f$ with boundaries $\Gamma_{f^+}$ and $\Gamma_{f^-}$.
For a fault interface, the body force is zero, $\vec{f}(\vec{x},t) = \vec{0}$.
The tractions on the positive and negative fault faces are
\begin{gather}
  \tau^+(\vec{x},t) = \tensor{\sigma}^+ \cdot \vec{n} + \vec{\lambda} \\
  \tau^-(\vec{x},t) = \tensor{\sigma}^- \cdot \vec{n} - \vec{\lambda},
\end{gather}
where $\vec{\lambda}$ is the Lagrange multiplier that corresponds to the fault traction generating the prescribed slip and $\tensor{\sigma}^+$ and $\tensor{\sigma}^-$ are the stresses in the domain at the positive and negative sides of the fault.
Thus, for a fault interface, we have
\begin{equation}
  \int_{\Omega_f} \rho(\vec{x}) \frac{\partial \vec{v}}{\partial t} \, d\Omega = \int_{\Gamma_{f^+}} \tensor{\sigma} \cdot \vec{n} + \vec{\lambda} \, d\Gamma + \int_{\Gamma_{f^-}} \tensor{\sigma} \cdot \vec{n} - \vec{\lambda} \, d\Gamma.
\end{equation}

\begin{table}[htbp]
  \caption{Mathematical notation for elasticity equation with
    infinitesimal strain and prescribed slip on faults.}
  \label{tab:notation:elasticity:prescribed:slip}
  \begin{tabular}{lcp{3in}}
    \toprule
    {\bf Category} & {\bf Symbol} & {\bf Description} \\
    \midrule
    Unknowns & $\vec{u}$ & Displacement field \\
    & $\vec{v}$ & Velocity field \\
    & $\vec{\lambda}$ & Lagrange multiplier field \\
    Derived quantities & $\tensor{\sigma}$ & Cauchy stress tensor \\
                   & $\tensor{\epsilon}$ & Cauchy strain tensor \\
    Common constitutive parameters & $\rho$ & Density \\
  & $\mu$ & Shear modulus \\
  & $K$ & Bulk modulus \\
Source terms & $\vec{f}$ & Body force per unit volume, for example $\rho \vec{g}$ \\
    & $\vec{d}$ & Slip vector field on the fault corresponding to a
      jump in the displacement field across the fault \\
    \bottomrule
  \end{tabular}
\end{table}

\subsection{Quasistatic}

As in the case of elasticity without faults, we first consider the quasistatic case in which we neglect the inertial term ($\rho \frac{\partial \vec{v}}{\partial t} \approx \vec{0}$).
We place all of the terms in the elasticity equation on the LHS, consistent with implicit time stepping.
Our equation of the conservation of momentum on the fault interface reduces to
\begin{equation}
  \int_{\Gamma_{f^+}} \tensor{\sigma} \cdot \vec{n} + \vec{\lambda} \, d\Gamma + \int_{\Gamma_{f^-}} \tensor{\sigma} \cdot \vec{n} - \vec{\lambda} \, d\Gamma = 0.
\end{equation}
We enforce this equation on each portion of the fault interface along with our prescribed slip constraint, which leads to
\begin{gather}
  \tensor{\sigma} \cdot \vec{n} + \vec{\lambda} = \vec{0} \text{ on } \Gamma_{f^+}, \\
  \tensor{\sigma} \cdot \vec{n} - \vec{\lambda} = \vec{0}\text{ on } \Gamma_{f^-}, \\
  \vec{u}^+ - \vec{u}^- - \vec{d}(\vec{x},t) = \vec{0},  
\end{gather}

Our solution vector consists of both displacements and Lagrange multipliers, and the strong form for the system of equations is
\begin{gather}
  % Solution
  \vec{s}^T = \left( \vec{u} \quad \vec{\lambda} \right)^T \\
  % Elasticity
  \vec{f}(\vec{x},t) + \tensor{\nabla} \cdot \tensor{\sigma}(\vec{u}) = \vec{0} \text{ in }\Omega, \\
  % Neumann
  \tensor{\sigma} \cdot \vec{n} = \vec{\tau}(\vec{x},t) \text{ on }\Gamma_\tau, \\
  % Dirichlet
  \vec{u} = \vec{u}_0(\vec{x},t) \text{ on }\Gamma_u, \\
  % Prescribed slip
  \vec{u}^+ - \vec{u}^- - \vec{d}(\vec{x},t) = \vec{0} \text{ on }\Gamma_f,  \\
  \tensor{\sigma} \cdot \vec{n} = -\vec{\lambda}(\vec{x},t) \text{ on }\Gamma_{f^+}, \text{ and}\\
  \tensor{\sigma} \cdot \vec{n} = +\vec{\lambda}(\vec{x},t) \text{ on }\Gamma_{f^-}.
\end{gather}
We create the weak form by taking the dot product with the trial function $\trialvec[u]$ or $\trialvec[\lambda]$ and integrating over the domain.
After using the divergence theorem and incorporating the Neumann boundary and fault interface conditions, we have
\begin{gather}
  % Elasticity
  \int_\Omega \trialvec[u] \cdot \vec{f}(\vec{x},t) + \nabla \trialvec[v] : -\tensor{\sigma}(\vec{u}) \, d\Omega
  + \int_{\Gamma_\tau} \trialvec[u] \cdot \vec{\tau}(\vec{x},t) \, d\Gamma,
  + \int_{\Gamma_{f}} \trialvec[u^+] \cdot \left(-\vec{\lambda}(\vec{x},t)\right)
  + \trialvec[u^-] \cdot \left(+\vec{\lambda}(\vec{x},t)\right)\, d\Gamma = 0\\
  % Prescribed slip
  \int_{\Gamma_{f}} \trialvec[\lambda] \cdot \left(
    -\vec{u}^+ + \vec{u}^- + \vec{d}(\vec{x},t) \right) \, d\Gamma = 0.
\end{gather}
We have chosen the signs in the last equation so that the Jacobian will be symmetric with respect to the Lagrange multiplier.
We solve the system of equations using implicit time stepping, making use of residuals functions and Jacobians for the LHS.


\subsubsection{Residual Pointwise Functions}

Identifying $F(t,s,\dot{s})$ and $G(t,s)$, we have
\begin{align}
 % Fu
F^u(t,s,\dot{s}) &= \int_\Omega \trialvec[u] \cdot \eqnannotate{\vec{f}(\vec{x},t)}{\vec{f}^u_0} + \nabla \trialvec[u] : \eqnannotate{-\tensor{\sigma}(\vec{u})}{\tensor{f^u_1}} \, d\Omega
  + \int_{\Gamma_\tau} \trialvec[u] \cdot \eqnannotate{\vec{\tau}(\vec{x},t)}{\vec{f}^u_0} \, d\Gamma 
  + \int_{\Gamma_{f}} \trialvec[u^+] \cdot \eqnannotate{\left(-\vec{\lambda}(\vec{x},t)\right)}{\vec{f}^u_0}
  + \trialvec[u^-] \cdot \eqnannotate{\left(+\vec{\lambda}(\vec{x},t)\right)}{\vec{f}^u_0}\, d\Gamma \\
  % Fl
  F^\lambda(t,s,\dot{s}) &= \int_{\Gamma_{f}} \trialvec[\lambda] \cdot \eqnannotate{\left(
    -\vec{u}^+ + \vec{u}^- + \vec{d}(\vec{x},t) \right)}{\vec{f}^\lambda_0} \, d\Gamma, \\
  % Gu
  G^u(t,s) &= 0 \\
  % Gl
  G^\lambda(t,s) &= 0
\end{align}
Compared to the quasistatic elasticity case without a fault, we have simply added additional pointwise functions associated with the fault.
Our fault implementation does not change the formulation for the materials or external Dirichlet or Neumann boundary conditions.

\subsubsection{Jacobian Pointwise Functions}

The LHS Jacobians are:
\begin{align}
  % J_F uu
  J_F^{uu} &= \frac{\partial F^u}{\partial u} + s_\mathit{tshift} \frac{\partial F^u}{\partial \dot{u}}
      = \int_\Omega \nabla \trialvec[u] : -\tensor{C} : \frac{1}{2}(\nabla + \nabla^T)\basisvec[u] 
\, d\Omega 
      = \int_\Omega \trialscalar[u]_{i,k} \, \eqnannotate{\left( -C_{ikjl} \right)}{J_{f3}^{uu}} \, \basisscalar[u]_{j,l}\, d\Omega \\
  % J_F ul
  J_F^{u\lambda} &= \frac{\partial F^u}{\partial \lambda} + s_\mathit{tshift} \frac{\partial F^u}{\partial \dot{\lambda}}
      = \int_{\Gamma_{f}} \trialscalar[u^+]_i \eqnannotate{\left(-\delta_{ij}\right)}{J^{u\lambda}_{f0}} \basisscalar[\lambda]_j
                   + \trialscalar[u^-]_i \eqnannotate{\left(+\delta_{ij}\right)}{J^{u\lambda}_{f0}} \basisscalar[\lambda]_j\, d\Gamma, \\
  % J_F lu
  J_F^{\lambda u} &= \frac{\partial F^\lambda}{\partial u} + s_\mathit{tshift} \frac{\partial F^\lambda}{\partial \dot{u}}
      = \int_{\Gamma_{f}} \trialscalar[\lambda]_i 
                    \eqnannotate{\left(-\delta_{ij}\right)}{J^{\lambda u}_{f0}} \basisscalar[u^+]_j
                    + \trialscalar[\lambda]_i \eqnannotate{\left(+\delta_{ij}\right)}{J^{\lambda u}_{f0}} \basisscalar[u^-]_j \, d\Gamma, \\
  % J_F ll
  J_F^{\lambda \lambda} &= \tensor{0}
\end{align}
This LHS Jacobian has the structure
\begin{equation}
  J_F = \left( \begin{array} {cc} J_F^{uu} & J_F^{u\lambda} \\ J_F^{\lambda u} & 0 \end{array} \right)
      = \left( \begin{array} {cc} J_F^{uu} & C^T \\ C & 0 \end{array} \right),
\end{equation}
where $C$ contains entries of $\pm 1$ for degrees of freedom on the two sides of the fault. The Schur complement of $J$ with respect to $J_F^{uu}$ is $-C\left(J_F^{uu}\right)^{-1}C^T$.


\subsection{Dynamic}

The equation prescribing fault slip is independent of the Lagrange multiplier, so we do not have a system of equations that we can put in
the form $\dot{s} = G^*(t,s)$.
Instead, we have a differential-algebraic set of equations (DAEs), which we solve using an implicit-explicit (IMEX) time integration scheme.
As in the case of dynamic elasticity without faults, we introduce the velocity ($\vec{v}$) as an unknown to turn the elasticity equation into two first order equations.
Our constraint for prescribed slip is
\begin{equation}
  \vec{u}^+ - \vec{u}^- - \vec{d} = \vec{0},
\end{equation}
where $\vec{u}$ is the displacement vector and $\vec{d}$ is the slip vector.
In order to match the order of the time derivative in the elasticity equation, we take the second derivative of the prescribed slip equation with respect to time,
\begin{equation}
  \frac{\partial \vec{v}^+}{\partial t} - \frac{\partial \vec{v}^-}{\partial t} - \frac{\partial^2 \vec{d}}{\partial t^2} = \vec{0}.
\end{equation}
This means that our differential algebraic equations has a differentiation index of 2.

The strong form for our system of equations is:
\begin{gather}
  % Solution
  \vec{s}^T = \left( \vec{u} \quad \vec{v} \quad \vec{\lambda} \right)^T \\
  % Displacement-velocity
  \frac{\partial \vec{u}}{\partial t} = \vec{v}, \\
  % Elasticity
  \rho(\vec{x}) \frac{\partial \vec{v}}{\partial t} = \vec{f}(\vec{x},t) + \nabla \cdot \tensor{\sigma}(\vec{u}), \\
  % Neumann BC
  \tensor{\sigma} \cdot \vec{n} = \vec{\tau} \text{ on } \Gamma_\tau. \\
  % Dirichlet BC
  \vec{u} = \vec{u}_0 \text{ on } \Gamma_u, \\
  % Presribed slip
  \frac{\partial \vec{v}^+}{\partial t} - \frac{\partial \vec{v}^-}{\partial t} - \frac{\partial^2 \vec{d}(\vec{x},t)}{\partial t^2} = \vec{0}, \\
  \int_{\Omega_f} \rho(\vec{x}) \frac{\partial \vec{v}}{\partial t} \, d\Omega = \int_{\Gamma_{f^+}} \tensor{\sigma} \cdot \vec{n} + \vec{\lambda} \, d\Gamma + \int_{\Gamma_{f^-}} \tensor{\sigma} \cdot \vec{n} - \vec{\lambda} \, d\Gamma.
\end{gather}
We generate the weak form in the usual way,
\begin{gather}
  % Displacement-velocity
  \int_{\Omega} \trialvec[u] \cdot \frac{\partial \vec{u}}{\partial t} \, d\Omega =  \int_{\Omega} \trialvec[u] \cdot \vec{v} \, d\Omega, \\
  % Elasticity
  \begin{multlined}
  \int_{\Omega} \trialvec[v] \cdot \rho(\vec{x}) \frac{\partial \vec{v}}{\partial t} \, d\Omega  = \int_\Omega \trialvec[v] \cdot \vec{f}(\vec{x},t) + \nabla \trialvec[v] : -\tensor{\sigma}(\vec{u}) \, d\Omega + \int_{\Gamma_\tau} \trialvec[v] \cdot \vec{\tau}(\vec{x},t) \, d\Gamma \\
  + \int_{\Gamma_{f}} \trialvec[v^+] \cdot \left(-\vec{\lambda}(\vec{x},t)\right) + \trialvec[v^-] \cdot \left(+\vec{\lambda}(\vec{x},t)\right)\, d\Gamma,
  \end{multlined}\\
  % Prescribed slip
  \int_{\Gamma_f} \trialvec[\lambda] \cdot \left(\frac{\partial \vec{v}^+}{\partial t} - \frac{\partial \vec{v}^-}{\partial t} - \frac{\partial^2 \vec{d}(\vec{x},t)}{\partial t^2} \right) \, d\Gamma = 0. \\
  \int_{\Omega_f} \trialvec[\lambda] \cdot \rho(\vec{x}) \frac{\partial \vec{v}}{\partial t} \, d\Omega = \int_{\Gamma_{f^+}} \trialvec[\lambda] \cdot \left( \tensor{\sigma} \cdot \vec{n} + \vec{\lambda} \right) \, d\Gamma + \int_{\Gamma_{f^-}} \trialvec[\lambda] \cdot \left( \tensor{\sigma} \cdot \vec{n} - \vec{\lambda} \right) \, d\Gamma.
\end{gather}

For compatibility with PETSc TS IMEX implementations, we need $\dot{\vec{s}}$ on the LHS for the explicit part (displacement-velocity and elasticity equations) and we need $\vec{\lambda}$ in the equation for the implicit part (prescribed slip equation).
We first focus on the explicit part and select numerical quadrature that yields a lumped mass matrix, $M$, so that we have
\begin{gather}
  % Displacement-velocity
  \label{eqn:displacement:velocity:prescribed:slip:weak:form}
  \frac{\partial \vec{u}}{\partial t} = M_u^{-1} \int_{\Omega} \trialvec[u] \cdot \vec{v} \, d\Omega, \\
  % Elasticity
  \label{eqn:elasticity:prescribed:slip:dynamic:weak:form}
  \begin{multlined}
  \frac{\partial \vec{v}}{\partial t} = M_v^{-1} \int_\Omega \trialvec[v] \cdot \vec{f}(\vec{x},t) + \nabla \trialvec[v] : -\tensor{\sigma}(\vec{u}) \, d\Omega + M_v^{-1} \int_{\Gamma_\tau} \trialvec[v] \cdot \vec{\tau}(\vec{x},t) \, d\Gamma \\
  + M_{v^+}^{-1} \int_{\Gamma_{f}} \trialvec[v^+] \cdot \left(-\vec{\lambda}(\vec{x},t)\right) \, d\Gamma + M_{v^-}^{-1} \int_{\Gamma_{f}}\trialvec[v^-] \cdot \left(+\vec{\lambda}(\vec{x},t)\right) \, d\Gamma,
  \end{multlined}\\
  M_u = \mathit{Lump}\left( \int_\Omega \trialscalar[u]_i \delta_{ij} \basisscalar[u]_j \, d\Omega \right), \\
  M_v = \mathit{Lump}\left( \int_\Omega \trialscalar[v]_i \rho(\vec{x}) \delta_{ij} \basisscalar[v]_j \, d\Omega \right).
\end{gather}
For the implicit part, we can separate the integration of the weak form for negative and positive sides of the fault interface, which yields
\begin{gather}
  M_{v^+} \frac{\partial \vec{v}^+}{\partial t} = \int_{\Gamma_{f^+}} \trialvec[\lambda] \cdot \left( \tensor{\sigma} \cdot \vec{n} + \vec{\lambda} \right) \, d\Gamma, \\
  M_{v^-} \frac{\partial \vec{v}^-}{\partial t} = \int_{\Gamma_{f^-}} \trialvec[\lambda] \cdot \left( \tensor{\sigma} \cdot \vec{n} - \vec{\lambda} \right) \, d\Gamma.
\end{gather}
Using these equations to substitute in the expressions for the time derivative of the velocity on the negative and positive sides of the fault into the prescribed slip constraint equation yields
\begin{equation}
  \label{eqn:elasticity:prescribed:slip:dynamic:DAE:weak:form}
  M_{v^+}^{-1} \int_{\Gamma_f^+} \trialvec[\lambda] \cdot \left(\tensor{\sigma} \cdot \vec{n} + \vec{\lambda}\right) \, d\Gamma + M_{v^-}^{-1} \int_{\Gamma_f^-} \trialvec[\lambda] \cdot \left( -\tensor{\sigma} \cdot \vec{n} + \vec{\lambda} \right) \, d\Gamma - \int_{\Gamma_f} \trialvec[\lambda] \cdot \frac{\partial^2 \vec{d}}{\partial t^2} \, d\Gamma = \vec{0}.
\end{equation}


\subsubsection{Residual Pointwise Functions}

Combining the explicit parts of the weak form in equations~\ref{eqn:displacement:velocity:prescribed:slip:weak:form} and \ref{eqn:elasticity:prescribed:slip:dynamic:weak:form} with the implicit part of the weak form in equation~\ref{eqn:elasticity:prescribed:slip:dynamic:DAE:weak:form} and identifying $F(t,s,\dot{s})$ and $G(t,s)$, we have
\begin{gather}
  % Fu
  F^u(t,s,\dot{s}) = \frac{\partial \vec{u}}{\partial t} \\
  % Fv
  F^v(t,s,\dot{s}) = \frac{\partial \vec{v}}{\partial t} \\
  % Fl
    F^\lambda(t,s,\dot{s}) = \eqnannotate{M_{v^+}^{-1}}{c^+} \int_{\Gamma_f^+} \trialvec[\lambda] \cdot \eqnannotate{\left(\tensor{\sigma} \cdot \vec{n} + \vec{\lambda}\right)}{f^\lambda_0} \, d\Gamma + \eqnannotate{M_{v^-}^{-1}}{c^-} \int_{\Gamma_f^-} \trialvec[\lambda] \cdot \eqnannotate{\left( -\tensor{\sigma} \cdot \vec{n} + \vec{\lambda} \right)}{f^\lambda_0} \, d\Gamma - \int_{\Gamma_f} \trialvec[\lambda] \cdot \eqnannotate{\frac{\partial^2 \vec{d}}{\partial t^2}}{f^\lambda_0} \, d\Gamma = \vec{0}, \\
  % Gu
  G^u(t,s) = \eqnannotate{M_{u}^{-1}}{c} \int_\Omega \trialvec[u] \cdot \eqnannotate{\vec{v}}{\vec{g}^u_0} \, d\Omega, \\
  % Gv
  G^v(t,s) =  \eqnannotate{M_{v}^{-1}}{c} \left( \int_\Omega \trialvec[v] \cdot \eqnannotate{\vec{f}(\vec{x},t)}{\vec{g}^v_0} + \nabla \trialvec[v] : \eqnannotate{-\tensor{\sigma}(\vec{u})}{\tensor{g^v_1}} \, d\Omega
  + \int_{\Gamma_\tau} \trialvec[v] \cdot \eqnannotate{\vec{\tau}(\vec{x},t)}{\vec{g}^v_0} \, d\Gamma,
  + \int_{\Gamma_{f}} \trialvec[v^+] \cdot \eqnannotate{\left(-\vec{\lambda}(\vec{x},t)\right)}{\vec{g}^v_0}
             + \trialvec[v^-] \cdot \eqnannotate{\left(+\vec{\lambda}(\vec{x},t)\right)}{\vec{g}^v_0} \, d\Gamma \right), \\
  % Gl
  G^\lambda(t,s) = 0
\end{gather}
The integrals for the explicit part are all weighted by the inverse of the lumped mass matrix.
For the implicit part, only the integrals over the positive and negative sides of the fault are weighted by the inverse of the lumped mass matrix.

\subsubsection{Jacobian Pointwise Functions}

For the explicit part we have pointwise functions for computing the lumped LHS Jacobian. These are exactly the same pointwise functions as in the dynamic case without a fault,
\begin{align}
  % J_F uu
  J_F^{uu} &= \frac{\partial F^u}{\partial u} + s_\mathit{tshift} \frac{\partial F^u}{\partial \dot{u}} =
             \int_\Omega \trialscalar[u]_i \eqnannotate{s_\mathit{tshift} \delta_{ij}}{J^{uu}_{f0}} \basisscalar[u]_j  \, d\Omega, \\
  % J_F vv
  J_F^{vv} &= \frac{\partial F^v}{\partial v} + s_\mathit{tshift} \frac{\partial F^v}{\partial \dot{v}} =
             \int_\Omega \trialscalar[v]_i \eqnannotate{\rho(\vec{x}) s_\mathit{tshift} \delta_{ij}}{J ^{vv}_{f0}} \basisscalar[v]_j \, d\Omega
\end{align}
For the implicit part, we have pointwise functions for the LHS Jacobians associated with the prescribed slip,
\begin{gather}
  \begin{multlined}
  % J_F lu
  J_F^{\lambda u} = \frac{\partial F^\lambda}{\partial u} + s_\mathit{tshift} \frac{\partial F^\lambda}{\partial \dot{u}} = 
  \eqnannotate{M_{v^+}^{-1}}{c^+} \int_{\Gamma_{f^+}} \trialscalar[\lambda]_i \eqnannotate{ C_{kijl} n_k}{J^{\lambda u}_{f1}} \basisscalar[u]_{j,l} \, d\Gamma + \eqnannotate{M_{v^-}^{-1}}{c^-} \int_{\Gamma_{f^-}} \trialscalar[\lambda]_i \eqnannotate{- C_{kijl} n_k}{J^{\lambda u}_{f1}} \basisscalar[u]_{j,l} \, d\Gamma
                  \end{multlined} \\
  % J_F ll
  J_F^{\lambda \lambda} = \frac{\partial F^\lambda}{\partial \lambda} + s_\mathit{tshift} \frac{\partial F^\lambda}{\partial \dot{\lambda}} =
  \eqnannotate{M_{v^+}^{-1}}{c^+} \int_{\Gamma_{f^+}} \trialscalar[\lambda]_i \eqnannotate{ \delta_{ij}}{J^{\lambda\lambda}_{f0}} \basisscalar[\lambda]_j \, d\Gamma
            + \eqnannotate{M_{v^-}^{-1}}{c^-} \int_{\Gamma_{f^-}} \trialscalar[\lambda]_i \eqnannotate{ \delta_{ij}}{J^{\lambda\lambda}_{f0}} \basisscalar[\lambda]_j \, d\Gamma
\end{gather}



% End of file
% ----------------------------------------------------------------------
\section{Incompressible Isotropic Elasticity with Infinitesimal Strain (Bathe) and No Faults or Inertia}

Building from the elasticity equation
(equations~\ref{eqn:velocity:strong:form}
and~\ref{eqn:elasticity:order1:strong:form}), we consider an
incompressible material. As the bulk modulus ($K$) approaches
infinity, the volumetric strain ($\Tr(\epsilon)$) approaches zero and
the pressure remains finite, $p = -K \Tr(\epsilon)$. We consider
pressure $p$ as an independent variable and decompose the stress into the
pressure and deviatoric components. As a result, we write the stress tensor in terms of both the displacement and pressure fields,
\begin{equation}
  \tensor{\sigma}(\vec{u},p) = \tensor{\sigma}^\mathit{dev}(\vec{u}) - p\tensor{I}.
\end{equation}

\subsection{Notation}
\begin{itemize}
\item Unknowns
  \begin{description}
  \item[$\vec{u}$] Displacement field
  \item[$p$] Pressure field (positive pressure corresponds to negative stress)
  \end{description}
\item Derived quantities
  \begin{description}
    \item[$\tensor{\sigma}$] Stress tensor
    \item[$\tensor{\epsilon}$] Strain tensor
  \end{description}
\item Constitutive parameters
  \begin{description}
  \item[$\mu$] Shear modulus
  \item[$K$] Bulk modulus
  \item[$\rho$] Density
  \end{description}
\item Source terms
  \begin{description}
    \item[$\vec{f}$] Body force per unit volume, for example $\rho \vec{g}$
  \end{description}
\end{itemize}


\subsection{Implicit Time Stepping}

We only consider the case of an incompressible material while
neglecting inertia. The time dependence only arises from
history-dependent constitutive equations and boundary conditions. We
have
\begin{gather}
  % Solution
  \vec{s}^T = \left( \vec{u} \quad \ p \right)^T, \\
  % Elasticity
  \vec{0} = \vec{f}(t) + \tensor{\nabla} \cdot \left(\tensor{\sigma}^\mathit{dev}(\vec{u}) - p\tensor{I}\right) \text{ in }\Omega, \\
  % Pressure
  0 = \vec{\nabla} \cdot \vec{u} + \frac{p}{K}, \\
  % Neumann
  \tensor{\sigma} \cdot \vec{n} = \vec{\tau} \text{ on }\Gamma_\tau, \\
  % Dirichlet
  \vec{u} = \vec{u}_0 \text{ on }\Gamma_u, \\
  p = p_0 \text{ on }\Gamma_p.
\end{gather}

Using trial functions $\trialvec[u]$ and $\trialscalar[p]$ and
incorporating the Neumann boundary conditions, we write the weak form
as
\begin{gather}
  % Displacement
  0 = 
  \int_\Omega \trialvec[u] \cdot \vec{f}(t) + \nabla \trialvec[u] : \left(-\tensor{\sigma}^\mathit{dev}(\vec{u}) + p\tensor{I}
  \right)\, d\Omega + \int_{\Gamma_\tau} \trialvec[u] \cdot \vec{\tau}(t) \, d\Gamma, \\
  % Pressure
  0 = \int_\Omega \trialscalar[p] \cdot \left(\vec{\nabla} \cdot \vec{u} + \frac{p}{K} \right) 
\, d\Omega.
\end{gather}

Identifying $G(t,s)$, we have
\begin{gather}
  \label{eqn:incompressible:elasticity:displacement}
  0 = \int_\Omega \trialvec[u] \cdot \eqnannotate{\vec{f}(t)}{g_0^u} + \nabla \trialvec[u] :
  \eqnannotate{\left(-\tensor{\sigma}^\mathit{dev}(\vec{u}) + p\tensor{I}\right)}{g_1^u}  \, d\Omega
  + \int_{\Gamma_\tau} \trialvec[u] \cdot \eqnannotate{\vec{\tau}(t)}{g_0^u} \, d\Gamma, \\
%
  \label{eqn:incompressible:elasticity:pressure}
  0 = \int_\Omega \trialscalar[p] \cdot \eqnannotate{\left(\vec{\nabla} \cdot \vec{u} + 
\frac{p}{K} \right)}{g_0^p} \, d\Omega.
\end{gather}


\subsubsection{Jacobians}

With two fields we have four Jacobians for the RHS associated with the coupling of 
the two fields.
\begin{align}
  J_G^{uu} &= \frac{\partial G^u}{\partial u} = \int_\Omega \nabla \trialvec[u] : 
\frac{\partial}{\partial u}(-
\tensor{\sigma}^\mathit{dev}) \, d\Omega 
  = \int_\Omega \trialscalar[u]_{i,k} \, \eqnannotate{\left(-C^\mathit{dev}_{ikjl}\right)}
{J_{g3}^{uu}}  \, 
\basisscalar[u]_{j,l}\, d\Omega \\
  J_G^{up} &= \frac{\partial G^u}{\partial p} = \int_\Omega \nabla\trialvec[u] : \tensor{I} 
\basisscalar[p] \,  d\Omega = \int_\Omega \trialscalar[u]_{i,k} \eqnannotate{\delta_{ik}}{J_{g2}^{up}} \, 
\basisscalar[p] \, d\Omega \\
%
  J_G^{pu} &= \frac{\partial G^p}{\partial u} = \int_\Omega \trialscalar[p] \left(\vec{\nabla} 
\cdot \basisvec[u]\right) \, d\Omega = \int_\Omega \trialscalar[p] \eqnannotate{\delta_{jl}}{J_{g1}^{pu}} 
\basisscalar[u]_{j,l} \, d\Omega\\
  J_G^{pp} &= \frac{\partial G^p}{\partial p} = \int_\Omega \trialscalar[p] \eqnannotate{\frac{1}
{K}}{J_{g0}^{pp}} \basisscalar[p] \, d\Omega
\end{align}

For isotropic, linear incompressible elasticity, the deviatoric elastic constants are:
\begin{align}
    C_{1111} &= C_{2222} = C_{3333} = +\frac{4}{3} \mu \\
    C_{1122} &= C_{1133} = C_{2233} = -\frac{2}{3} \mu \\
    C_{1212} &= C_{1313} = C_{2323} = \mu
\end{align}

%% ----------------------------------------------------------------------
\section{Poroelasticity with Infinitesimal Strain and No Faults}

We base this formulation for poroelsticity on Zheng et al. and
Detournay and Cheng (1993). We assume a slightly compressible fluid
that completely saturates a porous solid, undergoing infinitesimal
strain.

We begin with the conservation of linear momentum, including inertia,
borrowed from linear elasticity:
\begin{equation}
  \rho_s\frac{\partial^2 \vec{u}}{\partial t^2} = \vec{f}(t) + \nabla \cdot \tensor{\sigma}(\vec{u},p).
\end{equation}
Enforcing mass balance of the fluid gives
\begin{gather}
  \frac{\partial \zeta(\vec{u},p)}{\partial t} + \nabla \cdot \vec{q}(p) =
  \gamma(\vec{x},t) \text{ in } \Omega, \\
  %
  \vec{q} \cdot \vec{n} = q_0(\vec{x},t) \text{ on }\Gamma_q, \\
  %
  p = p_0(\vec{x},t) \text{ on }\Gamma_p,
\end{gather}
where $\zeta$ is the variation in fluid content, $\vec{q}$ is the rate
of fluid volume crossing a unit area of the porous solid, $\gamma$ is
the rate of injected fluid per unit volume of the porous solid, $q_0$
is the outward fluid velocity normal to the boundary $\Gamma_q$, and
$p_0$ is the fluid pressure on boundary $\Gamma_p$.

We require the fluid flow to follow Darcy's law (Navier-Stokes
equation neglecting inertial effects),
\begin{equation}
  \vec{q}(p) = -\frac{\tensor{k}}{\mu_{f}}(\nabla p - \vec{f}_f),
\end{equation}
where $\tensor{k}$ is the intrinsic permeability, $\mu_f$ is the viscosity of the
fluid, $p$ is the fluid pressure, and $\vec{f}_f$ is the body force
in the fluid. If gravity is included in a problem, then usually
$\vec{f}_f = \rho_f \vec{g}$, where $\rho_f$ is the density of the
fluid and $\vec{g}$ is the gravitational acceleration vector.

\subsection{Constitutive Behavior}

We assume linear elasticity for the solid phase, so the constitutive behavior can be expressed
as
\begin{equation}
  \tensor{\sigma}(\vec{u},p) = \tensor{C} : \tensor{\epsilon} - \alpha p \tensor{I},
\end{equation}
where $\tensor{\sigma}$ is the stress tensor, $\tensor{C}$ is the
tensor of elasticity constants, $\alpha$ is the Biot coefficient
(effective stress coefficient), $\tensor{\epsilon}$ is the strain
tensor, and $\tensor{I}$ is the identity tensor.  For this case, we
will assume that the material properties are isotropic, resulting in
the following formulation for the stress tensor:
\begin{equation}
  \tensor{\sigma}(\vec{u},p) = \tensor{C}:\tensor{\epsilon} - \alpha p \tensor{I}
  = \lambda \tensor{I} \epsilon_{v} + 2 \mu - \alpha \tensor{I} p
\end{equation}
where $\lambda$ and $\mu$ are Lam\'e's parameters,
$\lambda = K_{d} - \frac{2 \mu}{3}$, $\mu$ is the shear modulus, and
the volumetric strain is defined as
$\epsilon_{v} = \nabla \cdot \vec{u}$.

For the constitutive behavior of the fluid, we use the volumetric
strain to couple the fluid-solid behavior,
\begin{gather}
  \zeta(\vec{u},p) = \alpha \Tr({\tensor{\epsilon}}) + \frac{p}{M}, \\
  %
  \frac{1}{M} = \frac{\alpha-\phi}{K_s} + \frac{\phi}{K_f},
\end{gather}
where $1/M$ is the specific storage coefficient at constant strain,
$K_s$ is the bulk modulus of the solid, and $K_f$ is the bulk modulus
of the fluid. We can write the trace of the strain tensor as the dot
product of the gradient and displacement field, so we have
\begin{equation}
  \zeta(\vec{u},p) = \alpha (\nabla \cdot \vec{u}) + \frac{p}{M}.
\end{equation}

\begin{table}[htbp]
  \caption{Mathematical notation for poroelasticity with
    infinitesimal strain.}
  \label{tab:notation:poroelasticity}
  \begin{tabular}{lcp{3.5in}}
    \toprule
    {\bf Category}                 & {\bf Symbol}        & {\bf Description}                                                                                             \\
    \midrule
    Unknowns                       & $\vec{u}$           & Displacement field                                                                                            \\
                                   & $\vec{v}$           & Velocity field                                                                                                \\
                                   & $p$                 & Pressure field (corresponds to pore fluid pressure)                                                           \\
                                   & $\epsilon_{v}$      & Volumetric (trace) strain                                                                                     \\
                                   & $P$                 & Time derivative of pressure field                                                                             \\
                                   & $E_{v}$             & Time derivative of volumetric (trace) strain                                                                  \\
    \hline
    Derived quantities             & $\tensor{\sigma}$   & Cauchy stress tensor                                                                                          \\
                                   & $\tensor{\epsilon}$ & Cauchy strain tensor                                                                                          \\
                                   & $\zeta$             & Variation of fluid content (variation of fluid vol. per unit vol. of PM), $\alpha \epsilon_{v} + \frac{p}{M}$ \\
                                   & $\rho_{b}$          & Bulk density, $\left(1 - \phi\right) \rho_{s} + \phi \rho_{f}$                                                \\
                                   & $\vec{q}$           & Darcy flux, $-\frac{\tensor{k}}{\mu_{f}} \cdot \left(\nabla p - \vec{f}_{f}\right)$                           \\
                                   & $M$                 & Biot Modulus, $\frac{K_{f}}{\phi} + \frac{K_{s}}{\alpha - \phi}$                                              \\
    \hline
    Common constitutive parameters & $\rho_{f}$          & Fluid density                                                                                                 \\
                                   & $\rho_{s}$          & Solid (matrix) density                                                                                        \\
                                   & $\phi$              & Porosity                                                                                                      \\
                                   & $\tensor{k}$        & Permeability                                                                                                  \\
                                   & $\mu_{f}$           & Fluid viscosity                                                                                               \\
                                   & $K_{s}$             & Solid grain bulk modulus                                                                                      \\
                                   & $K_{f}$             & Fluid bulk modulus                                                                                            \\
                                   & $K_{d}$             & Drained bulk modulus                                                                                          \\
                                   & $\alpha$            & Biot coefficient, $1 - \frac{K_{d}}{K_{s}}$                                                                   \\
    \hline
    Source terms                   & $\vec{f}$           & Body force per unit volume, for example: $\rho_{b} \vec{g}$                                                   \\
                                   & $\vec{f}_{f}$       & Fluid body force, for example: $\rho_{f} \vec{g}$                                                             \\
                                   & $\gamma$            & Source density; rate of injected fluid per unit volume of the porous solid                                    \\
    \bottomrule
  \end{tabular}
\end{table}


\subsection{Quasistatic}

For ease of solution in the quasistatic case, we introduce a third
variable in the form of volumetric strain ($\epsilon_v$).  The
strong form of the problem may be expressed as
\begin{gather}
  % Solution
  \vec{s}^{T} = \left(\vec{u} \quad p \quad \epsilon_v\right), \\
  % Elasticity
  \vec{f}(t) + \nabla \cdot \tensor{\sigma}(\vec{u},p) = \vec{0} \text{ in } \Omega, \\
  % Pressure
  \frac{\partial \zeta(\vec{u},p)}{\partial t} - \gamma(\vec{x},t) + \nabla \cdot \vec{q}(p) = 0 \text{ in } \Omega, \\
  % Vol. Strain
  \nabla \cdot \vec{u} - \epsilon_{v} = 0 \text{ in } \Omega, \\
  % Neumann traction
  \tensor{\sigma} \cdot \vec{n} = \vec{\tau}(\vec{x},t) \text{ on } \Gamma_{\tau}, \\
  % Dirichlet displacement
  \vec{u} = \vec{u}_0(\vec{x}, t) \text{ on } \Gamma_{u}, \\
  % Neumann flow
  \vec{q} \cdot \vec{n} = q_0(\vec{x}, t) \text{ on } \Gamma_{q}, \text{ and } \\
  % Dirichlet pressure
  p = p_0(\vec{x},t) \text{ on } \Gamma_{p}.
\end{gather}
We place all terms for the elasticity, pressure, an volumetric strain
equations on the left-hand-side, consistent with PETSc TS implicit
time stepping.

%
We create the weak form by taking the dot product with the trial
functions $\trialvec[u]$, $\trialscalar[p]$, and
$\trialscalar[\epsilon_{v}]$ and
integrating over the domain:
\begin{gather}
  % Weak conservation of momentum
  \int_\Omega \trialvec[u] \cdot \left( \vec{f}(\vec{x},t) + \tensor{\nabla} \cdot \tensor{\sigma} (\vec{u},p) \right) \, d\Omega = 0, \\
  % Weak conservation of mass
  \int_\Omega  \trialscalar[p] \left( \frac{\partial \zeta(\vec{u},p)}{\partial t} - \gamma(\vec{x},t) + \nabla \cdot \vec{q}(p)\right) \, d\Omega = 0,\\
  % Weak vol. strain
  \int_{\Omega} \trialscalar[\epsilon_{v}]\cdot \left( \nabla \cdot \vec{u} - \epsilon_v \right) \, d\Omega.
\end{gather}
%
Applying the divergence theorem to the first two equations and
incorporating the Neumann boundary conditions yields
\begin{gather}
  % Weak conservation of momentum
  \int_\Omega \trialvec[u] \cdot \vec{f}(\vec{x},t) + \nabla \trialvec[u] : -\tensor{\sigma}(\vec{u},p_f) \,
  d\Omega + \int_{\Gamma_\tau} \trialvec[u] \cdot \vec{\tau}(\vec{x},t) \, d\Gamma = 0, \\
  % Weak conservation of mass
  \int_\Omega  \trialscalar[p] \left( \frac{\partial \zeta(\vec{u},p_f)}{\partial t} - \gamma(\vec{x},t)\right)
  + \nabla \trialscalar[p] \cdot \left(-\vec{q}(p_f)\right) \, d\Omega + \int_{\Gamma_q} \trialscalar[p] q_0(\vec{x},t))\, d\Gamma = 0, \text{ and } \\
  % Weak vol. strain
  \int_{\Omega} \trialscalar[\epsilon_{v}] \cdot \left(\nabla \cdot \vec{u} - \epsilon_{v} \right) d\Omega = 0
\end{gather}

\subsubsection{Residual Pointwise Functions}

Identifying $F(t,s,\dot{s})$ and $G(t,s)$we have
\begin{align}
  % Displacement
  F^u(t,s,\dot{s})              & = \int_\Omega \trialvec[u] \cdot \eqnannotate{\vec{f}(\vec{x},t)}{\vec{f}^u_0}
  + \nabla \trialvec[u] : \eqnannotate{-\tensor{\sigma}(\vec{u},p_f)}{\tensor{f}^u_1} \, d\Omega
  + \int_{\Gamma_\tau} \trialvec[u] \cdot \eqnannotate{\vec{\tau}(\vec{x},t)}{\vec{f}^u_0} \, d\Gamma,                                                                               \\
  % Pressure
  F^p(t,s,\dot{s})              & = \int_\Omega  \trialscalar[p] \left[\eqnannotate{\frac{\partial \zeta(\vec{u},p_f)}{\partial t} - \gamma(\vec{x},t)} {f^p_0}\right]
  + \nabla \trialscalar[p] \cdot \eqnannotate{-\vec{q}(p_f)}{\vec{f}^p_1} \, d\Omega
  + \int_{\Gamma_q} \trialscalar[p] (\eqnannotate{q_0(\vec{x},t)}{f^p_0}) \, d\Gamma,                                                                                                \\
  % Volumetric Strain
  F^{\epsilon_{v}}(t,s,\dot{s}) & = \int_{\Omega} \trialscalar[\epsilon_{v}] \cdot \eqnannotate{\left(\nabla \cdot \vec{u} - \epsilon_{v} \right)}{f^{\epsilon_{v}}_{0}} \, d\Omega. \\
  G^u(t,s)                      & = 0,                                                                                                                                               \\
  G^p(t,s)                      & = 0,                                                                                                                                               \\
  G^{\epsilon_v}                & = 0.
\end{align}

\subsubsection{Jacobian Pointwise Functions}

Three field results in a potential nine Jacobian pointwise functions for the LHS:

\begin{align}
  %
  % JF_UU
  % Jf3uu
  J_F^{uu}                       & = \frac{\partial F^u}{\partial u} + t_{shift} \frac{\partial F^u}{\partial \dot{u}} = \int_{\Omega} \nabla \trialvec[u] : \frac{\partial}{\partial u} (- \sigma(\vec{u},p,\epsilon_{v})) \
  d\Omega = \int_{\Omega} \nabla \trialvec[u] : \frac{\partial}{\partial u} (-(\tensor{C}:\tensor{\varepsilon} -\alpha p \tensor{I})) \ d\Omega                                                                                                               \\
                                 & = \int_{\Omega} \nabla \trialvec[u] : -\tensor{C}: \frac{1}{2} (\nabla + \nabla^T) \basisvec[u] \ d\Omega = \int_{\Omega} \trialscalar[u]_{i,k}
  \eqnannotate{\left(-C_{ikjl}\right)}{J_{f3}^{uu}} \basisscalar[u]_{j,l} \ d\Omega                                                                                                                                                                           \\
  %
  % JF_UP
  % Jf2up
  J_F^{up}                       & = \frac{\partial F^u}{\partial p} + t_{shift} \frac{\partial F^u}{\partial \dot{p}} = \int_{\Omega} \nabla \trialvec[u] : \frac{\partial}{\partial p}(-(\tensor{C}:\tensor{\varepsilon} -\alpha p \tensor{I})) \ d\Omega =
  \int_{\Omega} \trialscalar[u]_{i,j} \eqnannotate{\left(\alpha \delta_{ij}\right)}{J_{f2}^{up}} \basisscalar[p] \ d\Omega                                                                                                                                    \\
  %
  % JF_UE
  % Jf2ue
  J_F^{u \epsilon_{v}}           & = \frac{\partial F^u}{\partial \epsilon_{v}} + t_{shift} \frac{\partial F^u}{\partial \dot{\epsilon_{v}}} = \int_{\Omega} \nabla \trialvec[u] : \frac{\partial}{\partial \epsilon_{v}}
  (-\sigma(\vec{u},p,\epsilon_{v})) \ d\Omega = \int_{\Omega} \nabla \trialvec[u] :
  \frac{\partial}{\partial \epsilon_{v}} (-(\tensor{C}:\tensor{\varepsilon} -\alpha p \tensor{I})) \ d\Omega                                                                                                                                                  \\
                                 & = \int_{\Omega} \nabla \trialvec[u] : \frac{\partial}{\partial \epsilon_{v}} \
  \left[-\left(2 \mu \tensor{\epsilon} + \lambda \tensor{I} \epsilon_{v} - \alpha \tensor{I} p \right) \right] d\Omega =
  \int_{\Omega} \trialscalar[u]_{i,j} \eqnannotate{\left(-\lambda \delta_{ij} \right)}{J_{f2}^{u \epsilon_{v}}} \basisscalar[\epsilon_{v}] d\Omega                                                                                                            \\
  %
  % JF_PU
  %
  J_F^{pu}                       & = \frac{\partial F^p}{\partial u} + t_{shift} \frac{\partial F^p}{\partial \dot{u}} = 0                                                                                                                                    \\
  %
  % JF_PP
  % Jf0pp
  J_F^{pp}                       & = \frac{\partial F^p}{\partial p} + t_{shift} \frac{\partial F^p}{\partial \dot{p}} =
  \int_{\Omega} \nabla \trialscalar[p] \cdot \frac{\partial}{\partial p} -\left[-\frac{\tensor{k}}{\mu_{f}} \left(\nabla p - \vec{f} \right) \right] \ d\Omega  +
  t_{shift}\int_{\Omega} \trialscalar[p] \frac{\partial}{\partial \dot{p}} \left[\alpha\dot{\epsilon}_{v} + \frac{\dot{p}}{M} - \gamma\left(\vec{x},t\right)\right] \ d\Omega                                                                                 \\
                                 & = \int_{\Omega} \nabla \psi_{trial}^ p \left(\frac{\tensor{k}}{\mu_{f}} \nabla \cdot \psi_{basis}^p \right) \ d\Omega +
  \int_{\Omega} \trialscalar[p] \left(t_{shift} \frac{1}{M}\right) \basisscalar[p] \ d\Omega                                                                                                                                                                  \\
                                 & = \int_{\Omega} \psi_{trial,k}^p \eqnannotate{\left(\frac{\tensor{k}}{\mu_{f}} \delta_{kl}\right)}{J_{f3}^{pp}} \psi_{basis,l}^p \ d\Omega +
  \int_{\Omega} \trialscalar[p] \eqnannotate{\left(t_{shift} \frac{1}{M}\right)}{J_{f0}^{pp}} \basisscalar[p] \ d\Omega                                                                                                                                       \\
  %
  % JF_PE
  % Jf0pe
  J_F^{p\epsilon_{v}}            & = \frac{\partial F^p}{\partial \epsilon_{v}} + t_{shift} \frac{\partial
    F^p}{\partial \dot{\epsilon_{v}}} = \int_{\Omega} \trialscalar[p] \eqnannotate{\left(t_{shift} \alpha \right)}{J_{f0}^{p\epsilon_{v}}}
  \basisscalar[\epsilon_{v}] \ d\Omega                                                                                                                                                                                                                        \\
  %
  % JF_EU
  % Jf1eu
  J_F^{\epsilon_{v}u}            & = \frac{\partial F^{\epsilon_{v}}}{\partial u} + t_{shift} \frac{\partial F^{\epsilon_{v}}}{\partial \dot{u}} =
  \int_{\Omega} \psi_{trial}^{\epsilon_{v}} \nabla \cdot \vec{\psi}_{basis}^u \ d\Omega = \int_{\Omega}
  \basisscalar[\epsilon_{v}] \eqnannotate{\left(\delta_{ij}\right)}{J_{f1}^{\epsilon_{v}u}}
  \basisscalar[u]_{i,j} \ d\Omega                                                                                                                                                                                                                             \\
  %
  % JF_EP
  %
  J_F^{\epsilon_{v}p}            & = \frac{\partial F^{\epsilon_{v}}}{\partial p} + t_{shift} \frac{\partial F^{\epsilon_{v}}}{\partial \dot{p}} = 0                                                                                                          \\
  %
  % JF_EE
  %
  J_F^{\epsilon_{v}\epsilon_{v}} & = \frac{\partial F^\epsilon_{v}}{\epsilon_{v}} + t_{shift} \frac{\partial F^{\epsilon_{v}}}{\partial \dot{\epsilon_{v}}} =
  \int_{\Omega} \basisscalar[\epsilon_{v}] \eqnannotate{\left(-1\right)}{J_{f0}^{\epsilon_{v}\epsilon_{v}}} \basisscalar[\epsilon_{v}] \ d\Omega
\end{align}

\subsection{Dynamic}

For compatibility with PETSc TS algorithms, we want to turn the second
order elasticity equation into two first order equations. We introduce
velocity as a unknown, $\vec{v}=\frac{\partial u}{\partial t}$, which
leads to a slightly different three field problem,
\begin{gather}
  % Solution
  \vec{s}^{T} = \left(\vec{u} \quad p \quad \vec{v}\right) \\
  % Displacement
  \frac{\partial \vec{u}}{\partial t} = \vec{v} \text{ in } \Omega \\
  % Pressure
  \frac{\partial \zeta(\vec{u},p)}{\partial t } - \gamma(\vec{x},t) + \nabla \cdot \vec{q}(p) = 0 \text{ in } \Omega \\
  % Velocity
  \rho_{b} \frac{\partial \vec{v}}{\partial t} = \vec{f}(\vec{x},t) + \nabla \cdot \tensor{\sigma}(\vec{u},p) \text{ in } \Omega \\
  % Neumann traction
  \tensor{\sigma} \cdot \vec{n} = \vec{\tau}(\vec{x},t) \text{ on } \Gamma_{\tau} \\
  % Dirichlet displacement
  \vec{u} = \vec{u}_{0}(\vec{x}, t) \text{ on } \Gamma_{u} \\
  % Neumann flow
  \vec{q} \cdot \vec{n} = q_{0}(\vec{x}, t) \text{ on } \Gamma_{q} \\
  % Dirichlet pressure
  p = p_{0}(\vec{x},t) \text{ on } \Gamma_{p}
\end{gather}

For compatibility with PETSc TS explicit time stepping algorithms, we
need the left hand side to be $F = (t,s,\dot{s}) = \dot{s}$. We
replace the variation of fluid content variable, $\zeta$, with its
definition in the conservation of fluid mass equation and solve for
the rate of change of pressure,
\begin{gather}
  \frac{\partial}{\partial t}\left[\alpha \epsilon_{v} + \frac{p}{M}\right] - \gamma\left(\vec{x},t\right) + \nabla \cdot \vec{q} = 0 \\
  \alpha \dot{\epsilon}_{v} + \frac{\dot{p}}{M} - \gamma \left(\vec{x},t\right) + \nabla \cdot \vec{q} = 0 \\
  \frac{\dot{p}}{M} = \gamma \left(\vec{x},t \right) - \alpha \dot{\epsilon}_{v} -\nabla \cdot \vec{q} \\
  \frac{\dot{p}}{M} = \gamma \left(\vec{x},t \right) - \alpha \left( \nabla \cdot \dot{\vec{u}} \right) -\nabla \cdot \vec{q}.
\end{gather}
We write the volumetric strain in terms of displacement, because this
dynamic formulation does not include the volumetric strain as an
unknown.

Using trial functions $\trialvec[u]$, $\trialscalar[p]$, and $\trialvec[v]$, and incorporating the
Neumann boundary conditions, the weak form may be written as:
\begin{align}
  % Displacement
  \int_{\Omega} \trialvec[u] \cdot \left( \frac{\partial \vec{u}}{\partial t} \right)d \Omega   & = \int_{\Omega} \trialvec[u] \cdot \left( \vec{v} \right) d \Omega          \\
  % Pressure
  \int_{\Omega} \trialscalar[p] \left( \frac{1}{M}\frac{\partial p}{\partial t} \right) d\Omega & =
  \int_{\Omega} \trialscalar[p] \left[\gamma(\vec{x},t) - \alpha \left(\nabla \cdot \dot{\vec{u}}\right) \right]  + \nabla \trialscalar[p] \cdot \vec{q}(p) \ d\Omega +
  \int_{\Gamma_q} \trialscalar[p] (-q_0(\vec{x},t)) \ d\Gamma,                                                                                                                \\
  % Velocity
  \int_\Omega \trialvec[v] \cdot \left( \rho_{b} \frac{\partial
    \vec{v}}{\partial t} \right) \,
  d\Omega                                                                                       & = \int_\Omega \trialvec[v] \cdot \vec{f}(\vec{x},t) + \nabla \trialvec[v] :
  -\tensor{\sigma} (\vec{u},p_f) \, d\Omega + \int_{\Gamma_\tau} \trialvec[u]
  \cdot \vec{\tau}(\vec{x},t) \, d\Gamma.
\end{align}



\subsubsection{Residual Pointwise Functions}

With explicit time stepping the PETSc TS assumes the LHS is $\dot{s}$ , so we only need the RHS residual functions:

\begin{align}
  % Displacement
  G^u(t,s) & = \int_{\Omega} \trialvec[u] \cdot \eqnannotate{\vec{v}}{\vec{g}_0^u} d \Omega,                                                                                                                                                                                                   \\
  % Pressure
  G^p(t,s) & = \int_\Omega \trialscalar[p] \eqnannotate{\left(\gamma(\vec{x},t) - \alpha (\nabla \cdot \dot{\vec{u}})\right)}{g^p_0} + \nabla \trialscalar[p] \cdot \eqnannotate{\vec{q}(p_f)}{\vec{g}^p_1} \, d\Omega
  + \int_{\Gamma_q} \trialscalar[p] (\eqnannotate{-q_0(\vec{x},t)}{g^p_0}) \, d\Gamma,                                                                                                                                                                                                         \\
  % Velocity
  G^v(t,s) & = \int_\Omega \trialvec[v] \cdot \eqnannotate{\vec{f}(\vec{x},t)}{\vec{g}^v_0} + \nabla \trialvec[v] :\eqnannotate{-\tensor{\sigma}(\vec{u},p_f)}{\tensor{g}^v_1} \, d\Omega + \int_{\Gamma_\tau} \trialvec[u] \cdot \eqnannotate{\vec{\tau}(\vec{x},t)}{\vec{g}^v_0} \, d\Gamma.
\end{align}

\subsubsection{Jacobians Pointwise Functions}

These are the pointwise functions associated with $M_{u}$, $M_{p}$,
and $M_{v}$ for computing the lumped LHS Jacobian. We premultiply the
RHS residual function by the inverse of the lumped LHS Jacobian while
$s_\mathit{tshift}$ remains on the LHS with $\dot{s}$. As a result,
we use LHS Jacobian pointwise functions, but set $s_\mathit{tshift} = 1$. The
LHS Jacobians are:
\begin{align}
  % Displacement
  M_{u} & = J_F^{uu} = \frac{\partial F^u}{\partial u} + s_{tshift} \frac{\partial F^u}{\partial \dot{u}} =
  \int_{\Omega} \trialscalar[u]_{i} \eqnannotate{s_{tshift} \delta_{ij}}{J^{uu}_{f0}} \basisscalar[u]_{j} \, d \Omega    \\
  % Pressure
  M_{p} & = J_F^{pp} = \frac{\partial F^p}{\partial p} + s_{tshift} \frac{\partial F^p}{\partial \dot{p}} =
  \int_{\Omega} \trialscalar[p] \eqnannotate{\left(s_{tshift} \frac{1}{M}\right)}{J_{f0}^{pp}} \basisscalar[p] \ d\Omega \\
  % Velocity
  M_{v} & = J_F^{vv} = \frac{\partial F^v}{\partial v} + s_{tshift} \frac{\partial F^v}{\partial \dot{v}} =
  \int_{\Omega} \trialscalar[v]_{i}\eqnannotate{\rho_{b}(\vec{x}) s_{tshift} \delta_{ij}}{J^{vv}_{f0}} \basisscalar[v]_{j} \  d \Omega
\end{align}
