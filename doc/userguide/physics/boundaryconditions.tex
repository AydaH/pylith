\section{Boundary Conditions}
\label{src:boundary:conditions}

\subsection{Assigning Boundary Conditions}

There are three basic steps in assigning a specific boundary condition
to a portion of the domain.
\begin{enumerate}
\item Create sets of vertices in the mesh generation process for each boundary
  condition.
\item Set the parameters for each boundary condition group using
  \filename{cfg} files and/or command line
  arguments.
\item Specify the spatial variation in parameters for the boundary
  condition using a spatial database file.
\end{enumerate}

\subsection{Creating Sets of Vertices}

The procedure for creating sets of vertices differs depending on the
mesh generator. For meshes specified using the PyLith mesh ASCII
format, the sets of vertices are specified using groups (see Appendix
\vref{sec:format:MeshIOAscii}).  In CUBIT/Trelis the groups of
vertices are created using nodesets. Similarly, in LaGriT, psets are
used. Note that we chose to associate boundary conditions with groups
of vertices because nearly every mesh generation package supports
associating a string or integer with groups of vertices.  Note also
that we currently associate boundary conditions with string
identifiers, so even if the mesh generator uses integers, the name is
specified as the digits of the integer value. Finally, note that every
vertex set that ultimately is associated with a boundary condition on
a cell face (e.g., Neumann boundary conditions and fault interface
conditions) must correspond to a simply-connected surface.

\subsection{Arrays of Boundary Condition Components}

A dynamic array of boundary condition components associates a name
(string) with each boundary condition. The default boundary condition
for each component in the array is the \object{DirichletTimeDependent}
object.  Other boundary conditions can be bound to the named items in
the array via a \filename{cfg} file or the command line.  The
parameters for the boundary condition are set using the name of the
boundary condition.

\begin{cfg}[Array of boundary conditions in a \filename{cfg} file]
<h>[pylithapp.problem]</h>
# Array of four boundary conditions
<p>bc</p> = [x_neg, x_pos, y_pos, z_neg]

# Default boundary condition is DirichletBC
# Keep default value for x_neg and x_pos
<f>bc.y_pos</f> = pylith.bc.AbsorbingDampers
<f>bc.z_neg</f> = pylith.bc.NeumannTimeDependent
\end{cfg}

\section{Time-Dependent Boundary Conditions}
\label{sec:boundary:conditions:time:dependent}

Several boundary conditions use a common formulation for the spatial
and temporal variation of the boundary condition parameters,
\begin{equation}
f(\vec{x})=f_{0}(\vec{x})+\dot{f}_{0}(\vec{x})(t-t_{0}(\vec{x}))+f_{1}(\vec{x})a(t-t_{1}(\vec{x})),
\end{equation}
where $f(\vec{x})$ may be a scalar or vector parameter, $f_{0}(\vec{x})$
is a constant value, $\dot{f}_{0}(\vec{x})$ is a constant rate of
change in the value, $t_{0}(\vec{x})$ is the onset time for the constant
rate of change, $f_{1}(\vec{x})$ is the amplitude for the temporal
modulation, $a(t)$ is the variation in amplitude with time, $t_{1}(\vec{x})$
is the onset time for the temporal modulation, and $\vec{x}$ is the
position of a location in space. This common formulation permits easy
specification of a scalar or vector with a constant value, constant
rate of change of a value, and/or modulation of a value in time. One
can specify just the initial value, just the rate of change of the
value (along with the corresponding onset time), or just the modulation
in amplitude (along with the corresponding temporal variation and
onset time), or any combination of the three.

\subsection{Time-Dependent Dirichlet Boundary Conditions (\protect\object{DirichletTimeDependent})}

Dirichlet boundary conditions in PyLith prescribe the a solution
subfield on a subset of the vertices of the finite-element
mesh. Currently, these constraints are required to be associated with
vertices on a simply-connected boundary surface.

The properties and components common to both the \object{DirichletTimeDependent} and
\object{DirichletBoundary} boundary conditions are:
\begin{inventory}
\propertyitem{label}{Label of the group of vertices associated with the boundary
  condition default="");}
\propertyitem{field}{Solution subfield associated with boundary condition (default=displacement);}
\facilityitem{db\_auxiliary\_field}{Database for boundary condition
  parameter values (default=\object{SimpleDB});}
\facilityitem{observers}{Observers of boundary condition, e.g., output
  (default=[\object{PhysicsObserver}]):}
\propertyitem{constrained\_dof}{Array of degrees of freedom to be fixed (first degree
  of freedom is 0, default=[]);}
\propertyitem{use\_initial}{Use initial term in time-dependent
  expression (default=True);}
\propertyitem{use\_rate}{Use rate term in time-dependent
  expression (default=False);}
\propertyitem{use\_time\_history}{Use time history term in time-dependent
  expression (default=False);}
\facilityitem{time\_history}{Time history database with normalized
  amplitude as a function of time (default=\object{TimeHistoryDB});
  and}
\facilityitem{auxiliary\_subfields}{Discretization of auxiliary subfields.}
\end{inventory}


\begin{cfg}[\object{DirichletTimeDependent} parameters in a \filename{cfg} file]
<h>[pylithapp.problem]</h>
<p>bc</p> = [mybc]

<h>[pylithapp.problem.bc.mybc]</h>
# Constrain the z-displacment (2) on a boundary associated with the `group A' vertices.
<p>label</p> = group A
<p>field</p> = displacement
<p>constrained_dof</p> = [2]
<p>observers.observer.writer.filename</p> = output/step02-groupA.h5

<f>db_auxiliary_field</f> = spatialdata.spatialdb.SimpleDB
<p>db_initial.iohandler.filename</p> = displacement.spatialdb
# Use linear interpolation
<p>db_initial.query_type</p> = linear

# Set basis order to its default value
auxiliary_subfields.initial_amplitude.basis_order = 1
\end{cfg}

\subsubsection{Dirichlet Boundary Condition Spatial Database Files}

The spatial database files for the Dirichlet boundary condition specify
the parameters for the time-dependent expression.

\important{The spatial database files for Dirichlet boundary
  conditions must contain values for all degrees of freedom (x and y
  for 2-D, and x, y, and z for 3-D even if they are not
  constrained. This limitation is imposed by the \object{DMPlex}
  interface.}

\begin{table}[htbp]
  \caption{Values in the spatial databases used for Dirichlet boundary conditions.}
  \begin{tabular}{lp{4in}}
    \toprule
    \thead{Flag} & \thead{Required Values}\\
    \midrule
    \property{use\_initial} & initial\_amplitude\_x, initial\_amplitude\_y, initial\_amplitude\_z\\
    \property{use\_rate} & rate\_start\_time, rate\_amplitude\_x, rate\_amplitude\_y, rate\_amplitude\_z\\
    \property{use\_time\_history} & time\_history\_start, time\_history\_amplitude\_x, time\_history\_amplitude\_y, time\_history\_amplitude\_z \\
    \bottomrule
  \end{tabular}
\end{table}


\subsection{Neumann Boundary Conditions}

Neumann boundary conditions are surface tractions applied over a boundary. As with the DirichletTimeDependent condition, each Neumann
boundary condition can only be applied to a simply-connected surface.
The surface over which the tractions are applied always has a spatial
dimension that is one less than the dimension of the finite-element
mesh. 

%\important{In the small (finite) strain formulation, we assume that
%  the normal and shear tractions are prescribed in terms of the
%  undeformed configuration as described in section
%  \vref{sec:small:strain:formulation}.}

The Neumann boundary condition properties and facilities are:
\begin{inventory}
\propertyitem{label}{Label of the group of vertices associated with the boundary
  condition default="");}
\propertyitem{field}{Solution subfield associated with boundary condition (default=displacement);}
\facilityitem{db\_auxiliary\_field}{Database for boundary condition
  parameter values (default=\object{SimpleDB});}
\facilityitem{observers}{Observers of boundary condition, e.g., output
  (default=[\object{PhysicsObserver}]):}
\propertyitem{scale\_name}{Type of scale for nondimensionalizing values (default="pressure");}
\propertyitem{use\_initial}{Use initial term in time-dependent
  expression (default=True);}
\propertyitem{use\_rate}{Use rate term in time-dependent
  expression (default=False);}
\propertyitem{use\_time\_history}{Use time history term in time-dependent
  expression (default=False);}
\propertyitem{ref\_dir\_1}{First choice for reference direction to discriminate among tangential directions in 3-D (default=[0,0,1]);}
\propertyitem{ref\_dir\_2}{Second choice for reference direction to discriminate among tangential directions in 3-D (default=[0,1,0]);}
\facilityitem{time\_history}{Time history database with normalized
  amplitude as a function of time (default=\object{TimeHistoryDB});
  and}
\facilityitem{auxiliary\_subfields}{Discretization of auxiliary subfields.}
\end{inventory}

The components are specified in the local normal/tangential coordinate
system for the boundary. Ambiguities in specifying the shear
(tangential) tractions in 3-D problems are resolved using the
\property{ref\_dir\_1} and \property{ref\_dir\_2} properties. The
first tangential direction is $\vec{z} \times \vec{r}_1$ unless these
are colinear, then $\vec{r}_2$ (ref\_dir\_2) is used. The second
tangential direction is $\vec{n} \times \vec{t}_1$.

\begin{cfg}[\object{Neumann} parameters in a \filename{cfg} file]
<h>[pylithapp.problem]</h>
<f>bc</f> = [x_neg, y_neg, x_pos, y_pos]
<f>bc.x_neg</f> = pylith.bc.DirichletTimeDependent
<f>bc.y_neg</f> = pylith.bc.DirichletTimeDependent
<f>bc.x_pos</f> = pylith.bc.NeumannTimeDependent
<f>bc.y_pos</f> = pylith.bc.NeumannTimeDependent

<h>[pylithapp.problem.bc.x_pos]</h>
<p>label</p> = boundary_xpos
<f>db_auxiliary_field</f> = spatialdata.spatialdb.UniformDB
<p>db_auxiliary_field.label</p> = Neumann BC +x edge
<p>db_auxiliary_field.values</p> = [initial_amplitude_tangential, initial_amplitude_normal]
<p>db_auxiliary_field.data</p> = [+4.5*MPa, 0*MPa]

<p>observers.observer.writer.filename</p> = output/step03_sheardisptract-bc_xpos.h5

<h>[pylithapp.problem.bc.y_pos]</h>
<p>label</p> = boundary_ypos
<f>db_auxiliary_field</f> = spatialdata.spatialdb.UniformDB
<p>db_auxiliary_field.label</p> = Neumann BC +y edge
<p>db_auxiliary_field.values</p> = [initial_amplitude_tangential, initial_amplitude_normal]
<p>db_auxiliary_field.data</p> = [-4.5*MPa, 0*MPa]

<p>observers.observer.writer.filename</p> = output/step03_sheardisptract-bc_ypos.h5
\end{cfg}


\subsubsection{Neumann Boundary Condition Spatial Database Files}

The spatial database file the auxiliary subfields for the Neumann
boundary condition specify the parameters for the time-dependent
expressions.

\begin{table}[htbp]
  \caption{Values in the auxiliary field spatial database used for Neumman boundary conditions.}
  \begin{tabular}{llp{4in}}
    \toprule
    \thead{Dimension} & \thead{Flag} & \thead{Required Values}\\
    \midrule
    \multirow{3}{*}{2}
      & \property{use\_initial} & initial\_amplitude\_normal, initial\_amplitude\_tangential \\
      & \property{use\_rate} & rate\_start\_time, rate\_amplitude\_normal, rate\_amplitude\_tangential\\
      & \property{use\_time\_history} & time\_history\_start, time\_history\_amplitude\_normal, time\_history\_amplitude\_tangential \\
    \multirow{3}{*}{3}
      & \property{use\_initial} & initial\_amplitude\_normal, initial\_amplitude\_tangential\_1, initial\_amplitude\_tangential\_2 \\
      & \property{use\_rate} & rate\_start\_time, rate\_amplitude\_normal, rate\_amplitude\_tangential\_1, rate\_amplitude\_tangential\_2 \\
      & \property{use\_time\_history} & time\_history\_start, time\_history\_amplitude\_normal, time\_history\_amplitude\_tangential\_1, time\_history\_amplitude\_tangential\_2 \\
    \bottomrule
  \end{tabular}
\end{table}

% \subsection{Point Force Boundary Conditions}

% Point force boundary conditions in PyLith prescribe the application
% of point forces to a subset of the vertices of the finite-element
% mesh. While point force boundary conditions can be applied to any
% vertex, usually they are applied to vertices on the lateral, top,
% and bottom boundaries of the domain.

% \subsubsection{Point Force Parameters}

% The properties and components for the \object{PointForce} boundary
% condition are:
% \begin{inventory}
% \propertyitem{label}{Label of the group of vertices associated with the boundary condition.}
% \propertyitem{bc\_dof}{Array of degrees of freedom to which forces are applied (first degree of freedom is 0).}
% \end{inventory}

% \begin{cfg}[\object{PointForce} parameters in a \filename{cfg} file]
% <h>[pylithapp.problem]</h>
% <p>bc</p> = [mybc]
% <f>bc.mybc</f> = pylith.bc.PointForce

% <h<[pylithapp.problem.bc.mybc]</h>
% <p>label</p> = group A 
% <p>bc_dof</p> = [2] ; force in z direction
% <f>db_initial</f> = spatialdata.spatialdb.SimpleDB
% <p>db_initial.iohandler.filename</p> = force\_A.spatialdb
% <p>db_initial.query_type</p> = nearest ; change query type to nearest point algorithm
% <f>db_rate</f> = spatialdata.spatialdb.UniformDB
% <p>db_rate.values</p> = [force-rate-z]
% <p>db_rate.data</p> = [1.0e+5*newton/s]
% \end{cfg}
% We have created an array with one boundary condition, mybc. The group
% of vertices associated with the boundary condition is group A. For
% the database associated with the constant force, we use a SimpleDB.
% We set the filename and query type for the database. For the rate
% of change of values, we use a \object{UniformDB} and specify the rate of change
% in the force to be 1.0e+5 Newton/s. See Section \vref{sec:spatial:databases}
% for a discussion of the different types of spatial databases available.

% \subsubsection{Point Force Spatial Database Files}

% The spatial database files for the point force boundary condition specify
% the forces applied. 

% \begin{table}[htbp]
%   \caption{Values in the spatial databases used for point force boundary conditions.}
%   \begin{tabular}{lp{4in}}
%     \textbf{Spatial database} & \textbf{Name in Spatial Database}\\
%     \hline 
%     \facility{db\_initial} & \texttt{force-x, force-y, force-z}\\
%     \facility{db\_rate} & \texttt{force-rate-x, force-rate-y, force-rate-z, rate-start-time}\\
%     \facility{db\_change} & \texttt{force-x, force-y, force-z, change-start-time}\\
%     \hline 
%   \end{tabular}
% \end{table}


\section{Absorbing Boundary Conditions (\protect\object{AbsorbingDampers})}
\label{sec:absorbing:boundaries}

This \object{AbsorbingDampers} boundary condition attempts to prevent
seismic waves reflecting off of a boundary by placing simple dashpots
on the boundary. Normally incident dilatational and shear waves are
perfectly absorbed. Waves incident at other angles are only partially
absorbed. This boundary condition is simpler than a perfectly matched
layer (PML) boundary condition but does not perform quite as well,
especially for surface waves. If the waves arriving at the absorbing
boundary are relatively small in amplitude compared to the amplitudes
of primary interest, this boundary condition gives reasonable results.

The \object{AbsorbingDampers} boundary condition properties and components are:
\begin{inventory}
\propertyitem{label}{Label of the group of vertices associated with the boundary
  condition default="");}
\propertyitem{field}{Solution subfield associated with boundary condition (default=displacement);}
\facilityitem{db\_auxiliary\_field}{Database for boundary condition
  parameter values (default=\object{SimpleDB});}
\facilityitem{observers}{Observers of boundary condition, e.g., output
  (default=[\object{PhysicsObserver}]):}
\end{inventory}

The auxiliary subfields in this case are the bulk rheology properties
for an isotrpoic, linear elastic material (density, vs (S-wave speed),
and vp (P-wave speed).

\subsection{Finite-Element Implementation of Absorbing Boundary}

\todo{brad}{Move this to the multiphysics implementation section.}

Consider a plane wave propagating at a velocity $c$. We can write
the displacement field as
\begin{equation}
\vec{u}(\vec{x},t)=\vec{u^{t}}(t-\frac{\vec{x}}{c}),
\end{equation}
where $\vec{x}$ is position, $t$ is time, and $\vec{u^{t}}$ is
the shape of the propagating wave. For an absorbing boundary we want
the traction on the boundary to be equal to the traction associated
with the wave propagating out of the domain. Starting with the expression
for the traction on a boundary, $T_{i}=\sigma_{ij}n_{j},$ and using
the local coordinate system for the boundary $s_{h}s_{v}n,$ where
$\vec{n}$ is the direction normal to the boundary, $\overrightarrow{s}_{h}$
is the horizontal direction tangent to the boundary, and $\overrightarrow{s}_{v}$
is the vertical direction tangent to the boundary, the tractions on
the boundary are
\begin{gather}
T_{s_{h}}=\sigma_{s_{h}n}\\
T_{s_{v}}=\sigma_{s_{v}n}\\
T_{n}=\sigma_{nn}.
\end{gather}
In the case of a horizontal boundary, we can define an auxiliary direction
in order to assign unique tangential directions. For a linear elastic
isotropic material, $\sigma_{ij}=\lambda\epsilon_{kk}\delta_{ij}+2\mu\epsilon_{ij},$
and we can write the tractions as 
\begin{gather}
T_{s_{h}}=2\mu\epsilon_{s_{h}n}\\
T_{s_{v}}=2\epsilon_{s_{v}n}\\
T_{n}=(\lambda+2\mu)\epsilon_{nn}+\lambda(\epsilon_{s_{h}s_{h}}+\epsilon_{s_{v}s_{v}}).
\end{gather}
For infinitesimal strains, $\epsilon_{ij}=\frac{1}{2}(u_{i,j}+u_{j,i})$
and we have
\begin{gather}
\epsilon_{s_{h}n}=\frac{1}{2}(u_{s_{h},n}+u_{n,s_{h}})\\
\epsilon_{s_{v}n}=\frac{1}{2}(u_{s_{v},n}+u_{n,s_{v}})\\
\epsilon_{nn}=u_{n,n}.
\end{gather}
For our propagating plane wave, we recognize that
\begin{equation}
\frac{\partial\vec{u^{t}}(t-\frac{\vec{x}}{c})}{\partial x_{i}}=-\frac{1}{c}\frac{\partial\vec{u^{t}}(t-\frac{\vec{x}}{c})}{\partial t},
\end{equation}
so that our expressions for the tractions become
\begin{gather}
T_{s_{h}}=-\frac{\mu}{c}\left(\frac{\partial u_{s_{h}}^{t}(t-\frac{\vec{x}}{c})}{\partial t}+\frac{\partial u_{n}^{t}(t-\frac{\vec{x}}{c})}{\partial t}\right),\\
T_{s_{v}}=-\frac{\mu}{c}\left(\frac{\partial u_{s_{v}}^{t}(t-\frac{\vec{x}}{c})}{\partial t}+\frac{\partial u_{n}^{t}(t-\frac{\vec{x}}{c})}{\partial t}\right).
\end{gather}
For the normal traction, consider a dilatational wave propagating
normal to the boundary at speed $v_{p}$; in this case $u_{s_{h}}=u_{s_{v}}=0$
and $c=v_{p}$. For the shear tractions, consider a shear wave propagating
normal to the boundary at speed $v_{s}$; we can decompose this into
one case where $u_{n}=u_{s_{v}}=0$ and another case where $u_{n}=u_{s_{h}}=0$,
with $c=v_{s}$ in both cases. We also recognize that $\mu=\rho v_{s}^{2}$
and $\lambda+2\mu=\rho v_{p}^{2}$. This leads to the following expressions
for the tractions:
\begin{gather}
T_{s_{h}}=-\rho v_{s}\frac{\partial u_{s_{h}}^{t}(t-\frac{\vec{x}}{c})}{\partial t}\\
T_{s_{v}}=-\rho v_{s}\frac{\partial u_{v}^{t}(t-\frac{\vec{x}}{c})}{\partial t}\\
T_{n}=-\rho v_{p}\frac{\partial u_{n}^{t}(t-\frac{\vec{x}}{c})}{\partial t}
\end{gather}
We write the weak form of the boundary condition as
\[
\int_{S_{T}}T_{i}\phi_{i}\, dS=\int_{S_{T}}-\rho c_{i}\frac{\partial u_{i}}{\partial t}\phi_{i}\, dS,
\]
where $c_{i}$ equals $v_{p}$ for the normal traction and $v_{s}$
for the shear tractions, and $\phi_{i}$ is our weighting function.
We express the trial solution and weighting function as linear combinations
of basis functions,
\begin{gather}
u_{i}=\sum_{m}a_{i}^{m}N^{m},\\
\phi_{i}=\sum_{n}c_{i}^{n}N^{n}.
\end{gather}
Substituting into our integral over the absorbing boundaries yields
\begin{equation}
\int_{S_{T}}T_{i}\phi_{i}\, dS=\int_{S_{T}}-\rho c_{i}\sum_{m}\dot{a}_{i}^{m}N^{m}\sum_{n}c_{i}^{n}N^{n}\, dS.
\end{equation}
In the derivation of the governing equations, we recognized that the
weighting function is arbitrary, so we form the residual by setting
the terms associated with the coefficients $c_{i}^{n}$ to zero,

\begin{equation}
r_{i}^{n}=\sum_{\text{tract cells}}\sum_{\text{quad pts}}-\rho(x_{q})c_{i}(x_{q})\sum_{m}\dot{a}_{i}^{m}N^{m}(x_{q})N^{n}(x_{q})w_{q}|J_{cell}(x_{q})|,
\end{equation}
 where $x_{q}$ are the coordinates of the quadrature points, $w_{q}$
are the weights of the quadrature points, and $|J_{cell}(x_{q})|$
is the determinant of the Jacobian matrix evaluated at the quadrature
points associated with mapping the reference cell to the actual cell.

The appearance of velocity in the expression for the residual means
that the absorbing dampers also contribute to the system Jacobian
matrix. Using the central difference method, the velocity is written
in terms of the displacements,
\begin{equation}
\dot{u}_{i}(t)=\frac{1}{2\Delta t}(u_{i}(t+\Delta t)-u_{i}(t-\Delta t)).
\end{equation}
Expressing the displacement at time $t+\Delta t$ in terms of the
displacement at time $t$ ($u_{i}(t)$) and the increment in the displacement
at time $t$ ($du_{i}(t)$) leads to
\begin{equation}
\dot{u}_{i}(t)=\frac{1}{2\Delta t}(du_{i}(t)+u_{i}(t)-u_{i}(t-\Delta t))
\end{equation}
The terms contributing to the system Jacobian are associated with
the increment in the displacement at time $t$. Substituting into
the governing equations and isolating the term associated with the
increment in the displacement at time t yields
\begin{equation}
A_{ij}^{nm}=\sum_{\text{tract cells}}\sum_{\text{quad pts}}\delta_{ij}\frac{1}{2\Delta t}\rho(x_{q})v_{i}(x_{q})N^{m}(x_{q})N^{n}(x_{q})w_{q}|J_{cells}(x_{q})|,
\end{equation}
where $A_{ij}^{mn}$ is an $nd$ by $md$ matrix ($d$ is the dimension
of the vector space), $m$ and $n$ refer to the basis functions and
$i$ and $j$ are vector space components.



% End of file
