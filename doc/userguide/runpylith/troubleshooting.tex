\section{Troubleshooting}
\label{sec:troubleshooting}

\subsection{Tips and Hints For Running PyLith}
\begin{itemize}
\item Examine the examples for a problem similar to the one you want to
run and dissect it in detail.
\item Start with a uniform-resolution coarse mesh to debug the problem setup.
Increase the resolution as necessary to resolve the solution fields
of interest (resolving stresses/strains may require a higher resolution
than that for resolving displacements).
\item Merge materials using the same material model. This will result in
only one VTK or HDF5 file for each material model rather than several
files.
\item The rate of convergence in quasi-static (implicit) problems can sometimes
be improved by renumbering the vertices in the finite-element mesh
to reduce the bandwidth of the sparse matrix. PyLith can use the reverse
Cuthill-McKee algorithm to reorder the vertices and cells.
\item If you encounter errors or warnings, run \filename{pylithinfo} or use
the \commandline{-{}-help}, \commandline{-{}-help-components}, and \commandline{-{}-help-properties}
command-line arguments when running PyLith to check the parameters
to make sure PyLith is using the parameters you intended.
\item Use the \commandline{-{}-petsc.log\_}view, \commandline{-{}-petsc.ksp\_monitor},
\commandline{-{}-petsc.ksp\_view}, \commandline{-{}-petsc.ksp\_converged\_reason}, and \commandline{-{}-petsc.snes\_converged\_reason}
command-line arguments (or set them in a parameter file) to view PyLith
performance and monitor the convergence.
\item Turn on the journals (see the examples) to monitor the progress of
the code.
\end{itemize}

\subsection{Troubleshooting}
\label{sec:Troubleshooting}

Consult the PyLith category in the CIG community forum
(\url{https://community.geodynamics.org}) and the PyLith FAQ webpage
(\url{https://wiki.geodynamics.org/software:pylith:help:hints}) which
contains a growing list of common problems and their corresponding
solutions.

\subsubsection{Import Error and Missing Library}
\begin{shell}
ImportError: liblapack.so.2: cannot open shared object file: No such file or directory
\end{shell}

PyLith cannot find one of the libraries. You need to set up your environment
variables (e.g., PATH, PYTHONPATH, and LD\_LIBRARY\_PATH) to match
your installation. If you are using the PyLith binary on Linux or
Mac OS X, run the command \filename{source setup.sh }in the directory
where you unpacked the distribution. This will set up your environment
variables for you. If you are building PyLith from source, please
consult the instructions for building from source.

\subsubsection{Unrecognized Property 'p4wd'}

\begin{shell}
-- pyre.inventory(error) } \\
-- p4wd <- 'true' } \\
-- unrecognized property 'p4wd' } \\
>> command line:: } \\
-- pyre.inventory(error) } \\
-- p4pg <- 'true' } \\
-- unrecognized property ' p4pg'}
\end{shell}
Verify that the filename{mpirun} command included in the PyLith package is
the first one on your PATH: \filename{which mpirun}. If it is not, adjust your PATH environment variable accordingly.

\subsubsection{Detected zero pivor in LU factorization}

\begin{shell}
-- Solving equations.
[0] PETSC ERROR: ----------------
Error Message -------------------------------
[0] PETSC ERROR: Detected zero pivot in LU factorization
see http://www.mcs.anl.gov/petsc/petsc-as/documentation/faq.html\#ZeroPivot!
\end{shell}

This usually occurs when the null space of the system Jacobian is
nonzero, such as the case of a problem without Dirichlet boundary
conditions on any boundary. If this arises when using the split fields
and algebraic multigrid preconditioning, and no additional Dirichlet
boundary conditions are desired, then the workaround is to revert
to using the Additive Schwarz preconditioning without split fields
as discussed in Section \vref{sec:petsc:options}.

\subsubsection{Bus Error}

This often indicates that PyLith is using incompatible versions of
libraries. This can result from changing your environment variables
after configuring or installing PyLith (when building from source) or
from errors in setting the environment variables (\filename{PATH},
\filename{LD\_LIBRARY\_PATH}, and \filename{PYTHONPATH}). If the
former case, simply reconfigure and rebuild PyLith. In the latter
case, check your environment variables (order matters!) to make sure
PyLith finds the desired directories before system directories.

\subsubsection{Segmentation Fault}

A segmentation fault usually results from an invalid read/write to
memory. It might be caused by an error that wasn't trapped or a bug in
the code. Please report these cases so that we can fix these problems
(either trap the error and provide the user with an informative error
message, or fix the bug). If this occurs with any of the problems
distributed with PyLith, simply submit a bug report (see Section
\vref{sec:help}) indicating which problem you ran and your
platform.

\important{Also note that PETSc will often report errors as
semgentation faults even if the underlying problem is not an invalid
read/write. If you see PETSc reporting a segmentation fault, examine
the output carefully for other error messages that indicate the real
issue is something else.}

If the crash occurs for a problem you created, it is a great
help if you can try to reproduce the crash with a very simple problem
(e.g., adjust the boundary conditions or other parameters of one of
the examples to reproduce the segmentation fault). Submit a bug report
along with log files showing the backtrace from a debugger (e.g., gdb)
and the valgrind log file (only available on Linux platforms).  You
can generate a backtrace using the debugger by using the
\commandline{-{}-petsc.start\_in\_debugger} command-line argument:
\begin{shell}
$ pylith [..args..] --petsc.start_in_debugger
(gdb) continue
(gdb) backtrace
\end{shell}

To use valgrind to detect the memory error, first go to your working
directory and run the problem with \commandline{-{}-launcher.dry}:
\begin{shell}
$ pylith [..args..] --launcher.dry
\end{shell}

Instead of actually running the problem, this causes PyLith to dump
the mpirun/mpiexec command it will execute. Copy and paste this command
into your shell so you can run it directly. Insert the full path to
valgrind before the full path to mpinemesis and tell valgrind to use
a log file:
\begin{shell}
$ mpirun /path/to/valgrind --log-file=valgrind-log /path/to/mpinemesis --pyre-start
  [..lots of junk..]
\end{shell}

% End of file
