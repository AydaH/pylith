% -*- LaTeX -*-
%
% ----------------------------------------------------------------------
%
% Brad T. Aagaard, U.S. Geological Survey
% Charles A. Williams, GNS Science
% Matthew G. Knepley, University of Chicago
%
% This code was developed as part of the Computational Infrastructure
% for Geodynamics (http://geodynamics.org).
%
% Copyright (c) 2010-2017 University of California, Davis
%
% See COPYING for license information.
%
% ----------------------------------------------------------------------
%
\documentclass{pylithdoc}

\title{Style Guide for PyLith Documentation}

\begin{document}

\userwarning{This is a warning.}
\usertip{This is a tip, helpful hint, or suggestion.}
\important{This is something important.}

\todo{brad}{This is something for Brad to do. 

  \todocomment{matt}{This is a comment directed towards Matt.}
}

Use the {\tt shell} environment when listing commands to 
be run in a Unix shell.
% Example use of shell environment.
\begin{shell}[Bash shell]
# This is a comment.
$ ls -l
\end{shell}

Use the {\tt cfg} environment when demonstrating how to set parameters
in a {{\tt .cfg} file. Use {\tt <h>} and {\tt </h>} to delimit section
  headers, {\tt <p>} and {\tt </p>} to delimit properties, and {\tt <f>}
  and {\tt </f>} to delimit facilities.
% Example use of cfg environment
\begin{cfg}[Excerpt from \filename{pylithapp.cfg}]
# This is a comment.
<h>[pylithapp.problem]</h>
<p>timestep</p> = 2.0*s ; Time step comment.
<f>bc</f> = [x_pos, x_neg]
\end{cfg}

Example of \filename{.cfg} environment without title.
% Example use of cfg environment
\begin{cfg}
# This is a comment.
<h>[pylithapp.problem]</h>
<p>timestep</p> = 2.0*s ; Time step comment.
<f>bc</f> = [x_pos, x_neg]
\end{cfg}

Use the {\tt inventory} environment when listing the Pyre properties
and facilities of an object. Use the {\tt
  \textbackslash propertyitem\{NAME\}\{DESCRIPTION\}} command for properties and {\tt
  \textbackslash facilityitem\{NAME\}\{DESCRIPTION\}} for facilities.
% Example use of inventory environment
\begin{inventory}
\propertyitem{time\_step}{Time step in simulation.}
\facilityitem{bc}{Array of boundary conditions.}
\end{inventory}


\begin{python}[ParaView Python shell]
>>> EXODUS_FILE = "/home/johndoe/pylith/examples/3d/subduction/mesh/mesh_tet.exo"
\end{python}


\section{My Feature}
\newfeature{v2.1.4}

Use the {\tt \textbackslash newfeature\{vX.X.X\}} command to indicate when a
feature was implemented (if known).

\section{Examples}

\input{./examples/example_table}

\end{document}
